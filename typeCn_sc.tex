\section{The p-adic symplectic group}

Let $G=\Sp_{2n}(F)$ be the simply connected $p$-adic group of type $C_n$, whose complex dual group $G^\vee=\SO_{2n+1}(\CC)$. We note that $G$ has no pure inner twists other than itself. The main result of this section is the following.

\begin{theorem}[In progress]\label{thm:Cn_subregular}
    Let $u$ be a subregular unipotent element of $\SO_{2n+1}(\CC)$. For any elliptic pair $(s,h)\in\mathcal{Y}(\Gamma_u)_{\mathrm{ell}}$,
    $$\res_{\un}^{\mathrm{par}}(\Pi(u,h,s))=\FT^{\mathrm{par}}(\res_{\un}^{\mathrm{par}}(\Pi(u,s,h))).$$
    In other words, Conjecture \ref{conj:main} holds when $G=\Sp_{2n}(F)$ and $u$ is a subregular unipotent element.
\end{theorem}

For the sake of completeness, we briefly verify that the conjecture is true if $u$ is a regular unipotent element of $G^\vee$. It is well-known that $\dim Z_{G^\vee}(u)=\rk\SO_{2n+1}=n$ and that $\Gamma_u=Z(G^\vee)=1$ and consequently $\mathcal{Y}(\Gamma_u)_{\text{ell}}=\{(1,1)\}$. The representation $\Pi(u,1,1)=\pi(u,1,\mathbf{1})$ is fixed by $\FT^{\text{ell}}$ and moreover the Springer correspondence implies that 
$$\res_{K_i}\pi(u,1,\mathbf{1})=\St_{K_i}\quad\text{for any maximal parahoric }K_i\subset G.$$
This representation is fixed by each $\FT^{K_i}$, so the conjecture is verified.

To progress any further, we need to understand representation theoretic aspects of the reductive quotients of maximal parahoric subgroups of $G=\Sp_{2n}(F)$ and some structural properties of $G^\vee=\SO_{2n+1}(\CC)$, which we discuss now.

\subsection{Unipotent representations of \texorpdfstring{$\Sp_{2n}(\FF_q)$}{PDFstring}}\label{subsec:C_unipreps}

We recall from Proposition \ref{prop:harishchandra} that for any unipotent representation $\chi$ of $\Sp_{2n}(\FF_q)$ there exists a parabolic subgroup $P\subseteq\Sp_{2n}(\FF_q)$ with Levi decomposition $P=LN$ and a cuspidal unipotent character $\phi$ of $L$ such that $\chi\hookrightarrow\Ind_{P}^{\Sp_{2n}(\FF_q)}\phi$ and that moreover the pair $(P,\phi)$ are unique up to conjugacy. Lusztig showed that finite groups of Lie type of type $A_m, m\geq1$ have no cuspidal unipotent representations, while those of type $C_m$ have a unique cuspidal unipotent representation $\phi_s$ if $m=s^2+s$ for some $s\geq0$, and none otherwise. Thus, the only parabolic subgroups of $\Sp_{2n}(\FF_q)$ whose Levi component has a cuspidal unipotent representation has type $C_m$ for some $m=s^2+s$.

For each $s\geq0$ such that $s^2+s\leq n$, let $P_s=L_sN_s$ be the standard parabolic of type $C_{s^2+s}$ and let $\phi_s$ be the unique cuspidal unipotent representation of $L_s$. The unipotent representations of $\Sp_{2n}(\FF_q)$ are the components of the representations
\begin{equation*}
    \Ind_{P_s}^{\Sp_{2n}(\FF_q)}\phi_s\quad\text{for all }s\geq0\text{ satisfying }s^2+s\leq n.
\end{equation*}

For each choice of $s$, the components of $\Ind_{P_s}^{\Sp_{2n}(\FF_q)}\phi_s$ define a \textit{series} of unipotent representations of $\Sp_{2n}(\FF_q)$ denoted by $\mathcal{E}(\Sp_{2n}(\FF_q),P_s,\phi_s)$. Let us focus first on $s=0$, corresponding to the principal series representations. In this case, $P_0=B$ is a Borel subgroup, $L_0=T$ is a maximal torus and $\phi_0=\mathbf{1}$, so we aim to characterize the components of $\Ind_B^{\Sp_{2n}(\FF_q)}\mathbf{1}$. In \ref{subsec:unipotent_families}, we proved that there is a canonical bijection between principal series representations of $\Sp_{2n}(\FF_q)$ and irreducible representations of its Weyl group $W(C_n)\cong S_n\ltimes C_2^n$. As we explained in Remark \ref{rem:Weylgroups}, most of our computation take place inside the Weyl $W(B_n)$ group of $\SO_{2n+1}(\CC)$ isomorphic to $W(C_n)$ by swapping short and long roots. For convenience, therefore, we state the classification theorem for irreducible $W(B_n)$ in terms of combinatorial data (see \cite[\S 11.4]{Carter1993}).

\begin{theorem}\label{thm:WBn_reps}
    There is a natural one-to-one correspondence between irreducible representations of $W(C_n)$ and ordered pairs of partitions (or bipartitions) $(\alpha,\beta)$ with $|\alpha|+|\beta|=n$. Let $\alpha=(\alpha_1,\alpha_2,\ldots)$ and $\beta=(\beta_1,\beta_2,\ldots)$ with 
    $$0\leq\alpha_1\leq\alpha_2\ldots\quad 0\leq\beta_1\leq\beta_2\ldots,$$
    where some of the $\alpha_i,\beta_i$ may be $0$. Let $\alpha^*,\beta^*$ be the dual partitions of $\alpha,\beta$. Then consider a subsystem $\Phi'\subseteq\Phi$ of type 
    $$D_{\alpha_1^*}+D_{\alpha_2^*}+\cdots+B_{\beta_1^*}+B_{\beta_2^*}+\cdots,$$
    where $D_1$ is the empty root system and $D_2=A_1\times A_1$ has only long roots. If $W'$ is the Weyl group of $\Phi'$, then the corresponding irreducible representation $\phi_{\alpha,\beta}$ of $W(B_n)$ is the Macdonald representation ${j_{W'}}^{W}(\varepsilon_{W'})$ obtained from $W'$. Moreover, the dimension of $\phi_{\alpha,\beta}$ is given by the formula
    $$\dim\phi_{\alpha,\beta}=\frac{n!}{\mathrm{hook}(\alpha)\mathrm{hook}(\beta)}.$$
\end{theorem}



Throughout, we will label the representations of $W(B_n)$ by $(\alpha,\beta)$ instead of $\phi_{\alpha,\beta}$. The above theorem provides a complete classification of the irreducible representations of $W(B_n)$ and, with some work, provides a way to obtain explicit models of each representation a subrepresentation of $\Sym^N(V^*)$ (homogeneous degree $N$ polynomials on $V$), where $V=\oplus_{i=1}^n\CC e_i$ is the natural reflection representation of $W(B_n)$ described above and $N$ is the number of positive roots of the associated root subsystem $\Phi'$. To simplify notation, let $\{x_1,\ldots,x_n\}\subset V^*$ be the dual basis of $\{e_1,\ldots,e_n\}$. 

Let us discuss some important examples and models of the above theorem.
\begin{enumerate}
    \item If $(\alpha,\beta)=(n,\emptyset)$, then $\Phi'=\emptyset$ so $N=0$ and $(n,\emptyset)$ is the trivial representation.
    \item If $(\alpha,\beta)=(\emptyset, n)$, then $\Phi'=\{\text{short roots}\}$ so $N=n$ and $(\emptyset,n)$ is the non-trivial character where long reflections act trivially and whose model is given by 
    $$\Span_\CC\left\{x_1\cdots x_n\right\}\leq\Sym^{n}(V^*).$$
    \item If $(\alpha,\beta)=(1^n,\emptyset)$, then $\Phi'=\{\text{long roots}\}$ so $N=n^2-n$ and $(1^n,\emptyset)$ is the non-trivial character where short reflections act trivially and whose model is given by 
    $$\Span_\CC\left\{\prod_{1\leq i<j\leq n}(x_i^2-x_j^2)\right\}\leq\Sym^{n^2-n}(V^*).$$
    \item If $(\alpha,\beta)=(\emptyset, 1^n)$, then $\Phi'=\Phi$ and $W'=W$. Then $N=n^2$ and $(\emptyset,1^n)$ is the sign representation whose model is given by 
    $$\Span_\CC\left\{x_1\cdots x_n\cdot\prod_{1\leq i<j\leq n}(x_i^2-x_j^2)\right\}\leq\Sym^{n^2}(V^*).$$
    \item If $(\alpha,\beta)=(n-1,1)$, then $\Phi'$ has type $A_1$ so $N=1$ and $(n-1,1)=\Sym^1V^*=V^*$ is the natural reflection representation of $W$.
    \item If $(\alpha,\beta)=((n-1)1,\emptyset)$, then $\Phi'$ has type $A_1\times A_1$ so $N=2$ and $((n-1)1,\emptyset)$ is the $n-1$-dimensional representation with model
    $$\Span_\CC\{x_1^2-x_2^2,\ldots,x_{n-1}^2-x_n^2\}.$$
    \item If $(\alpha,\beta)=(n-2,2)$, then $\Phi'$ has type $A_1\times A_1$ so $N=2$ too and $(n-2,2)$ is the $n(n-1)/2$-dimensional representation with model 
    $$\Span_\CC\{x_ix_j\ |\ 1\leq i<j\leq n\}.$$
    In particular, by noting that $\CC(x_1^2+\cdots+x_n^2)$ affords the trivial representation and by counting dimensions, this shows that $\Sym^2(V^*)=(n,\emptyset)+((n-1)1,\emptyset)+(n-2,2)$. 
\end{enumerate}

In addition to the previous examples, we also provide an explicit description of the effect of twisting by the sign representation $\varepsilon=(\emptyset,1^n)$.

\begin{lemma}\label{lem:twist_sign}
    Let $(\alpha,\beta)$ be a representation of $W(B_n)$ labelled as in Theorem \ref{thm:WBn_reps} and let $\alpha^*,\beta^*$ be the dual partitions. Then 
    $$\varepsilon\otimes(\alpha,\beta)=(\emptyset,1^n)\otimes(\alpha,\beta)=(\beta^*,\alpha^*).$$
    In particular, if $\phi_1$ and $\phi_2$ are two characters of the same family, so are $\varepsilon\otimes\phi_1$ and $\varepsilon\otimes\phi_2$.
\end{lemma}

\begin{lemma}
    The smallest normal subgroup of $W(B_n)=S_n\ltimes C_2^n$ containing a short reflection is isomorphic to $C_2^n$. The set of short reflections lies in the kernel of a representation $(\alpha,\beta)$ if and only if $\beta=\emptyset$, and the bijection 
    \begin{align*}
        \Irr(W(A_{n-1}))&\longrightarrow\Irr(W(B_n))\\
        \alpha&\longmapsto(\alpha,\emptyset)
    \end{align*}
    is the natural inflation map of representations.
\end{lemma}

Next, we wish to understand, using the previous parametrization, the families of the Weyl group as defined in \ref{subsec:unipotent_families}. To do this, we need to introduce the notion of \textit{symbols} associated to a pair of partitions $(\alpha,\beta)$. Choose an appropriate number of zeros as parts of $\alpha$ or $\beta$ so that $\alpha$ has one more part than $\beta$. Then define the symbol of $(\alpha,\beta)$ to be the array
\[
\left(
\begin{array}{ccccccccccc}
\alpha_1 & \phantom{\beta_1} & \alpha_2+1 & \phantom{\beta_2} & \alpha_3+2 & \phantom{\beta_3} & \cdots & \alpha_m+(m-1) & \phantom{\beta_m} & \alpha_{m+1}+m \\[6pt]
\phantom{\alpha_1} & \beta_1 & \phantom{\alpha_2+1} & \beta_2+1 & \phantom{\alpha_3+2} & \beta_3+2 & \cdots & \phantom{\alpha_m+(m-1)} & \beta_m+(m-1) &
\end{array}
\right)
\]

We consider the equivalence relation on the symbols generated by 
\begin{align*}
    &\left(
    \begin{array}{ccccccccccc}
    0 & \phantom{\mu_1+1} & \lambda_1+1 & \phantom{\mu_1+1} & \lambda_2+1 & \phantom{\mu_2+1} & \cdots & \phantom{\mu_{m-1}+1} & \lambda_m+1 & \phantom{\mu_m+1} & \lambda_{m+1}+1 \\[6pt]
    \phantom{0} & 0 & \phantom{\lambda_1+1} & \mu_1+1 & \phantom{\lambda_2+1} & \mu_2+1 & \cdots & \mu_{m-1}+1 & \phantom{\lambda_m+1} & \mu_m+1 &
    \end{array}
    \right)\\
    \sim
    &\left(
    \begin{array}{ccccccccccc}
    \lambda_1 & \phantom{\mu_1} & \lambda_2 & \phantom{\mu_2} & \cdots & \phantom{\mu_{m-1}} & \lambda_m & \phantom{\mu_m} & \lambda_{m+1} \\[6pt]
    \phantom{\lambda_1} & \mu_1 & \phantom{\lambda_2} & \mu_2 & \cdots & \mu_{m-1} & \phantom{\lambda_m} & \mu_m &
    \end{array}
    \right).
\end{align*}

Thus, each pair $(\alpha,\beta)$ defines a unique equivalence class of symbols. 

\begin{theorem}
    Two characters of $W(B_n)$ lie in the same family if and only if they possess symbols for which the unordered sets $\{\lambda_1,\ldots,\lambda_{m+1},\mu_1,\ldots,\mu_m\}$ are the same. Moreover, each family contains a unique character whose symbol has the property that 
    $$\lambda_1\leq\mu_1\leq\lambda_2\leq\cdots\leq\mu_m\leq\lambda_{m+1},$$
    and these are precisely the special characters of $W(B_n)$.
\end{theorem}

This concludes the classification of principal series representations of $\Sp_{2n}(\FF_q)$. To understand the remaining series, we need to consider more general symbols of the form

\[
\left(
\begin{array}{ccccccccc}
\lambda_1 & \phantom{\mu_1} & \lambda_2 & \phantom{\mu_2} & \lambda_3 & \phantom{\mu_3} & \cdots & \phantom{\mu_b} & \lambda_b \\[6pt]
\phantom{\lambda_1} & \mu_1 & \phantom{\lambda_2} & \mu_2 & \phantom{\lambda_3} & \mu_3 & \cdots & \mu_b &
\end{array}
\right)
\]

where $0\leq\lambda_1<\lambda_2<\ldots<\lambda_a$, $0\leq\mu_1<\mu_2<\ldots<\mu_b$, $a-b$ is odd and positive and $\lambda_1,\mu_1$ are not both $0$. We define the \textit{defect} of such a symbol as $d=a-b$ and its \textit{rank} as
$$\sum_{i=1}^{a}\lambda_i+\sum_{j=1}^{b}\mu_j-\left[\left(\frac{a+b-1}{2}\right)^2\right].$$

We can finally state the classification theorem for unipotent characters of $\Sp_{2n}(\FF_q)$.

\begin{theorem}
    Unipotent characters of $\Sp_{2n}(\FF_q)$ are parametrized in terms of symbols of the form as above whose rank equals $n$. Such a character is a component of $\Ind_{P_s}^{\Sp_{2n}(\FF_q)}\phi_s$, where $d=2s+1$. Finally, two unipotent characters lie in the same family of characters if and only if their symbols contain the same entries with the same multiplicities.
\end{theorem}




\subsection{Structure theory of \texorpdfstring{$\SO_{2n+1}(\CC)$}{PDFstring}}

 We recall that 
$$\SO_{2n+1}=\{A\in\GL_{2n+1}\ |\ AJA^T=J\},\quad\text{where}\quad J=\begin{pmatrix}
    1 & & \\
     & & I_n \\
     & I_n & \\
\end{pmatrix}$$
The diagonal matrices in $\SO_{2n+1}(\CC)$ are
$$T_n=\left\{h(a_1,\ldots,a_n):=\begin{pmatrix}
1 & & & & & & \\
& a_1 & & & & & \\
& & \ddots& & & & \\
& & & a_n & & & \\
& & & & a_1^{-1} & & \\
& & & & & \ddots& \\
& & & & & & a_n^{-1}\\
\end{pmatrix}\ |\ a_1,\ldots,a_n\in\CC^\times\right\},$$
and they form a maximal torus in $\SO_{2n+1}(\CC)$. We fix a set of simple roots $\Delta_n=\{\alpha_1,\alpha_2,\ldots,\alpha_n\}$ given by 
\begin{equation*}
    \alpha_i(h(a_1,\ldots,a_n))=\begin{cases}
        a_ia_{i+1}^{-1} & \text{ if } 1\leq i\leq n-1,\\
        a_{n} & \text{ if } i=n.
    \end{cases}
\end{equation*}
The simple root $\alpha_n$ is the only short one and the corresponding Dynkin diagram is
\[
    \alpha_1 \relbar \alpha_2 \relbar\cdots\relbar \alpha_{n-1} \xRightarrow{\quad} \alpha_n.
\]
Using the Killing form, we can embed $\Delta_n$ in a $n$-dimensional Euclidean space $V$ with orthonormal basis ${e_1,\ldots,e_n}$ and such that $\alpha_i=e_i-e_{i+1}$ for $1\leq i\leq n-1$ and $\alpha_n=e_n$.


We also let $\alpha_0:=\alpha_1+\alpha_2+\cdots+\alpha_n=e_1$ be the highest short root of $\Phi(B_n)$. This is slightly unconventional, since $\alpha_0$ is normally the highest root of $\Phi$, but this notation will be very useful. For example, $\calpha_0$ is the highest root of type $C_n$ so $S_{\aff}(C_n):=\{-\calpha_0,\calpha_1,\ldots,\calpha_n\}$ is the set of affine roots of type $C_n$ with corresponding affine Dynkin diagram 
\[
    -\calpha_0 \xRightarrow{\quad} \calpha_1 \relbar\calpha_2 \relbar\cdots\relbar \calpha_{n-1} \xLeftarrow{\quad} \calpha_n.
\]

In particular, $\alpha_0(h(a_1,\ldots,a_n))=a_1$ and the corresponding simple coroots are 
\begin{equation*}
    \calpha_i(t)=\begin{cases}
        h(t^2,1,\ldots,1) & \text{ if } i=0,\\
        h(1,\ldots,1,t,t^{-1},1,\ldots,1) & \text{ if } 1\leq i\leq n-1,\\
        h(1,\ldots,1,t^2) & \text{ if } i=n.
        
    \end{cases}
\end{equation*}

The Weyl group $W(B_n)=N(T_n)/T_n$ is isomorphic to the group $S_n\ltimes C_2^n$ and it acts on $T_n$ faithfully by the transformations 
$$h(a_1,\ldots,a_n)\longmapsto h(a_{\sigma(1)}^{\pm1},\ldots,a_{\sigma(n)}^{\pm1}),\quad\text{where }\sigma\in S_n.$$
For each $i\in\{0,1,2,\ldots,n\}$, let $s_i\in W(B_n)$ be the simple reflection associated to the root $\alpha_i$. Then 
$$W(B_n)=\langle s_1,\ldots,s_{n-1}\text{ (long reflections)}, s_n\text{ (short reflection)}\rangle=\langle s_0,s_1,\ldots,s_{n-1}\rangle,$$ 
and the simple reflections act by
\begin{equation*}
    s_i\cdot h(a_1,\ldots,a_n)=\begin{cases}
        h(a_1^{-1},a_2,\ldots,a_n) & \text{ if } i=0,\\
        h(a_1,\ldots,a_{i+1},a_{i},\ldots,a_n) & \text{ if } 1\leq i\leq n-1,\\
        h(a_1,\ldots,a_{n-1},a_n^{-1}) & \text{ if } i=n.
    \end{cases}
\end{equation*}

Moreover, the Euclidean space $V=\oplus_{i=1}^n e_i$ is the Lie algebra of $T_n$ so admits a natural $W(B_n)$-action given by 
$$s_i(v)=v-\frac{2(\alpha_i,v)}{(\alpha_i,\alpha_i)}\alpha_i,\quad v\in V\quad\text{so}\quad s_i=\begin{psmallmatrix}
    I_{i-1} & & & \\
    & 0 & 1 & \\
    & 1 & 0 & \\
    & & & I_{n-i-1}
\end{psmallmatrix}\text{ for }1\leq i\leq n-1\quad\text{and}\quad s_n=\begin{psmallmatrix}
    I_{n-1} & \\
    & -1
\end{psmallmatrix}$$
and under this action $V$ becomes the natural reflection representation of $W(B_n)$.

\iffalse

Next, we study the structure of unipotent conjugacy classes of $\SO_{2n+1}(\CC)$. We recall from Section \ref{subsec:unipotent_classes} that each complex simple group has three canonical classes of unipotent elements: $C_{\reg}, C_{subreg}$ and $C_{\min}$. For $\SO_{2n+1}(\CC)$, the unipotent conjugacy classes can be parametrized as follows.

\begin{proposition}\label{prop:unip_odd_so}
    The conjugacy classes of $\SO_{2n+1}(\CC)$ are parametrized by pairs $(\lambda,\mu)$ of permutations such that $2|\lambda|+|\mu|=2n+1$ and $\mu$ has distinct odd parts and no even parts. The regular orbit is parametrized by $(\emptyset,2n+1)$, the subregular orbit by $(1,2n-1)$, the minimal orbit by $(21^{n-2},1)$ and the zero orbit by $(1^n,1)$.
\end{proposition}

In addition to this result, one can also find nice representatives for each of these orbits. The results are summarized in the following table \textcolor{red}{(need to reference this well)}.

\begin{table}[!ht]
    \begin{tabular}{c|c|c|c|c|c}
        Unipotent & $(\lambda,\mu)$ & Dynkin Diagram & $\dim Z_{G^\vee}(u)$ &$\dim\mathcal{B}^u$ & $\Gamma_u$ \\ \hline
        $\prod_{i=1}^n x_{\alpha_i}(1)$ & $(\emptyset,2n+1)$ & & $n$ & $0$ & $1$ \\
        $\prod_{i=2}^{n} x_{\alpha_i}(1)$ & $(1,2n-1)$ & & $n+2$ & $1$ & $\CC^\times\rtimes C_2$ \\
        ... & ... & ... & ... & ... & ... \\
        $x_{\alpha_1}(1)$ & $(21^{n-2},1)$ & & $2n^2-3n+4$ & $n^2-2n+2$ & $B_{n-2}\times A_1$ \\
        $1$ & $(1^n,1)$ & & $2n^2+n$ & $n^2$ & $\SO_{2n+1}$     
    \end{tabular}
\end{table}

In the next subsection, we will prove that Conjecture \ref{conj:main} holds for the subregular unipotent orbit containing $u=\prod_{i=1}^{n-1}x_{\alpha_i}(1)$. Before we do this, however, we need to discuss the representation theory of parahoric reductive quotients of $\Sp_{2n}(F)$.

\subsection{Springer correspondence of \texorpdfstring{$\SO_{2n-1}(\CC)$}{PDFstring}}
\fi


\subsection{The subregular unipotent orbit of \texorpdfstring{$\SO_{2n+1}(\CC)$}{PDFstring}}
In this section, we prove Conjecture \ref{conj:main} for the subregular unipotent class, commonly labelled by the partition $(2n-1,1,1)$. The first step is to find a nice representative of this class, which is fortunately possible.

\begin{lemma}
    The unipotent element $u=\exp(e_{\alpha_2}+\cdots+e_{\alpha_{n-1}}+e_{\alpha_n})$ lies in the subregular orbit and the reductive part of its centralizer is $\Gamma_u=\Gamma_u^0\rtimes\langle h_uw_{\alpha^0}\rangle$, where
    $$\Gamma_u^0=\{(\alpha^0)^\vee(t)\ |\ t\in\CC^\times\}\quad\text{and}\quad h_u=\begin{psmallmatrix}
    1 & \\
     & -I_{2n}
    \end{psmallmatrix}=\begin{cases}
        \calpha_1(-1)\calpha_3(-1)\cdots\calpha_{n-2}(-1)\calpha_n(i) & \text{ if $n$ is odd,}\\
        \calpha_1(-1)\calpha_3(-1)\cdots\calpha_{n-3}(-1)\calpha_{n-1}(-1) & \text{ if $n$ is even.}
    \end{cases}$$
\end{lemma}
\begin{proof}
    The first part can be found in \cite[\S 5.2]{CollingwoodMcGovern1993}, while the second one can be easily checked with a computer algebra program such as GAP.
\end{proof}

The above result implies that there is an isomorphism 
$$\Gamma_u\cong\langle z,\delta\ |\ z\in\CC^\times,\delta^2=1,\delta z\delta^{-1}=z^{-1}\rangle\cong C^\times\rtimes C_2$$
given by 
\begin{align*}
    1\longleftrightarrow t_0=&(\alpha^0)^\vee(\pm1)=I_{2n+1},\quad -1\longleftrightarrow t_1=(\alpha^0)^\vee(\pm i)=\begin{psmallmatrix}
        1 & & & &  \\
        & -1 & & &\\
        & & I_{n-1} & & \\
        & & & -1 &\\
        & & & & I_{n-1}\\
    \end{psmallmatrix},\\
    &\delta\longleftrightarrow t_2=h_uw_{\alpha^0}=\begin{psmallmatrix}
        -1 & & & & \\
        & & & 1 & \\
        & & -I_{n-1}& &\\
        & 1 & & &\\
        & &  & & -I_{n-1}
    \end{psmallmatrix}.
\end{align*}
In particular, we note that $\Gamma_{ut_0}=\Gamma_{ut_1}=\Gamma_u$ so $A_{ut_0}=A_{ut_1}=C_2$ while $\Gamma_{ut_2}=A_{ut_2}=\{t_0,t_1,t_2,t_1t_2\}=C_2\times C_2$. We label its representations by $\mathbf{1}\boxtimes\mathbf{1}$, $\mathbf{1}\boxtimes\varepsilon$, $\varepsilon\boxtimes\mathbf{1}$, $\varepsilon\boxtimes\varepsilon$, where the first term indicates the action of $t_1$ and the second term indicates the action of $t_2$.
\begin{lemma}
    The group $\Gamma_u$ has six elliptic pairs up to $\Gamma_u$-conjugacy:
    $$\Gamma_u\backslash\mathcal{Y}(\Gamma_u)_{\mathrm{ell}}=\{(\pm1,\delta),(\delta,\pm1),(\delta,\pm\delta)\}.$$
\end{lemma}
\begin{proof}
    All elements in $\Gamma_u$ are semisimple, and any two $\gamma_1,\gamma_2\in\Gamma_u$ commute if and only if $\gamma_1,\gamma_2\in\CC^\times$ (in which case their common centralizer is infinite) or 
    $(\gamma_1,\gamma_2)\in\{(\pm1,z\delta),(z\delta,\pm1),(z\delta,\pm z\delta)\ | \ z\in\CC^\times\}$. It is easy to see that these pairs have finite centralizer and are $\CC^\times$-conjugate to $\{(\pm1,\delta),(\delta,\pm1),(\delta,\pm\delta)\}$, respectively.
\end{proof}

Thus, we wish to compute the parahoric restrictions with respect to open maximal compact subgroups of the virtual representations
\begin{align*}
    \Pi(u,1,\delta)=\pi(ut_0,\mathbf{1})-\pi(ut_0,\varepsilon),\quad \Pi(u,-1,\delta)=\pi(ut_1,\mathbf{1})-\pi(ut_1,\varepsilon),\\
    \Pi(u,\delta,1)=\pi(ut_2,\mathbf{1}\boxtimes\mathbf{1})+\pi(ut_2,\mathbf{1}\boxtimes\varepsilon)+\pi(ut_2,\varepsilon\boxtimes\mathbf{1})+\pi(ut_2,\varepsilon\boxtimes\varepsilon),\\
    \Pi(u,\delta,-1)=\pi(ut_2,\mathbf{1}\boxtimes\mathbf{1})+\pi(ut_2,\mathbf{1}\boxtimes\varepsilon)-\pi(ut_2,\varepsilon\boxtimes\mathbf{1})-\pi(ut_2,\varepsilon\boxtimes\varepsilon),\\
    \Pi(u,\delta,\delta)=\pi(ut_2,\mathbf{1}\boxtimes\mathbf{1})-\pi(ut_2,\mathbf{1}\boxtimes\varepsilon)+\pi(ut_2,\varepsilon\boxtimes\mathbf{1})-\pi(ut_2,\varepsilon\boxtimes\varepsilon),\\
    \Pi(u,\delta,-\delta)=\pi(ut_2,\mathbf{1}\boxtimes\mathbf{1})-\pi(ut_2,\mathbf{1}\boxtimes\varepsilon)-\pi(ut_2,\varepsilon\boxtimes\mathbf{1})+\pi(ut_2,\varepsilon\boxtimes\varepsilon).
\end{align*}

We first compute these restrictions for representations having Iwahori-fixed vectors, whose restrictions to $\overline{K}_J$ are a direct sum of principal series representations. The restrictions of the remaining representations will be calculated afterwards. 

We recall from Lemma \ref{lem:multiplicity} and Proposition \ref{prop:second_red} that if $\pi(ut,\rho)$ has Iwahori-fixed vectors and $J\subsetneq S_{\aff}=\{-\calpha_0,\calpha_1,\calpha_2,\ldots,\calpha_n\}$, for any irreducible $K_J$ module $\chi$ trivial on $U_J$, we have that 
$$\langle\chi,V^{U_J}\rangle_{K_J}=\langle\chi^{\mathcal{I}},V^{\mathcal{I}}\rangle_{\mathcal{H}(K_J,\mathcal{I},\mathbf{1})}=\langle\chi^{\mathcal{I}}_{q=1},\pi(ut,\rho)^{\mathcal{I}}_{q=1}\rangle_{\widetilde{W}_J}.$$

The operation $\chi\mapsto\chi^{\mathcal{I}}_{q=1}$ corresponds to the aforementioned bijection between principal series representations of $\overline{K}_J$ and irreducible representations of $\widetilde{W}_J$, while Theorem \ref{thm:Kato_module} and Proposition \ref{prop:ind_cohomo} imply that
$$\pi(ut,\rho)^{\mathcal{I}}_{q=1}=\varepsilon\otimes\Ind_{\widetilde{W}_t}^{\widetilde{W}}[s\otimes H^*(\mathcal{B}_t^u)^\rho]$$
as a $\widetilde{W}$-module, up to semisimplification. In particular, we are interested in the $W_J$-module (see \eqref{eqn:mackey_easycase})
\begin{equation}\label{eqn:mackey}
    \psi_J^*\left(\pi(ut,\rho)^{\mathcal{I}}_{q=1}|_{\widetilde{W}_J}\right)=\varepsilon\otimes\bigoplus_{w\in W_t\backslash W/W_J}\Ind_{W_J\cap W_{t^w}}^{W_J}\chi_{t^w}^{J}\otimes\left[H^*(\mathcal{B}^u_t)^\rho\right]^w.
\end{equation}


Our aim is to compute parahoric restriction with respect to maximal compact subgroups. Since $\Sp_{2n}(F)$ is simply connected, these coincide with maximal parahoric subgroups and its conjugacy classes are in natural bijection with maximal subsets $J\subsetneq S_{\aff}$. For each $r=0,\ldots,n$, let $J_r=S_{\aff}\backslash\{\calpha_r\}$ be the maximal subset of $S_{\aff}$, $K_r$ be the corresponding parahoric subgroup with unipotent radical $U_r$ and reductive quotient $\overline{K}_r$ isomorphic to $\Sp_{2r}(\FF_q)\times\Sp_{2(n-r)}(\FF_q)$. In particular, the Weyl group $W_{J_r}$ of $\overline{K}_r$ is generated by 
$$W_{J_r}=\langle s_0,\ldots,s_{r-1}\rangle\times\langle s_{r+1},\ldots,s_n\rangle\cong W(B_r)\times W(B_{n-r}).$$

\textbf{Case 0: $t=t_0$.} In this case, $Z_{G^\vee}(t_0)=G^\vee$ so $W_{t^0}=W$, $\mathcal{B}_{t^0}^u=\mathcal{B}^u$ and the character $\chi_{t_0}^{J_r}$ is trivial. Hence, the right hand side of \eqref{eqn:mackey} becomes
\[
\varepsilon\otimes H(\mathcal{B}^u)^\rho|_{W_{J_r}}.
\]
The variety $\mathcal{B}^u$ is one-dimensional, so $H(\mathcal{B}^u)=H^0(\mathcal{B}^u)+H^2(\mathcal{B}^u)$.
The $W$-action on $H(\mathcal{B}^u)$ is described by the Springer correspondence. The $0$-th cohomology group is easy to describe since $H^0(\mathcal{B}^u)^\mathbf{1}$ affords the trivial $W$-representation while $H^0(\mathcal{B}^u)^\varepsilon=0$. On the other hand, the top cohomology groups $H^2(\mathcal{B}^u)^\mathbf{1}$ and $H^2(\mathcal{B}^u)^\varepsilon$ afford the $W$-representations $(n-1,1)$ and $((n-1)1,\emptyset)$, respectively.

The restriction of the $0$-th cohomology modules to $W_{J_r}$ is straightforward, so we focus on the restriction of the top cohomology modules to $W_{J_r}$. Let $V_{(n-1,1)}=\oplus_{i=1}^n\CC x_i$ be the natural reflection representation and let $V_{((n-1)1,\emptyset)}=\oplus_{i=1}^{n-1}\CC(x_i^2-x_{i+1}^2)$ be the representation labelled by $((n-1)1,\emptyset)$. 

If $r=0,n$, then $W_{J_r}=W$ so there is nothing to prove. If $1\leq r\leq n-1$, then there are direct sum decompositions
\begin{align*}
    V_{(n-1,1)}&=\Span\{x_i\ |\ 1\leq i\leq r\}\oplus\Span\{x_i\ |\ r+1\leq i\leq n\},\\
    V_{((n-1)1,\emptyset)}&=\Span\{x_i^2-x_{i+1}^2\ |\ 1\leq i\leq r-1\}\oplus\Span\{x_i^2-x_{i+1}^2\ |\ r+1\leq i\leq n-1\}\oplus\Span\{(n-r)\sum_{i=1}^{r}x_i^2-r\sum_{i=r+1}^{n}x_i^2\}
\end{align*}
into irreducible $W_{J_r}$-modules, affording the representations $(r-1,1)\boxtimes(n-r,\emptyset)$ and $(r,\emptyset)\boxtimes(n-r-1,1)$ for the first line, and $((r-1)1,\emptyset)\boxtimes(n-r,\emptyset)$, $(r,\emptyset)\boxtimes((n-r-1)1,\emptyset)$ and $(r,\emptyset)\boxtimes(n-r,\emptyset)$ for the second line. This finishes the first case, which is the content of the first two rows of Table \ref{tbl:par_res}.

\textbf{Case 1: $t=t_1$.} In this case, 
\begin{align}\label{eqn:centralizer_t1}
    Z_{G^\vee}(t_1)=Z_{G^\vee}(t_1)^0\times\langle s_0\rangle\quad\text{where}\quad
    Z_{G^\vee}(t_1)^0=\langle T,\mathfrak{X}_{\pm\alpha_2},\ldots,\mathfrak{X}_{\pm\alpha_{n-1}},\mathfrak{X}_{\pm\alpha_{n}}\rangle
\end{align}
is a connected reductive group of type $B_{n-1}$. The Weyl group of $Z_{G^\vee}(t_1)$ is 
$$W_{t_1}=\langle s_0\rangle\times\langle s_2,\ldots,s_{n-1},s_n\rangle\cong C_2\times W(B_{n-1})\cong C_2^n\rtimes S_n$$ 
and acts on $T$ by the transformations 
$$h(a_1,\ldots,a_{n-1},a_n)\longmapsto h(a_{1}^{\pm1},a_{\tau(2)}^{\pm1},\ldots,a_{\tau(n)}^{\pm1}),\quad\text{where }\tau\in S_{n-1}.$$
Thus $|W_{t_1}|=2^n(n-1)!$ and the index inside $W$ is $[W:W_{t_1}]=n$. To understand the right hand side of \eqref{eqn:mackey}, we first need to understand the double coset space $W_{t_1}\backslash W/W_{J_r}$. If $r=0$ or $r=n$, then $W_{J_r}\cong W$, so there is only one double coset. In these two cases, the right hand side of \eqref{eqn:mackey} simplifies to 
$$\varepsilon\otimes\Ind_{W_{t_1}}^{W}\chi_{t_1}^{J_r}\otimes H^*(\mathcal{B}_{t_1}^u)^\rho.$$
Using the explicit formulas in \cite[\S 8.3]{Reeder2000}, one can compute that $\chi_{t_1}^{J_0}$ is the trivial character while $\chi_{t_1}^{J_n}$ is trivial on $\langle s_2,\ldots,s_n\rangle$, but it takes $s_0$ to $-1$. To finish the calculations, we need to study the $W_{t_1}$-action on the cohomology complex $H^*(\mathcal{B}^u_{t_1})$. From \eqref{eqn:centralizer_t1}, $u\in Z_{G^\vee}(t_1)$ and is a regular unipotent element, so $\dim\mathcal{B}^u_{t_1}=0$. However, since $Z_{G^\vee}(t_1)$ is not connected, the structure of $\mathcal{B}^u_{t_1}$ is more interesting. The following result gives an explicit description of $H^0(\mathcal{B}_{t_1}^u)$ as a $W_{t_1}$-module.

\begin{lemma}
    The variety $\mathcal{B}^u_{t_1}$ consists of two points. The two dimensional complex vector space $H^0(\mathcal{B}^u_{t_1})$ has a natural action of $W_{t_1}\times A_{ut_1}$ and it admits a direct sum decomposition 
    $$H^0(\mathcal{B}^u_{t_1})=H^0(\mathcal{B}^u_{t_1})^\mathbf{1}+H^0(\mathcal{B}^u_{t_1})^\varepsilon$$ into $1$-dimensional representations of $W_{t_1}\cong C_2\times W(B_{n-1})$ affording the characters $\mathbf{1}\boxtimes(n-1,\emptyset)$, $\varepsilon\boxtimes(n-1,\emptyset)$, respectively.
\end{lemma}
\iffalse
\begin{proof}
    Since $u$ is a regular unipotent element in the connected reductive group $Z_{G^\vee}(t_1)^0$, it is contained in a unique Borel subgroup of $Z_{G^\vee}(t_1)^0$, generated by $B=\langle T,\mathfrak{X}_{\alpha_2},\ldots,\mathfrak{X}_{\alpha_{n-1}},\mathfrak{X}_{\alpha_{n}}\rangle$. We know that $t_2=h_uw_{\alpha^0}\in Z_{G^\vee}(u)-B$, so $u\in \prescript{t_2}{}{B}$. but $B\neq\prescript{t_2}{}{B}$. \textcolor{red}{Since $A_u=C_2$, these are the only two Borel subgroups containing $u$.}

    We note that $A_{ut_1}=\{1,t_2\}$ and that $t_2$ acts on $\mathcal{B}^u_{t_1}$ by permuting both points and therefore 
    $$H^0(\mathcal{B}^u_{t_1})^\mathbf{1}=\CC(B+\prescript{t_2}{}{B}),\quad\text{while}\quad H^0(\mathcal{B}_{t_1}^u)^\varepsilon=\CC(B-\prescript{t_2}{}{B})$$


    By the classical Springer correspondence of type $B_{n-1}$ applied to the regular orbit containing $u$, we have that $\langle s_2,\ldots,s_{n-1},s_n\rangle$ acts trivially on $H^0(\mathcal{B}^u_{t_1})$. 
    On the other hand, since $t_2\in N(T_n)$ maps to $s_0$, the action of $s_0$ on $H^0(\mathcal{B}_{t_1}^u)$ coincides with that of $t_2$.
    \textcolor{red}{Maybe should justify this with the generalized Springer correspondence for disconnected reductive groups.}
\end{proof}\fi

It is therefore enough, in order to calculate these restrictions, to decompose the abstract $W$-representations 
$$\Ind_{W_{t_1}}^W\mathbf{1}\boxtimes(n-1,\emptyset)\quad\text{and}\quad\Ind_{W_{t_1}}^W\varepsilon\boxtimes(n-1,\emptyset)$$
into a direct sum of irreducible representations. Firstly, we note that the trivial representation $(n,\emptyset)$ is certainly an irreducible factor of the first representation. Similarly, the representation $((n-1)1,\emptyset)$, with model given by $\Span\{x_i^2-x_{i+1}^2\ |\ 1\leq i\leq n-1\}$ has a one-dimensional subspace spanned by $(n-1)x_1^2-x_2^2-\cdots-x_n^2$ on which $W_{t_1}$ acts trivially. By Frobenius reciprocity, it follows that $((n-1)1,\emptyset)$ is another irreducible factor. By counting dimensions, we have that 
\begin{equation}\label{eqn:Ind_trivial}
    \Ind_{W_{t_1}}^W\mathbf{1}\boxtimes(n-1,\emptyset)=(n,\emptyset)+((n-1)1,\emptyset).
\end{equation}
To calculate the second induction, we follow a similar argument. The one-dimensional subspace of the natural reflection representation of $W$ spanned by $e_1$ is $W_{t_1}$ invariant, and it affords the character $\varepsilon\otimes(n-1,\emptyset)$ of $W_{t_1}$. By Frobenius reciprocity and counting dimensions,
\begin{equation}\label{eqn:Ind_sign}
    \Ind_{W_{t_1}}^W\mathbf{1}\boxtimes(n-1,\emptyset)=(n-1,1).
\end{equation}
Putting everything together immediately gives the desired parahoric restrictions.

If $1\leq r\leq n-1$, the problem is more involved. The natural quotient map $W\twoheadrightarrow S_n$ induces a bijection
$$W_{t_1}\backslash W/W_{J_r}\longrightarrow\Stab_{S_n}(1)\backslash S_n/\Stab_{S_n}(1,\ldots,r)\times\Stab_{S_n}(r+1,\ldots,n),$$
and the action of $\Stab_{S_n}(1,\ldots,r)\times\Stab_{S_n}(r+1,\ldots,n)$ on $\Stab_{S_n}(1)\backslash S_n$ has two orbits:
$$\{\Stab(1),\Stab(1)(1\ 2),\ldots,\Stab(1\ r)\}\quad\text{and}\quad\{\Stab(1)(1\ r+1),\ldots,\Stab(1\ n)\}.$$
For convenience, we choose representatives $e$ and $\eta$ of $W_{t_1}\backslash W/W_{J_r}$, where $\eta$ acts on $T$ by the permutation $(1\ n)$. This calculation shows that 
\begin{equation}\label{eqn:mackey_t1}
    \bigoplus_{w\in W_{t_1}\backslash W/W_{J_r}}\Ind_{W_{J_r,t_1^w}}^{W_{J_r}}\chi_{t_1^w}^{J_r}\otimes\left[H^*(\mathcal{B}^u_{t_1})^\rho\right]^w=\Ind_{W_{J_r,t_1}}^{W_{J_r}}\chi_{t_1}^{J_r}\otimes H(\mathcal{B}^u_{t_1})^\rho\oplus\Ind_{W_{J_r,t_1^\eta}}^{W_{J_r}}\chi_{t_1^\eta}^{J_r}\otimes\left[H(\mathcal{B}^u_{t_1})^\rho\right]^\eta
\end{equation}
To calculate the $W_{J_r}$-modules above, we first calculate the characters $\chi_{t_1}^{J_r}$ and $\chi_{t_1^\eta}^{J_r}$. Using the explicit formulas in \cite[\S 8.3]{Reeder2000}, we can show that $\chi_{t_1^\eta}^{J_r}$ is the trivial character. On the other hand, $W_{J_r,t_1}\cong\langle s_0\rangle\times W(B_{r-2})\times W(B_{n-r})$ as a subgroup of $W_{J_r}\cong W(B_{r})\times W(B_{n-r})$, and $\chi_{t_1}^{J_r}$ is trivial on $W(B_{r-2})\times W(B_{n-r})$ while $\chi_{t_1}^{J_r}(s_0)=-1$.

We are now ready to calculate the right hand side of \eqref{eqn:mackey_t1}. If $\rho=\mathbf{1}$, then $H^0(\mathcal{B}_{t_1}^u)^\mathbf{1}$ is the trivial module so
\begin{align*}
    \Ind_{W_{J_r,t_1}}^{W_{J_r}}\chi_{t_1}^{J_r}&=\Ind_{\langle s_0\rangle\times W(B_{r-2})\times W(B_{n-r})}^{W(B_r)\times W(B_{n-r})}\varepsilon\boxtimes(r-2,\emptyset)\boxtimes(n-r,\emptyset)=(r-1,1)\boxtimes(n-r,\emptyset),\\
    \Ind_{W_{J_r,t_1^\eta}}^{W_{J_r}}\chi_{t_1^\eta}^{J_r}&=\Ind_{W(B_{r})\times W(B_{n-r-2})\times\langle s_n\rangle}^{W(B_r)\times W(B_{n-r})}(r,\emptyset)\boxtimes(n-r-2,\emptyset)\boxtimes\mathbf{1}=(r,\emptyset)\boxtimes[(n-r,\emptyset)+((n-r-1)1,\emptyset)].
\end{align*}
If $\rho=\varepsilon$, then $s_0$ acts on $H^0(\mathcal{B}_{t_1}^u)^\varepsilon$ by $-1$ while $s_n$ acts on $[H^0(\mathcal{B}_{t_1}^u)^\varepsilon]^\eta$ also by $-1$. Thus, 
\begin{align*}
    \Ind_{W_{J_r,t_1}}^{W_{J_r}}\chi_{t_1}^{J_r}\otimes H^0(\mathcal{B}_{t_1}^u)^\varepsilon&=\Ind_{\langle s_0\rangle\times W(B_{r-2})\times W(B_{n-r})}^{W(B_r)\times W(B_{n-r})}\mathbf{1}\boxtimes(r-2,\emptyset)\boxtimes(n-r,\emptyset)=[(r,\emptyset)+((r-1)1,\emptyset)]\boxtimes(n-r,\emptyset),\\
    \Ind_{W_{J_r,t_1^\eta}}^{W_{J_r}}\chi_{t_1^\eta}^{J_r}\otimes [H^0(\mathcal{B}_{t_1}^u)^\varepsilon]^\eta&=\Ind_{W(B_{r})\times W(B_{n-r-2})\times\langle s_n\rangle}^{W(B_r)\times W(B_{n-r})}(r,\emptyset)\boxtimes(n-r-2,\emptyset)\boxtimes\varepsilon=(r,\emptyset)\boxtimes(n-r-1,1).
\end{align*}

\begin{remark}
    The above computations do not hold if $r=1$ or $r=n-1$, but very similar ideas apply. If $r=1$, all formulas above hold without change except for 
    $$\Ind_{W_{J_1,t_1}}^{W_{J_1}}\chi_{t_1}^{J_1}\otimes H^0(\mathcal{B}_{t_1}^u)^\varepsilon=(1,\emptyset)\boxtimes(n-1,\emptyset).$$
    Similarly, all formulas hold for $r=n-1$ except for 
    $$\Ind_{W_{J_{n-1},t_1^\eta}}^{W_{J_{n-1}}}\chi_{t_1^\eta}^{J_{n-1}}\otimes [H^0(\mathcal{B}_{t_1}^u)^\mathbf{1}]^\eta=(n-1,\emptyset)\boxtimes(1,\emptyset)$$
\end{remark}
This gives us all the parahoric restrictions, and the results can be found in the third and fourth rows in Table \ref{tbl:par_res}.

\textbf{Case 2: $t=t_2$.} Firstly, we note that $t_2$ is conjugate to the diagonal matrix $\begin{psmallmatrix}
    1 & \\
     & -I_{2n}
\end{psmallmatrix}$, and therefore 
\begin{align*}
    Z_{G^\vee}(t_2)=Z_{G^\vee}(t_2)^0\times\langle s_0\rangle\quad\text{where}\quad
    Z_{G^\vee}(t_2)^0=\langle T,\mathfrak{X}_{\beta}\ |\ \beta\text{ is a long root}\rangle
\end{align*}
is a connected reductive group of type $D_n$. Thus, the Weyl group of $Z_{G^\vee}(t_2)$ is $W_{t_2}=W$ and $u$ is a regular unipotent element in $Z_{G^\vee}(t_2)$, so $\dim\mathcal{B}_{t_2}^u=0$. The right hand side of \eqref{eqn:mackey} simplifies to 
\begin{equation}\label{eqn:t2_case}
    \varepsilon\otimes\chi_{t_2}^{J_r}\otimes H^0(\mathcal{B}^u_{t_2})^\rho|_{W_{J_r}}.
\end{equation}

The character $\chi_{t_2}^{J_r}$ of $W_{J_r}=\langle s_0,s_1,\ldots,s_{r-1}\rangle\times\langle s_{r+1},\ldots,s_{n-1},s_n\rangle$ is trivial on the long reflections, while $\chi_{t_2}^{J_r}$ sends $s_n$ to $1$ and $s_0$ to $-1$. Thus, $\chi_{t_2}^{J_r}$ is the $W_{J_r}$ representation labelled by $(\emptyset,r)\boxtimes(n-r,\emptyset)$.

Finally, we need an explicit description of the $W$-action on $H^*(\mathcal{B}^u_{t_2})=H^0(\mathcal{B}^u_{t_2})$.

\begin{lemma}
    The variety $\mathcal{B}^u_{t_2}$ consists of two points. The two dimensional complex vector space $H^0(\mathcal{B}^u_{t_2})$ has a natural action of $W\times A_{ut_2}$ where $A_{ut_2}=\{t_0,t_1,t_2,t_1t_2\}$. It admits a direct sum decomposition
    $$H^*(\mathcal{B}_{t_2}^u)=H^0(\mathcal{B}_{t_2}^u)^{\mathbf{1}\otimes\mathbf{1}}\oplus H^0(\mathcal{B}_{t_2}^u)^{\varepsilon\otimes\mathbf{1}}$$
    into $1$-dimensional $W$-representations, affording the characters $(n,\emptyset)$ and $(\emptyset,n)$, respectively.
\end{lemma}

In particular, the above result implies that $H^0(\mathcal{B}_{t_2}^u)^{\mathbf{1}\otimes\mathbf{1}}|_{W_{J_r}}$ is the trivial representation labelled by $(r,\emptyset)\boxtimes(n-r,\emptyset)$ while $H^0(\mathcal{B}_{t_2}^u)^{\mathbf{1}\otimes\mathbf{1}}|_{W_{J_r}}$ is the character labelled by $(\emptyset,r)\boxtimes(\emptyset,n-r)$. We can then calculate the parahoric restriction directly using \eqref{eqn:t2_case}, and the results are contained in the fifth and seventh row of Table \ref{tbl:par_res}.

This concludes the computations of all parahoric restrictions of Iwahori-spherical representations. It remains to compute the restrictions for the remaining two representations $\pi(u,t_2,\mathcal{1}\otimes\varepsilon)$ and $\pi(ut_2,\varepsilon\otimes\varepsilon)$ that are not Iwahori-spherical representations. 

\begin{proposition}[Proof in progress. It relies on the generalized Springer correspondence]
    For $r\geq2$, let $\theta_r'$ be the irreducible representation of $\Sp_{2r}(\FF_q)$ labelled by the symbol
    $$\begin{pmatrix}
        0 & 1 & 2 & \ldots & r-2 & r-1 & r\\
        && 1 & \ldots & r-2 &&
    \end{pmatrix}.$$
    Then the parahoric restrictions of $\pi(ut_2,\varepsilon\otimes\mathbf{1})$ and $\pi(ut_2,\varepsilon\otimes\varepsilon)$ with respect to $K_{J_r}$ are irreducible representations labelled by $\theta_r'\boxtimes(\emptyset,1^{n-r})$ and $(\emptyset,1^r)\boxtimes\theta_{n-r}'$, respectively. For $r=0,1$, we set $\theta_r'=0$.
\end{proposition}

Assuming this last result, it is a routine calculation to prove Theorem \ref{thm:Cn_subregular}. Table \ref{tbl:par_res_virtual} contains the restrictions of the virtual characters $\Pi(u,s,h)$, from which the proof easily follows.


