\section{The p-adic symplectic group}

In this chapter, we verify that Conjecture \ref{conj:main} holds when $G=\Sp_{2n}(F)$ for some explicit nontrivial examples. In this case, $G$ is a simply connected simple $p$-adic group, whose complex dual group is $G^\vee=\SO_{2n+1}(\CC)$. In particular, $G$ has no pure inner twists and all maximal compact subgroups are maximal parahoric subgroups. Following the remark after Conjecture \ref{conj:main}, we will prove the result for specific unipotent conjugacy classes of $\SO_{2n+1}(\CC)$ and maximal parahoric subgroups.

Firstly, we note that the conjecture is true if $u$ is a regular unipotent element of $G^\vee$. It is well-known that $\dim Z_{G^\vee}(u)=\rk\SO_{2n+1}=n$ and that $\Gamma_u=Z(G^\vee)=1$ and consequently $\mathcal{Y}(\Gamma_u)_{\text{ell}}=\{(1,1)\}$. The representation $\Pi(u,1,1)=\pi(u,1,\mathbf{1})$ is fixed by $\FT^{\text{ell}}$ and moreover the Springer correspondence implies that 
$$\res_{K_i}\pi(u,1,\mathbf{1})=\St_{K_i}\quad\text{for any maximal parahoric }K_i\subset G,$$
and this representation is also fixed by each $\FT^{K_i}$, so the conjecture is verified.

\subsection{Structure theory of \texorpdfstring{$\SO_{2n+1}(\CC)$}{PDFstring}}

To progress any further, we need to understand some structural properties of $G^\vee=\SO_{2n+1}(\CC)$ and representation theoretic aspects of the reductive quotients of maximal parahoric subgroups of $G=\Sp_{2n}(F)$, which we discuss now. We recall that 
$$\SO_{2n+1}=\{A\in\GL_{2n+1}\ |\ AJA^T=J\},\quad\text{where}\quad J=\begin{pmatrix}
    1 & & \\
     & & I_n \\
     & I_n & \\
\end{pmatrix}$$
The diagonal matrices in $\SO_{2n+1}(\CC)$ are
$$T_n=\left\{h(a_1,\ldots,a_n):=\begin{pmatrix}
1 & & & & & & \\
& a_1 & & & & & \\
& & \ddots& & & & \\
& & & a_n & & & \\
& & & & a_1^{-1} & & \\
& & & & & \ddots& \\
& & & & & & a_n^{-1}\\
\end{pmatrix}\ |\ a_1,\ldots,a_n\in\CC^\times\right\},$$
and they form a maximal torus in $\SO_{2n+1}(\CC)$. We fix a set of simple roots $\Delta_n=\{\alpha_1,\alpha_2,\ldots,\alpha_n\}$ given by 
\begin{equation*}
    \alpha_i(h(a_1,\ldots,a_n))=\begin{cases}
        a_1^{-1} & \text{ if } i=1,\\
        a_{i-1}a_i^{-1} & \text{ if } 2\leq i\leq n.
    \end{cases}
\end{equation*}
The simple root $\alpha_1$ is the only short one and the corresponding Dynkin diagram is
\textcolor{red}{Insert/draw Dynkin diagram}.

The corresponding simple coroots are 
\begin{equation*}
    \calpha_i(t)=\begin{cases}
        h(t^{-2},1,\ldots,1) & \text{ if } i=1,\\
        h(1,\ldots,1,t,t^{-1},1,\ldots,1) & \text{ if } 2\leq i\leq n.
    \end{cases}
\end{equation*}

The Weyl group $W(B_n)=N(T_n)/T_n$ is isomorphic to the group $S_n\rtimes C_2^n$ and it acts on $T_n$ faithfully by the transformations 
$$h(a_1,\ldots,a_n)\longmapsto h(a_{\sigma(1)}^{\pm1},\ldots,a_{\sigma(n)}^{\pm1}),\quad\text{where }\sigma\in S_n.$$
Explicitly, $W(B_n)=\langle w_{\alpha_1}\text{ (short reflection)}, w_{\alpha_2},\ldots,w_{\alpha_n}\text{ (long reflections)}\rangle$ and the simple reflections act by
\begin{equation*}
    w_{\alpha_i}\cdot h(a_1,\ldots,a_n)=\begin{cases}
        h(a_1^{-1},a_2,\ldots,a_n) & \text{ if } i=1,\\
        h(a_1,\ldots,a_{i},a_{i-1},\ldots,a_n) & \text{ if } 2\leq i\leq n.
    \end{cases}
\end{equation*}

% \subsection{Unipotent conjugacy classes of \texorpdfstring{$\SO_{2n+1}(\CC)$}{PDFstring}}

Next, we study the structure of unipotent conjugacy classes of $\SO_{2n+1}(\CC)$. We recall from Section \ref{subsec:unipotent_classes} that each complex simple group has three canonical classes of unipotent elements: $C_{\reg}, C_{subreg}$ and $C_{\min}$. For $\SO_{2n+1}(\CC)$, the unipotent conjugacy classes can be parametrized as follows.

\begin{proposition}\label{prop:unip_odd_so}
    The conjugacy classes of $\SO_{2n+1}(\CC)$ are parametrized by pairs $(\lambda,\mu)$ of permutations such that $2|\lambda|+|\mu|=2n+1$ and $\mu$ has distinct odd parts and no even parts. The regular orbit is parametrized by $(\emptyset,2n+1)$, the subregular orbit by $(1,2n-1)$, the minimal orbit by $(21^{n-2},1)$ and the zero orbit by $(1^n,1)$.
\end{proposition}

In addition to this result, one can also find nice representatives for each of these orbits. The results are summarized in the following table \textcolor{red}{(need to reference this well)}.

\begin{table}[!ht]
    \begin{tabular}{c|c|c|c|c|c}
        Unipotent & $(\lambda,\mu)$ & Dynkin Diagram & $\dim Z_{G^\vee}(u)$ &$\dim\mathcal{B}^u$ & $\Gamma_u$ \\ \hline
        $\prod_{i=1}^n x_{\alpha_i}(1)$ & $(\emptyset,2n+1)$ & & $n$ & $0$ & $1$ \\
        $\prod_{i=1}^{n-1} x_{\alpha_i}(1)$ & $(1,2n-1)$ & & $n+2$ & $1$ & $\CC^\times\rtimes C_2$ \\
        ... & ... & ... & ... & ... & ... \\
        $x_{\alpha_2}(1)$ & $(21^{n-2},1)$ & & $2n^2-3n+4$ & $n^2-2n+2$ & $B_{n-2}\times A_1$ \\
        $1$ & $(1^n,1)$ & & $2n^2+n$ & $n^2$ & $\SO_{2n+1}$     
    \end{tabular}
\end{table}

In the next subsection, we will prove that Conjecture \ref{conj:main} holds for the subregular unipotent orbit containing $u=\prod_{i=1}^{n-1}x_{\alpha_i}(1)$. Before we do this, however, we need to discuss the representation theory of parahoric reductive quotients of $\Sp_{2n}(F)$.

\subsection{Unipotent representations of \texorpdfstring{$\Sp_{2n}(\FF_q)$}{PDFstring}}

We recall from Proposition \ref{prop:harishchandra} that for any unipotent representation $\chi$ of $\Sp_{2n}(\FF_q)$ there exists a parahoric subgroup $P\subseteq\Sp_{2n}(\FF_q)$ with Levi decomposition $P=LN$ and a cuspidal unipotent character $\phi$ of $L$ such that $\chi\hookrightarrow\Ind_{P}^{\Sp_{2n}(\FF_q)}\phi$ and that moreover the pair $(P,\phi)$ are unique up to conjugacy. In the 90s (\textcolor{red}{check this}), Lusztig showed that finite groups of Lie type of type $A_m, m\geq1$ have no cuspidal unipotent representations, while those of type $C_m$ have a unique cuspidal unipotent representation $\phi_s$ if $m=s^2+s$ for some $s\geq0$, and none otherwise. 

For each $s\geq0$ such that $m:=s^2+s\leq n$, let $P_s=L_sN_s$ be the standard parabolic of type $C_m$ and let $\phi_s$ be the unique cuspidal unipotent representation of $L_s$. The unipotent representations of $\Sp_{2n}(\FF_q)$ are the components of the representations
\begin{equation*}
    \Ind_{P_s}^{\Sp_{2n}(\FF_q)}\phi_s\quad\text{for all }s\geq0\text{ satisfying }s^2+s\leq n.
\end{equation*}

For each choice of $s$, the components of $\Ind_{P_s}^{\Sp_{2n}(\FF_q)}\phi_s$ define a \textit{series} of unipotent representations of $\Sp_{2n}(\FF_q)$ denoted by $\mathcal{E}(\Sp_{2n}(\FF_q),P_s,\phi_s)$. Let us focus first on $s=0$, corresponding to the principal series representations. In this case, $P_0=B$ is a Borel subgroup, $L_0=T$ is a maximal torus and $\phi_0=\mathbf{1}$, so we aim to characterize the components of $\Ind_B^{\Sp_{2n}(\FF_q)}\mathbf{1}$. In \ref{subsec:unipotent_families}, we proved that there is a canonical bijection between principal series representations of $\Sp_{2n}(\FF_q)$ and irreducible representations of its Weyl group $W(C_n)\cong S_n\rtimes C_2^l$. The Weyl group $W(B_n)$ of $\SO_{2n+1}(\CC)$ is canonically isomorphic to $W(C_n)$ by swapping between short and long simple reflections. Thus irreducible representations of $W(B_n)$ also parametrize principal series representations of $\Sp_{2n}(\FF_q)$. We shall abuse notation and denote both types of representations with the same symbol -- it will be clear from context whether we refer to principal series representations of $\Sp_{2n}(\FF_q)$ or irreducible representations of $W(B_n)$. 

\begin{theorem}\label{thm:WBn_reps}
    There is a natural one-to-one correspondence between irreducible representations of $W(B_n)$ and ordered pairs of partitions (or bipartitions) $(\alpha,\beta)$ with $|\alpha|+|\beta|=n$. Let $\alpha=(\alpha_1,\alpha_2,\ldots)$ and $\beta=(\beta_1,\beta_2,\ldots)$ with 
    $$0\leq\alpha_1\leq\alpha_2\ldots\quad 0\leq\beta_1\leq\beta_2\ldots,$$
    where some of the $\alpha_i,\beta_i$ may be $0$. Let $\alpha^*,\beta^*$ be the dual partitions of $\alpha,\beta$. Then consider a subsystem $\Phi'\subseteq\Phi$ of type 
    $$D_{\alpha_1^*}+D_{\alpha_2^*}+\cdots+B_{\beta_1^*}+B_{\beta_2^*}+\cdots,$$
    where $D_1$ is the empty root system and $D_2=A_1\times A_1$. If $W'$ is the Weyl group of $\Phi'$, then the corresponding irreducible representation $\phi_{\alpha,\beta}$ of $W(B_n)$ is the Macdonald representation ${j_{W'}}^{W}(\varepsilon_{W'})$ obtained from $W'$. Moreover, the dimension of $\phi_{\alpha,\beta}$ is given by the formula
    $$\dim\phi_{\alpha,\beta}=\frac{n!}{\mathrm{hook}(\alpha)\mathrm{hook}(\beta)}.$$
\end{theorem}

\begin{example}
    Let us discuss some important examples of the above discussion theorem.
    \begin{enumerate}
        \item If $(\alpha,\beta)=(n,\emptyset)$, then $\Phi'=\emptyset$ and $W'=\{1\}$. It follows that $\phi_{(n,\emptyset)}={j_{\{1\}}}^{W}(\varepsilon_{\{1\}})$ is the trivial representation.
        \item If $(\alpha,\beta)=(\emptyset, 1^n)$, then $\Phi'=\Phi$ and $W'=W$. It follows that $\phi_{(\emptyset,1^n)}={j_{W}}^{W}(\varepsilon_{W})=\varepsilon$ is the sign representation.
        \item If $(\alpha,\beta)=(n-1,1)$, then $\Phi'=\{\pm\alpha_1\}$ and $W=\langle w_{\alpha_1}\rangle$. It follows that $\phi_{(n-1,1)}={j_{\langle w_{\alpha_1}\rangle}}^{W}(\varepsilon_{\langle w_{\alpha_1}\rangle})$ is the natural reflection representation.
    \end{enumerate}
\end{example}

Throughout, we will label the representations of $W(B_n)$ by $(\alpha,\beta)$ instead of $\phi_{\alpha,\beta}$. As a corollary of this construction, we can describe the effect of twisting representations by the sign representation $\varepsilon=(\emptyset,1^n)$.

\begin{lemma}\label{lem:twist_sign}
    Let $(\alpha,\beta)$ be a representation of $W(B_n)$ labelled as in Theorem \ref{thm:WBn_reps} and let $\alpha^*,\beta^*$ be the dual representations. Then 
    $$\varepsilon\otimes(\alpha,\beta)=(\emptyset,1^n)\otimes(\alpha,\beta)=(\beta^*,\alpha^*).$$
    In particular, if $\phi_1$ and $\phi_2$ are two characters of the same family, so are $\varepsilon\otimes\phi_1$ and $\varepsilon\otimes\phi_2$.
\end{lemma}

Next, we wish to understand, using the previous parametrization, the families of the Weyl group as defined in \ref{subsec:unipotent_families}. To do this, we need to introduce the notion of \textit{symbols} associated to a pair of partitions $(\alpha,\beta)$. Choose an appropriate number of zeros as parts of $\alpha$ or $\beta$ so that $\alpha$ has one more part than $\beta$. Then define the symbol of $(\alpha,\beta)$ to be the array

\textcolor{red}{Insert here picture page 375 Carter}

We consider the equivalence relation on the symbols generated by 

\textcolor{red}{Insert here picture page 375 Carter}

Thus, each pair $(\alpha,\beta)$ defines a unique equivalence class of symbols. 

\begin{theorem}
    Two characters of $W(B_n)$ lie in the same family if and only if they possess symbols for which the unordered sets $\{\lambda_1,\ldots,\lambda_{m+1},\mu_1,\ldots,\mu_m\}$ are the same. Moreover, each family contains a unique character whose symbol has the property that 
    $$\lambda_1\leq\mu_1\leq\lambda_2\leq\cdots\leq\mu_m\leq\lambda_{m+1},$$
    and these are precisely the special characters of $W(B_n)$.
\end{theorem}

This concludes the classification of principal series representations of $\Sp_{2n}(\FF_q)$. To understand the remaining series, we need to consider more general symbols of the form

\textcolor{red}{Insert here picture page 466 Carter}

where $0\leq\lambda_1<\lambda_2<\ldots<\lambda_a$, $0\leq\mu_1<\mu_2<\ldots<\mu_b$, $a-b$ is odd and positive and $\lambda_1,\mu_1$ are not both $0$. We define the \textit{defect} of such a symbol as $d=a-b$ and its \textit{rank} as
$$\sum_{i=1}^{a}\lambda_i+\sum_{j=1}^{b}\mu_j-\left[\left(\frac{a+b-1}{2}\right)^2\right].$$


\begin{theorem}
    Unipotent characters of $\Sp_{2n}(\FF_q)$ are parametrized in terms of symbols of the form as above whose rank equals $n$. Such a character is a component of $\Ind_{P_s}^{\Sp_{2n}(\FF_q)}\phi_s$, where $d=2s+1$. Finally, two unipotent characters lie in the same family of characters if and only if their symbols contain the same entries with the same multiplicities.
\end{theorem}

\subsection{Springer correspondence of \texorpdfstring{$\SO_{2n-1}(\CC)$}{PDFstring}}



\subsection{The subregular unipotent orbit of \texorpdfstring{$\SO_{2n+1}(\CC)$}{PDFstring}}
In this section, we prove Conjecture \ref{conj:main} for the subregular unipotent element. We recall from Proposition \ref{prop:unip_odd_so} that the subregular unipotent orbit is parametrized by $(1,2n-1)$. To simplify notation, let $\alpha^0=\alpha_1+\alpha_2+\cdots+\alpha_n$ be the highest short root of $\Phi(B_n)$.

\begin{lemma}
    The unipotent element $u=\exp(e_{\alpha_1}+e_{\alpha_2}+\cdots+e_{\alpha_{n-1}})$ lies in the subregular orbit and the reductive part of its centralizer is $\Gamma_u=\Gamma_u^0\rtimes\langle h_uw_{\alpha^0}\rangle$, where
    $$\Gamma_u^0=\{\check{\alpha^0}(t)\ |\ t\in\CC^\times\}\quad\text{and}\quad h_u=\begin{psmallmatrix}
    1 & \\
     & -I_{2n}
    \end{psmallmatrix}=\begin{cases}
        \calpha_1(i)\calpha_3(-1)\cdots\calpha_n(-1) & \text{ if $n$ is odd,}\\
        \calpha_2(-1)\calpha_4(-1)\cdots\calpha_n(-1) & \text{ if $n$ is even.}
    \end{cases}$$
\end{lemma}
\begin{proof}
    
\end{proof}

The above result implies that there is an isomorphism 
$$\Gamma_u\cong\langle z,\delta\ |\ z\in\CC^\times,\delta^2=1,\delta z\delta^{-1}=z^{-1}\rangle\cong C^\times\rtimes C_2$$
given by 
\begin{align*}
    1\longleftrightarrow s_0=\check{\alpha^0}(\pm1)=I_{2n+1},\quad -1\longleftrightarrow s_1=\check{\alpha^0}(\pm i)=\begin{psmallmatrix}
        1 & & & &  \\
        & I_{n-1} & & &\\
        & & -1 & & \\
        & & & I_{n-1} &\\
        & & & & -1\\
    \end{psmallmatrix},\quad
    \delta\longleftrightarrow s_2=h_uw_{\alpha^0}=\begin{psmallmatrix}
        -1 & & & & \\
        & -I_{n-1} & & & \\
        & & & & 1\\
        & & & -I_{n-1} &\\
        & & 1 & &
    \end{psmallmatrix}.
\end{align*}
In particular, we note that $\Gamma_{us_0}=\Gamma_{us_1}=\Gamma_u$ so $A_{us_0}=A_{us_1}=C_2$ while $\Gamma_{us_2}=A_{us_2}=\{s_0,s_1,s_2,s_1s_2\}=C_2\times C_2$. We label its representations by $\mathbf{1}\otimes\mathbf{1}$, $\mathbf{1}\otimes\varepsilon$, $\varepsilon\otimes\mathbf{1}$, $\varepsilon\otimes\varepsilon$, where the first term indicates the action of $s_1$ and the second term indicates the action of $s_2$.
\begin{lemma}
    The group $\Gamma_u$ has six elliptic pairs up to $\Gamma_u$-conjugacy:
    $$\Gamma_u\backslash\mathcal{Y}(\Gamma_u)_{\mathrm{ell}}=\{(\pm1,\delta),(\delta,\pm1),(\delta,\pm\delta)\}.$$
\end{lemma}
\begin{proof}
    All elements in $\Gamma_u$ are semisimple, and any two $\gamma_1,\gamma_2\in\Gamma_u$ commute if any only if $\gamma_1,\gamma_2\in\CC^\times$ (in which case their common centralizer is infinite) or 
    $(\gamma_1,\gamma_2)\in\{(\pm1,z\delta),(z\delta,\pm1),(z\delta,\pm z\delta)\ | \ z\in\CC^\times\}$. It is easy to see that these pairs have finite centralizer and are $\CC^\times$-conjugate to $\{(\pm1,\delta),(\delta,\pm1),(\delta,\pm\delta)\}$, respectively.
\end{proof}

Thus, we wish to compute the parahoric restriction of the virtual representations
\begin{align*}
    \Pi(u,1,\delta)=\pi(us_0,\mathbf{1})-\pi(us_0,\varepsilon),\quad \Pi(u,-1,\delta)=\pi(us_1,\mathbf{1})-\pi(us_1,\varepsilon),\\
    \Pi(u,\delta,1)=\pi(us_2,\mathbf{1}\otimes\mathbf{1})+\pi(us_2,\mathbf{1}\otimes\varepsilon)+\pi(us_2,\varepsilon\otimes\mathbf{1})+\pi(us_2,\varepsilon\otimes\varepsilon),\\
    \Pi(u,\delta,-1)=\pi(us_2,\mathbf{1}\otimes\mathbf{1})+\pi(us_2,\mathbf{1}\otimes\varepsilon)-\pi(us_2,\varepsilon\otimes\mathbf{1})-\pi(us_2,\varepsilon\otimes\varepsilon),\\
    \Pi(u,\delta,\delta)=\pi(us_2,\mathbf{1}\otimes\mathbf{1})-\pi(us_2,\mathbf{1}\otimes\varepsilon)+\pi(us_2,\varepsilon\otimes\mathbf{1})-\pi(us_2,\varepsilon\otimes\varepsilon),\\
    \Pi(u,\delta,-\delta)=\pi(us_2,\mathbf{1}\otimes\mathbf{1})-\pi(us_2,\mathbf{1}\otimes\varepsilon)-\pi(us_2,\varepsilon\otimes\mathbf{1})+\pi(us_2,\varepsilon\otimes\varepsilon).
\end{align*}

The set of affine roots of $\Sp_{2n}(F)$ is $S_{\aff}=\{\calpha_1,\calpha_2,\ldots,\calpha_n,\check{\alpha^0}\}$. We recall from Lemma \ref{lem:multiplicity} and Proposition \ref{prop:second_red} that if $\pi(us,\rho)$ has Iwahori-fixed vectors and $J\subsetneq S_{\aff}$, for any irreducible $K_J$ module $\chi$ trivial on $U_J$, we have that 
$$\langle\chi,V^{U_J}\rangle_{K_J}=\langle\chi^{\mathcal{I}},V^{\mathcal{I}}\rangle_{\mathcal{H}(K_J,\mathcal{I},\mathbf{1})}=\langle\chi^{\mathcal{I}}_{q=1},\pi(us,\rho)^{\mathcal{I}}_{q=1}\rangle_{\widetilde{W}_J}.$$

The operation $\chi\mapsto\chi^{\mathcal{I}}_{q=1}$ corresponds to the aforementioned bijection between principal series representations of $\overline{K}_J$ and irreducible representations of $\widetilde{W}_J$, while 
$$\pi(us,\rho)^{\mathcal{I}}_{q=1}=\varepsilon\otimes\Ind_{\widetilde{W}_s}^{\widetilde{W}}[s\otimes H(\mathcal{B}_s^u)^\rho]$$
as a $\widetilde{W}$-module, up to semisimplification. We first compute this restriction with respect to the hyperspecial parahoric subgroup $K_0$ of $\Sp_{2n}(F)$ corresponding to $J_0=\{\calpha_1,\calpha_2,\ldots,\calpha_n\}$, with reductive quotient $\overline{K}_0=\Sp_{2n}(\FF_q)$ and Weyl group $\widetilde{W}_{J_0}=W$.

To do this, one distinguish between the Iwahori-spherical representations, whose restrictions to $\overline{K}_0$ will be a direct sum of principal series representations, and non Iwahori-spherical representations having $U_{J_0}$ fixed vectors, whose restrictions to $\overline{K}_0$ will contain irreducible components of $\Ind_{P_s}^{\Sp_{2n}(\FF_q)}\phi_s$ for some $s\geq 1$.

We first fix our attention towards Iwahori-spherical representations. If $\pi(us,\rho)$ is such a representation, then
\begin{equation}\label{eqn:hyperspecial_rest}
    \pi(us,\rho)^{\mathcal{I}}_{q=1}|_{W}=\varepsilon\otimes\Ind_{W_s}^W H(\mathcal{B}_s^u)^\rho,
\end{equation}
and the $W_s$-module $H(\mathcal{B}_s^u)$ can be computed using the Springer correspondence. In this section, we need the following result.

\begin{lemma}\label{lem:springer_subregular}
    The graded complex vector space $H(\mathcal{B}^u)=H^0(\mathcal{B}^u)\oplus H^2(\mathcal{B}^u)$ has a natural action of the group $W(B_n)\times A_u$ preserving the grading. The $W(B_n)$-modules $H^0(\mathcal{B}^u)^\mathbf{1}$, $H^2(\mathcal{B}^u)^\mathbf{1}$ and $H^2(\mathcal{B}^u)^\varepsilon$ afford the representations labelled by $(n,\emptyset)$, $(n-1,1)$ and $((n-1)1,\emptyset)$, respectively, while $H^0(\mathcal{B}^u)^\varepsilon=0$
\end{lemma}

From this, we can immediately compute that 
\begin{align*}
    \pi(us_0,\mathbf{1})^\mathcal{I}_{q=1}|_W&=\varepsilon\otimes H(\mathcal{B}^u)^\mathbf{1}=\varepsilon\otimes\left[(n,\emptyset)+(n-1,1)\right]=(\emptyset,1^n)+(1,1^{n-1}),\\
    \pi(us_0,\varepsilon)^\mathcal{I}_{q=1}|_W&=\varepsilon\otimes H(\mathcal{B}^u)^\varepsilon=\varepsilon\otimes((n-1)1,\emptyset)=(\emptyset,21^{n-2}),
\end{align*}

The analogous computation for $\pi(us_1,\mathbf{1})^\mathcal{I}_{q=1}|_W$ and $\pi(us_1,\varepsilon)^\mathcal{I}_{q=1}|_W$ is significantly more involved. Firstly, we note that 
\begin{align*}
    Z_{G^\vee}(s_1)=Z_{G^\vee}(s_1)^0\times\langle w_{\alpha^0}\rangle\quad\text{where}\quad
    Z_{G^\vee}(s_1)^0=\langle T,\mathfrak{X}_{\pm\alpha_1},\mathfrak{X}_{\pm\alpha_2},\ldots,\mathfrak{X}_{\pm\alpha_{n-1}}\rangle
\end{align*}
is a connected reductive group of type $B_{n-1}$. The Weyl group of $Z_{G^\vee}(s_1)$ is 
$$W_{s_1}=\langle w_{\alpha_1},w_{\alpha_2},\ldots,w_{\alpha_{n-1}}\rangle\times\langle w_{\alpha^0}\rangle\cong W(B_{n-1})\times C_2$$ 
and acts on $T$ by the transformations 
$$h(a_1,\ldots,a_{n-1},a_n)\longmapsto h(a_{\tau(1)}^{\pm1},\ldots,a_{\tau(n-1)}^{\pm1},a_n^{\pm1}),\quad\text{where }\tau\in S_{n-1}.$$
Thus $|W_{s_1}|=2^n(n-1)!$ and the index inside $W$ is $[W:W_{s_1}]=n$. From the descriptions above, $u\in Z_{G^\vee}(s_1)$ and is a regular unipotent element, so $\dim\mathcal{B}^u_{s_1}=0$. However, since $Z_{G^\vee}(s_1)$ is not connected, the structure of $\mathcal{B}^u_{s_1}$ is more interesting.

\begin{lemma}
    The variety $\mathcal{B}^u_{s_1}$ consists of two points. The two dimensional complex vector space $H^0(\mathcal{B}^u_{s_1})$ has a natural action of $W_{s_1}\times A_{us_1}$ and $H^0(\mathcal{B}^u_{s_1})^\mathbf{1}$, $H^0(\mathcal{B}^u_{s_1})^\varepsilon$ are one-dimensional representations of $W_{s_1}=\langle w_{\alpha_1},\ldots,w_{\alpha_{n-1}}\rangle\times\langle w_{\alpha^0}\rangle\cong W(B_{n-1})\times C_2$ affording the characters $(n-1,\emptyset)\otimes\mathbf{1}$, $(n-1,\emptyset)\otimes\varepsilon$, respectively.
\end{lemma}
\begin{proof}
    Since $u\in Z_{G^\vee}(s_1)$ is regular, it is contained in a unique Borel subgroup $B=\langle T,\mathfrak{X}_{\alpha_1},\mathfrak{X}_{\alpha_2},\ldots,\mathfrak{X}_{\alpha_{n-1}}\rangle$. We know that $s_2=h_uw_{\alpha^0}\in Z_{G^\vee}(u)-B$, so $u\in \prescript{s_2}{}{B}$. but $B\neq\prescript{s_2}{}{B}$. \textcolor{red}{Since $A_u=C_2$, these are the only two Borel subgroups containing $u$.}

    We note that $A_{us_1}=\{1,s_2\}$ and that $s_2$ acts on $\mathcal{B}^u_{s_1}$ by permuting both points and therefore 
    $$H^0(\mathcal{B}^u_{s_1})^\mathbf{1}=\CC(B+\prescript{s_2}{}{B}),\quad\text{while}\quad H^0(\mathcal{B}_{s_1}^u)^\varepsilon=\CC(B-\prescript{s_2}{}{B})$$


    By the classical Springer correspondence of type $B_{n-1}$ applied to the regular orbit containing $u$, we have that $\langle w_{\alpha_1},\ldots,w_{\alpha_{n-1}}\rangle$ acts trivially on $H^0(\mathcal{B}^u_{s_1})$. 
    On the other hand, since $s_2\in N(T_n)$ maps to $w_{\alpha^0}$, the action of $w_{\alpha^0}$ on $H^0(\mathcal{B}_{s_1}^u)$ coincides with that of $s_2$.
    \textcolor{red}{Maybe should justify this with the generalized Springer correspondence for disconnected reductive groups.}
\end{proof}
 
From these structural results, we can compute the parahoric restrictions of $\pi(us_1,\mathbf{1})$ and $\pi(us_1,\varepsilon)$.

\begin{lemma}\label{lem:indWs1}
    With the notation as above, we have that, as $W$-modules,
    $$\pi(us_1,\mathbf{1})^\mathcal{I}_{q=1}|_W=(\emptyset,1^n)+(\emptyset,21^{n-2})\quad\text{and}\quad\pi(us_1,\varepsilon)^\mathcal{I}_{q=1}|_W=(1,1^{n-1}).$$
\end{lemma}
\begin{proof}
    This is a direct calculation. From \refeq{eqn:hyperspecial_rest}, we know that 
    \begin{align*}
        \pi(us_1,\mathbf{1})^\mathcal{I}_{q=1}|_W&=\varepsilon\otimes\Ind_{W_{s_1}}^W H(\mathcal{B}^u_{s_1})^\mathbf{1}=\varepsilon\otimes\Ind_{W(B_{n-1})\times C_2}^W (n-1,\emptyset)\otimes\mathbf{1},\\
        \pi(us_1,\varepsilon)^\mathcal{I}_{q=1}|_W&=\varepsilon\otimes\Ind_{W_{s_1}}^W H(\mathcal{B}^u_{s_1})^\varepsilon=\varepsilon\otimes\Ind_{W(B_{n-1})\times C_2}^W (n-1,\emptyset)\otimes\varepsilon,
    \end{align*}
    and thus we analyze each induced representation. Recall that $(n-1,1)$ is the natural $n$-dimensional reflection representation of $W(B_n)$ acting on the $\CC$-span of $\alpha_1,\ldots,\alpha_n$. The line $l$ spanned by the root $\alpha^0=\alpha_1+\ldots+\alpha_n$ is a $1$-dimensional $W_{s_1}=\langle w_{\alpha_1},\ldots,w_{\alpha_{n-1}}\rangle\times\langle w_{\alpha^0}\rangle$-stable subspace, where $w_{\alpha_1},\ldots,w_{\alpha_{n-1}}$ act trivially while $w_{\alpha^0}$ acts by $\varepsilon$. Thus 
    $$\Hom_W(\Ind_{W(B_{n-1})\times C_2}^W (n-1,\emptyset)\otimes\varepsilon,(n-1,1))=\Hom_{W(B_{n-1})\times C_2}((n-1,\emptyset)\otimes\varepsilon,(n-1,1))=\CC,$$
    and since $[W:W_{s_1}]=n=\dim(n-1,1)$, it follows that $\Ind_{W_{s_1}}^W H(\mathcal{B}^u_{s_1})^\varepsilon=(n-1,1)$. We then twist by $\varepsilon$ to obtain 
    $$\pi(us_1,\varepsilon)^\mathcal{I}_{q=1}|_W=(1,1^{n-1}).$$

    \iffalse On the other hand, we note that there is a double space decomposition $W=W_{s_1}\sqcup W_{s_1}w_{\alpha_n}W_{s_1}$, and since $H^0(\mathcal{B}^u_{s_1})^\mathbf{1}$ affords the trivial $W_{s_1}$ representation, by combining Mackey theory and Frobenius reciprocity we observe that 
    \begin{equation*}
        \langle\Ind_{W_{s_1}}^W H(\mathcal{B}^u_{s_1})^\mathbf{1},\Ind_{W_{s_1}}^W H(\mathcal{B}^u_{s_1})^\mathbf{1}\rangle_W=|W_{s_1}\backslash W/W_{s_1}|=2,
    \end{equation*}
    so $\Ind_{W_{s_1}}^W H(\mathcal{B}^u_{s_1})^\mathbf{1}=(n,\emptyset)+(\alpha,\beta)$, where $(\alpha,\beta)$ is a $n-1$-dimensional irreducible representation of $W(B_n)$. 
    \fi

    On the other hand, the first statement of the Lemma will follow immediately if we prove that 
    \begin{equation}\label{eqn:IndWs1}
        \Ind_{W_{s_1}}^W H(\mathcal{B}^u_{s_1})^\mathbf{1}=\varepsilon\otimes\left((\emptyset,1^n)+(\emptyset,21^{n-2})\right)=(n,\emptyset)+((n-1)1,\emptyset).
    \end{equation}
    We note that $\dim(n,\emptyset)=1$ while $\dim((n-1)1,\emptyset)=n-1$, so the dimensions in \eqref{eqn:IndWs1} agree in both sides. Moreover, using the fact that $|W_{s_1}\backslash W/W_{s_1}|=2$, together with Mackey theory, one can easily show that the above induction decomposes into two irreducible subrepresentations. Since $H(\mathcal{B}^u_{s_1})^\mathbf{1}$ is the trivial representation of $W_{s_1}$, the trivial representation $(n,\emptyset)$ must also appear in the induction, so it remains to show that $((n-1)1,\emptyset)$ is the other component. Similarly to the computation above, it is enough by Frobenius reciprocity to show that $((n-1)1,\emptyset)$ contains a (necessarily unique) $1$-dimensional vector subspace on which $W_{s_1}$ acts trivially.

    To prove this last assertion, we need to analyze the construction of $((n-1)1,\emptyset)$ as a Macdonald representation of $W(B_n)$. By \textcolor{red}{Carter 11.4.2}, the root system associated to the bipartition $((n-1)1,\emptyset)$ has type $D_2=A_1\times A_1$, and by convenience we choose $\Phi'=\{\pm\alpha_n,\pm\alpha_0\}$ with corresponding Weyl group $W'=\langle w_{\alpha_n},w_{\alpha_0}\rangle\cong C_2^2$. Let $V$ be the $n$-dimensional vector space affording the natural reflection representation $(n-1,n)$ of $W(B_n)$. Since $\Phi'$ has two positive roots, the representation $((n-1)1,\emptyset)$ is a component of the representation $\Sym^2(V^*)$. By an analogous argument, the irreducible $n(n-1)/2$ - dimensional representation $(n-2,2)$ is also a subrepresentation of $\Sym^2(V^*)$. We now calculate explicitly a basis for each of these representations. 

    Let us fix an orthonormal basis $\{e_1,\ldots,e_n\}$ of $V$ such that $\alpha_1=e_1$ and $\alpha_i=e_i-e_{i-1}$ for $2\leq i\leq n$. Then one can calculate that $W(B_n)$ acts on $V^*\cong V$ by the transformations
    \begin{equation*}
        w_{\alpha_1}=\begin{psmallmatrix}
            -1 & 0\\
            0 & I_{n-1}
        \end{psmallmatrix},\quad\text{and}\quad w_{\alpha_i}=\begin{psmallmatrix}
            I_{i-2} & & &\\
            & 0 & 1 & \\
            & 1 & 0 & \\
            & & & I_{n-i}
        \end{psmallmatrix}\quad\text{for } 2\leq i\leq n.
    \end{equation*}
    Let $\{x_1,\ldots,x_n\}$ be the dual basis of $\{e_1,\ldots,e_n\}$ in $V^*$ so $\Sym^2(V^*)=\Span_\CC\{x_ix_j\ |\ 1\leq i\leq j\leq n\}$ is the space of degree $2$ homogeneous polynomials on $V$. There is a direct sum decomposition $\Sym^2(V^*)=U_1\oplus U_2\oplus U_3$ in $W(B_n)$-invariant subspaces, where  
    $$U_1=\Span_\CC\{\sum_{i=1}^n x_i^2\},\quad U_2=\Span_\CC\{x_{i-1}^2-x_i^2\ |\ 2\leq i\leq n\},\quad U_3=\Span_\CC\{x_ix_j\ | \ 1\leq i<j\leq n\}$$
    have dimensions $1$, $n-1$ and $n(n-1)/2$, respectively. In the preceding paragraph we discussed the fact that both $((n-1)1,\emptyset)$ and $(n-2,2)$ are components of $\Sym^2(V^*)$ and it thus follows that the subspaces $U_1$, $U_2$ and $U_3$ afford the representations $(n,\emptyset)$, $((n-1)1,\emptyset)$ and $(n-2,2)$ of $W(B_n)$, respectively. 
    
    It is now easy to see from this explicit model of $((n-1)1,\emptyset)$ that $W_{s_1}=\langle w_{\alpha_1},\ldots,w_{\alpha_{n-1}},w_{\alpha^0}\rangle$ acts trivially on the $1$-dimensional subspace of $\Span_\CC\{x_{i-1}^2-x_i^2\ |\ 2\leq i\leq n\}$ spanned by $$x_1^2+x_2^2+\cdots+x_{n-1}^2-(n-1)x_n^2.$$
    This immediately implies that 
    $$\Ind_{W(B_{n-1})\times C_2}^W (n-1,\emptyset)\otimes\mathbf{1}=(n,\emptyset)+((n-1)1,\emptyset),$$
    thus finishing the proof.
\end{proof}


Finally, we need to compute the parahoric restrictions of $\pi(us_2,\underline{ })$. To apply similar ideas to the previous case, we first note that $s_2$ is conjugate to the diagonal matrix $\begin{psmallmatrix}
    1 & \\
     & -I_{2n}
\end{psmallmatrix}$, and therefore 
\begin{align*}
    Z_{G^\vee}(s_2)=Z_{G^\vee}(s_2)^0\times\langle w_{\alpha_1}\rangle\quad\text{where}\quad
    Z_{G^\vee}(s_2)^0=\langle T,\mathfrak{X}_{\beta}\ |\ \beta\text{ is a long root}\rangle,
\end{align*}
where $Z_{G^\vee}(s_2)^0$ is a connected reductive group of type $D_n$. We can then deduce that $W_{s_2}=W$ and that $u$ is a regular unipotent element in $Z_{G^\vee}(s_2)$, so $\dim\mathcal{B}_{s_2}^u=0$. The following result gives an explicit description of $H(\mathcal{B}_{s_2}^u)=H^0(\mathcal{B}_{s_2}^u)$ as a $W$-representation.

\begin{lemma}
    The space cohomology space $H(\mathcal{B}_{s_2}^u)$ is $2$-dimensional and has a natural action of $W\times A_{us_2}$ where $A_{us_2}=\{s_0,s_1,s_2,s_1s_2\}$. It admits a direct sum decomposition
    $$H(\mathcal{B}_{s_2}^u)=H^0(\mathcal{B}_{s_2}^u)^{\mathbf{1}\otimes\mathbf{1}}\oplus H^0(\mathcal{B}_{s_2}^u)^{\varepsilon\otimes\mathbf{1}}$$
    into $1$-dimensional $W$-representations, affording the characters $(n,\emptyset)$ and $(\emptyset,n)$, respectively.
\end{lemma}

From this we immediately get that 
\begin{align*}
    \pi(us_2,\mathbf{1}\otimes\mathbf{1})=\varepsilon\otimes H(\mathcal{B}^u_{s_2})^{\mathbf{1}\otimes\mathbf{1}}=\varepsilon\otimes(n,\emptyset)=(\emptyset,1^n),\\
    \pi(us_2,\varepsilon\otimes\mathbf{1})=\varepsilon\otimes H(\mathcal{B}^u_{s_2})^{\varepsilon\otimes\mathbf{1}}=\varepsilon\otimes(\emptyset,n)=(1^n,\emptyset),
\end{align*}
while $\pi(us_2,\mathbf{1}\otimes\varepsilon)$ and $\pi(us_2,\varepsilon\otimes\varepsilon)$ are not Iwahori-spherical. 

\begin{lemma}
    The representation $\pi(us_2,\mathbf{1}\otimes\varepsilon)$ has no $U_{J_0}$ fixed vectors, while the parahoric restriction of $\pi(us_2,\varepsilon\otimes\varepsilon)$ to $\overline{K}_0$ is the defect $1$ irreducible $\Sp_{2n}(\FF_q)$ representation labelled by the symbol
    $$\begin{pmatrix}
        0 & 1 & 2 & \ldots & n-2 & n-1 & n\\
        && 1 & \ldots & n-2 &&
    \end{pmatrix}.$$
\end{lemma}

\begin{table}[]
    \centering
    \begin{tabular}{|c|c|}
        \hline
        $\pi(us,\phi)$                        & $K_0\rightarrow \Sp_{2n}(\FF_q)$ \\ \hline
        $(s_0,\mathbf{1})$                    & $(\emptyset,1^n)+(1,1^{n-1})$    \\ \hline
        $(s_0,\varepsilon)$                   & $(\emptyset,21^{n-2})$           \\ \hline
        $(s_1,\mathbf{1})$                    & $(\emptyset,1^n)+(\emptyset,21^{n-2})$     \\ \hline
        $(s_1,\varepsilon)$                   & $(1,1^{n-1})$                    \\ \hline
        $(s_2,\mathbf{1}\otimes\mathbf{1})$   & $(\emptyset,1^n)$                \\ \hline
        $(s_2,\mathbf{1}\otimes\varepsilon)$  & $0$                              \\ \hline
        $(s_2,\varepsilon\otimes\mathbf{1})$  & $(1^n,\emptyset)$                \\ \hline
        $(s_2,\varepsilon\otimes\varepsilon)$ & $\theta_{1,K_0}$                 \\ \hline
    \end{tabular}
\end{table}