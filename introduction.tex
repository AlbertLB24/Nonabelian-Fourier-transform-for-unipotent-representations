\section{Introduction}

The overarching aim of the Local Langlands Correspondence (LLC) is to classify the smooth admissible complex representations of reductive $p$-adic groups, of algebraic nature, in terms of (hopefully simpler) geometric and number theoretic data. More concretely, if $G$ is a reductive $p$-adic group over some non-archimedean field $F$, the Langlands philosophy predicts that the (smooth admissible) complex representations should be controlled by the Weil group $\mathcal{W}_F$ of the field $F$, together with the geometry of the complex dual group $G^\vee$ of $G$. Over the past few decades, the local Langlands correspondence has sparked remarkable interest, precise conjectures have been stated and many important results are now known. While the Weil group $\mathcal{W}_F$ is necessary to classify all representation of the $p$-adic group $G$, the geometry of the complex dual group $G^\vee$ is enough to classify a important class of representations of $G$. This class is commonly referred to as \textit{unipotent representations} of $G$ and denoted by $\Irr_{\un}(G)$, and in particular contains all representations containing a vector fixed by an Iwahori subgroup (usually denoted \textit{Iwahori-spherical} representations). The work of Kazhdan and Lusztig in \cite{KazhdanLusztig1987} first classified Iwahori-spherical representations in terms of geometric data related to $G^\vee$; however this classification ``missed'' to capture some important information. A few years later, Lusztig remedied this gap by defining the larger set of unipotent representations and considering simultaneously all such representations for all pure inner twists of $G$. Explicitly, in \cite{Lusztig1995} he proved that there is a natural bijection between the sets 
\begin{equation}\label{eqn:LLC_unipotent}
    \LLC^p:\bigsqcup_{G'}\Irr_{\un}(G')\longleftrightarrow G^\vee\backslash\{(x,\phi)\ |\ x\in G^\vee, \phi\in\widehat{A_x}\},
\end{equation}
where the union on the left is taken over all pure inner twists $G'$ of $G$ and $A_x$ is the component group of the centralizer $Z_{G^\vee}(x)$. The elements of the $L$-packets are then parametrized by the representations $\phi\in\widehat{A_x}$.

Fortunately, unipotent representations can be defined explicitly using the $p$-adic topology of $G$. Associated to $G$, there a polysimplial space, called the \textit{Bruhat--Tits} building of $G$ on which $G$ acts. This action beautifully captures many structural properties of the $p$-adic group and can be studied to understand representation theoretic aspects of $G$. In section \ref{sec:padic_unipotent}, we discuss this action in some detail. The main idea exploited by Lusztig is that, associated to each facet $c$ of the building, we can associate a certain filtration of open compact subgroups (first defined by Moy and Prasad in \cite{MoyPrasad1994}). The first terms of these filtrations $K_c$ are called \textit{parahoric} subgroups and the second terms are their pro-unipotent radical $U_c$. Then, if some representation of $G$ has vectors fixed by $U_c$ for some facet $c$, then one immediately obtains a representation of the quotient $\overline{K}_c:=K_c/U_c$, which happens to be a finite reductive group that can be described completely explicitly. Representations satisfying this property for some facet $c$ are called \textit{depth-zero} representations.

Unipotent representations are a proper subset of depth-zero representations. To define them, one needs to have a good understanding of the representation theory of the groups $\overline{K}_c$, commonly known as \textit{finite groups of Lie type}. In the $20$ years before the classification \eqref{eqn:LLC_unipotent} stated above, Lusztig and many others had been studying in depth the characters of these groups. In Deligne and Lusztig's groundbreaking paper \cite{DeligneLusztig1976}, deep geometric methods were used to construct characters for finite groups of Lie type. As a result of this geometric approach, a family of characters with remarkable properties, arose naturally. Characters in this family were called \textit{unipotent}. A representation of the $p$-adic group is therefore defined to be unipotent if there is some facet $c$ for which the corresponding $\overline{K}_c$-representation contains some unipotent character. This certainly includes Iwahori-spherical representations, for the quotient with its pro-unipotent radical is just a torus over a finite field, and the trivial character is indeed unipotent. The study of unipotent representations of $p$-adic groups is therefore deeply connected to the study of unipotent characters for finite groups of Lie type.

While many questions in the LLC have now been resolved in full generality, some of them remain open. One such question relates to the stability of $L$-packets. Moeglin and Walspurger tackled this problem in \cite{MoeglinWaldspurger1994} by considering the restriction of unipotent representations to parahoric subgroups, and then reducing them to representations of finite groups of Lie type. Work of Lusztig and others in the late 20th century revealed remarkable structure on unipotent characters, which Moeglin and Waldspurger exploited. Lusztig defined an involution on the space of virtual unipotent characters known as \textit{Lusztig's non-abelian} transform taking the irreducible characters to the so-called \textit{almost characters}. These characters can be obtained independently via the theory of character sheaves and, as a result, have nice geometrical and stability properties. 

The hope is that, if an involution on the space of unipotent representations of $G$ existed with analogous properties, one could use the resulting virtual characters to prove stability properties of $L$-packets. Moreover, it is believed by Lusztig that an analogous theory of character sheaves for $p$-adic groups should exist. Work of Ciubotaru \cite{Ciubotaru2020} and Aubert--Ciubotaru--Romano \cite{AubertCiubotaruRomano2024} has already considered this problem and formulated precise conjectures. We consider the same setting and similar notation to \cite{AubertCiubotaruRomano2024}, which we now describe.

Let us now explain the main results of this document more concretely. For a $p$-adic group $G$, let $\InnT^p(G)$ be the set of pure inner twists of $G$, a set in canonical bijection with $Z(G^\vee)$. Note that if $G$ is simply connected, it has no pure inner twists other than itself. For $G'\in\InnT^p(G')$ let $R_{\un}(G')$ be the complexification of the Grothendiek space spanned by $\Irr_{\un}(G')$ and consider the representation space $\oplus_{G'}R_{\un}(G)$ associated to $G$. In order to relate this space with the representation theory of finite groups of Lie type, we use open compact subgroups of $G'$. One of the main innovations in \cite{AubertCiubotaruRomano2024} was the realization that instead of considering maximal parahoric subgroups, we should instead consider maximal open compact subgroups. The distinction is that while reductive quotients of maximal parahoric subgroups are connected, this need not be the case for maximal open compact subgroups. The two notions coincide when $G$ is simply connected, but not otherwise. By taking fixed points under the pro-unipotent radical of each maximal open compact subgroup, we obtain the restriction map 
\begin{equation*}
    \res_{\un}^{\cpt}:\bigoplus_{G'\in\InnT^p} R_{\un}(G')\longrightarrow\mathcal{C}(G)_{\cpt,\un},\quad\text{where}\quad\mathcal{C}(G)_{\cpt,\un}=\bigoplus_{G'\in\InnT^p(G)}\bigoplus_{K'\in\max(G')}R_{\un}(\overline{K}'),
\end{equation*}
$\max(G')$ is the set of maximal open compact subgroups of $G'$ up to conjugacy and $R_{\un}(\overline{K}')$ is the $\CC$-span of the irreducible characters of $\overline{K}'$. This innovation unavoidably led to a second one. Lusztig's non-abelian Fourier transform was defined only for connected finite groups of Lie type, and in \cite[\S6]{AubertCiubotaruRomano2024} a generalization of this map was defined for disconnected groups, denoted by 
\begin{equation*}
    \FT^{\cpt}_{\un}:\mathcal{C}(G)_{\cpt,\un}\longrightarrow\mathcal{C}(G)_{\cpt,\un}.
\end{equation*}
In this document, we shall only consider simply connected $p$-adic groups, in which case we recover the classical case of maximal parahoric subgroups and Lusztig's nonabelian Fourier transform. The remaining ingredient to state the conjecture is, of course, the definition of the \textit{dual nonabelian Fourier transform} on the space $\oplus_{G'}R_{\un}(G')$ compatible with the restriction maps and the generalization of Lusztig's nonabelian Fourier transform $\FT_{\un}^{\cpt}$. The existence of such a map is conjectured, but no construction has been achieved on the entire space. Instead, previous work has focused on a certain subspace, (isometric to) the elliptic unipotent representation space $\mathcal{R}_{\un,\mathrm{ell}}^p(G)$ spanned by certain virtual characters $\Pi(u,s,h)$, where $u\in G^\vee$ and $(s,h)$ are \textit{elliptic pairs} defined by section \ref{subsec:dualFourier}. On this space, there is a natural definition for the dual nonabelian Fourier transform, namely
\begin{equation*}
    \FT^\vee_{\mathrm{ell}}:\mathcal{R}_{\un,\mathrm{ell}}^p(G)\longrightarrow\mathcal{R}_{\un,\mathrm{ell}}^p(G),\quad\Pi(u,s,h)\longmapsto\Pi(u,h,s).
\end{equation*}
We are finally ready to state the main conjecture.
\begin{conjecture}\label{conj:main}\cite[Conjecture 1.2]{AubertCiubotaruRomano2024}
    Let $G$ be a semisimple $p$-adic group. Then the diagram 
    \[
        \begin{tikzcd}
        \mathcal{R}_{\un,\text{ell}}^p(G) \arrow[r, "\FT^\vee_{\text{ell}}"] \arrow[d, "\res_{\un}^{\cpt}"] & \mathcal{R}_{\un,\text{ell}}^p(G) \arrow[d, "\res_{\un}^{\cpt}"] \\
        \mathcal{C}(G)_{\cpt,\un} \arrow[r, "\FT^{\cpt}_{\un}"] & \mathcal{C}(G)_{\cpt,\un},
        \end{tikzcd}
    \]
    commutes, up to certain roots of unity.
\end{conjecture}
In their paper, Aubert--Ciubotaru--Romano showed that this conjecture is true for $G=\SL_n(F),\PGL_n(F)$ or $\Sp_4(F)$. In previous work, Ciubotaru \cite{Ciubotaru2020} showed this conjecture holds for $G_2(F)$, and Waldspurger \cite{Waldspurger2018} showed it for $G=\SO_{2n+1}(F)$. The main results of this document naturally follow on from these. The first one is related to classical groups, while the second one is for exceptional ones.

\begin{theorem}[In progress, Theorem \ref{thm:Cn_subregular}] Let $G=\Sp_{2n}(F)$ and let $u$ be a subregular element of $G^\vee=\SO_{2n+1}(\CC)$. Then for all elliptic pairs $(s,h)$ related to $u$, 
$$\FT^{\cpt}_{\un}(\res_{\un}^{\cpt}(\Pi(u,s,h)))=\res_{\un}^{\cpt}(\Pi(u,h,s)).$$ 
    
\end{theorem}

\begin{theorem}[In progress, Theorem \ref{thm:F4}] Let $G$ be a simple $p$-adic group of type $F_4$. Then conjecture \ref{conj:main} holds for $G$.    
\end{theorem}

\subsection*{Structure of the document}
The document is organized as follows. We devote section $2$ to introduce finite groups of Lie type $G^F$ and Deligne--Lusztig theory. There is a vast amount of theory, so we focus on understanding the classification of unipotent characters of $G^F$ in terms of families. This naturally leads to the construction of Lusztig's nonabelian Fourier transform. In section $3$ we explain relevant structural and representation theoretic results of reductive $p$-adic groups $G$, where we introduce the Bruhat-Tits building, unipotent representations and the dual nonabelian Fourier transform. Section $4$ is the most technical; there we discuss in detail how to explicitly compute the restriction functors $\res_{\un}^{\cpt}$, a method developed by Reeder in \cite{Reeder2000}. Sections $5$ and $6$ contain original work and prove Theorems \ref{thm:Cn_subregular} and \ref{thm:F4} stated above.