\section{Structure theory and representations of p-adic groups}

\subsection{The apartment of a split maximal torus}

Let $F$ be a nonarchimedean local field with ring of integers $\cO$, uniformizer $\varpi$ and residue field $k$ of cardinality $q$, a power of a prime $p$. Let $\mathbf{G}$ be a connected, almost simple, split algebraic group over $F$ and let $G=\mathbf{G}(F)$. %Let $\Rep(G)$ be the category of admissible smooth complex representations of $G$ and let $\Irr(G)$ be the subset of the irreducible representations. We define $R(G)$ to be $\CC$-span of $\Irr(G)$ (i.e. the complexification of the Grothendieck group of the abelian category of smooth $G$-representations).

Let $T$ be a split maximal torus of $G$ over $F$, and let $X^*(T)$ (resp. $X_*(T)$) be its character (resp. cocharacter) lattice and let 
$$\langle\cdot,\cdot\rangle:X^*(T)\times X_*(T)\longrightarrow\ZZ$$
the natural perfect pairing between characters and cocharacters of $T$. Let $\Phi(G,T)\subset X^*(T)$ be the set of roots associated to $T$, with the corresponding set of coroots $\Phi^\vee(G,T)\subset X_*(T)$. We recall from the previous section that a choice of a Borel subgroup $B$ of $G$ containing $T$ is equivalent to the choice of simple roots $\Delta(G,T)=\{\alpha_1,\ldots,\alpha_r\}\subset\Phi(G,T)$, which we fix throughout. In addition, the group $B$ together with the normalizer $N:=N_G(T)$ form a $BN$-pair with corresponding Weyl group $W=N(F)/T(F)$. 

A natural object arising in the representation theory of $G$ is the apartment $\cA(G,T):=X_*(T)\otimes_\ZZ\RR$, a real vector space naturally containing all coroots. Moreover, $\cA(G,T)$ has the structure of a simplicial complex given by the hyperplanes
$$H_{\alpha,n}=\{x\in\cA(G,T)\ |\ \langle\alpha,x\rangle=n\},\quad\text{for each }\alpha\in\Phi(G,T)^+\text{ and } n\in\ZZ.$$
Whenever the torus $T$ is clear from context, we will omit it from the notation. The complexes on the apartment are called \textit{facets}, and the facets of largest dimension are called \textit{alcoves}. Our choice of simple roots $\Delta$ determines a canonical alcove
$$\mathcal{C}_0=\{x\in\cA\ |\ \langle\alpha,x\rangle>0, \alpha\in\Delta\ \text{ and } \langle\alpha_0,x\rangle<1\},$$
commonly referred to as the \textit{fundamental alcove}.

Another important property of the apartment is that it carries a natural action of the group $N$ satisfying
\begin{itemize}
    \item For any $\alpha\in\Phi$ and $\lambda\in F$, the element $\calpha(\lambda)\in T\subset N$ acts on $\cA$ by a translation $-\nu_p(\lambda)\calpha$.
    \item The centre of $G$ acts faithfully and fixes every alcove. \textcolor{red}{maybe important to explain this better?}
    \item For any $\alpha\in\Phi$, the element $w_\alpha(1)\in N$ acts on $\cA$ by a reflection along $H_{\alpha,0}$. This coincides with the natural action of $W$ on $\cA$.
\end{itemize}
This action preserves the simplicial structure of the apartment and is transitive on the set of alcoves of $\cA$. Moreover, the kernel of this action is $T(\cO)$ and therefore the \textit{extended Weyl group}
$$\widetilde{W}:=N(F)/T(\cO)\cong W\ltimes X_*(T)$$
acts faithfully on the apartment $\cA$ and transitively on the set of alcoves. We denote by $w_{\alpha,n}$ the unique element in $\widetilde{W}$ acting on $\mathcal{A}$ by a reflection on the hyperplane $H_{\alpha,n}$. 

In general, however, this action is not simple on the set of alcoves and the group $\Omega=\{w\in\widetilde{W}\ |\ w(\cO)=\cO\}$ is non-empty. These groups fit in a \textbf{splitting} short exact sequence
$$1\longrightarrow W_{\aff}\longrightarrow \widetilde{W}\longrightarrow \Omega\longrightarrow 1,$$
where $(W_{\aff},S_{\aff})$ is a Coxeter group generated by the simple reflections $s_0:=w_{\alpha_0,1}$, $s_i=w_{\alpha_i,0},\ i=1,\ldots, r$ along the walls of the fundamental alcove $\mathcal{C}_0$ and acting simply transitively on the set of alcoves of $\cA$. The group $W_{\aff}$ is the \textit{affine Weyl group} associated to the group $G$. The Weyl groups $W$, $\widetilde{W}$ and $W_{\aff}$ are independent of $T$, up to isomorphism.

\begin{example}
    \begin{enumerate}
        \item Let $G=\SL_2(F)$ and $T$ the set of diagonal matrices. Then $\Phi(G,T)=\{\pm\alpha\}$ where 
        \[
            \alpha\begin{psmallmatrix}
                t & 0\\
                0 & t^{-1}
            \end{psmallmatrix}=t^2 \quad\text{and}\quad \calpha(t)=\begin{psmallmatrix}
                t & 0\\
                0 & t^{-1}
            \end{psmallmatrix}\quad\text{for any }t\in F^\times,
        \]
        so $X^*(T)=\frac{\alpha}{2}\ZZ$ and $X_*(T)=\calpha\ZZ$. Moreover, we have that 
        $$N=T\cup\begin{psmallmatrix}
            0 & 1\\
            -1 & 0
        \end{psmallmatrix}T\quad\text{and}\quad w_\alpha(t)=\begin{psmallmatrix}
            0 & t\\
            -t^{-1} & 0
        \end{psmallmatrix},\ t \in F^\times.$$

        The apartment $\cA(\SL_2(F),T)$ is a one-dimensional real vector space whose hyperplanes $H_{\alpha,n}$ are the points $\frac{n}{2}\calpha$. It is easy to check that $\Omega=\{1\}$ so that $\widetilde{W}=W_{\aff}$ is generated by $s_0=\begin{psmallmatrix}
            0 & \varpi^{-1}\\
            -\varpi & 0
        \end{psmallmatrix}$ and $s_1=\begin{psmallmatrix}
            0 & 1\\
            -1 & 0
        \end{psmallmatrix}$.

        \item Let $G=\PGL_2(F)$ and $T$ the set of diagonal matrices. Then $\Phi(G,T)=\{\pm\alpha\}$ where 
        \[
            \alpha\begin{psmallmatrix}
                t & 0\\
                0 & 1
            \end{psmallmatrix}=t \quad\text{and}\quad \calpha(t)=\begin{psmallmatrix}
                t & 0\\
                0 & t^{-1}
            \end{psmallmatrix}=\begin{psmallmatrix}
                t^2 & 0\\
                0 & 1
            \end{psmallmatrix}\quad\text{for any }t\in F^\times,
        \]
        so $X^*(T)=\alpha\ZZ$ and $X_*(T)=\frac{\calpha}{2}\ZZ$. Similarly, 
        $$N=T\cup\begin{psmallmatrix}
            0 & 1\\
            1 & 0
        \end{psmallmatrix}T,\quad w_\alpha(t)=\begin{psmallmatrix}
            0 & t\\
            -t^{-1} & 0
        \end{psmallmatrix},\ t \in F^\times$$
        and the apartment $\cA(\SL_2(F),T)$ is a one-dimensional real vector space whose hyperplanes $H_{\alpha,n}$ are the points $\frac{n}{2}\calpha$. This time, however, $\Omega=\{1,\begin{psmallmatrix}
        0 & 1\\
        p & 0
        \end{psmallmatrix}\}$ is non-trivial, and 
        $$W_{\aff}=\left\langle s_0=\begin{psmallmatrix}
            0 & \varpi^{-1}\\
            \varpi & 0
        \end{psmallmatrix},s_1=\begin{psmallmatrix}
            0 & 1\\
            1 & 0
        \end{psmallmatrix}\right\rangle=\{w\in\widetilde{W}\ |\ \nu(\det(w))\text{ is even}\}$$ 
        is an index $2$ normal subgroup of $\widetilde{W}$.

        \item Let $G=\GL_2(F)$ and $T$ the set of diagonal matrices. Then $\Phi(G,T)=\{\pm\alpha\}$ where 
        \[
            \alpha\begin{psmallmatrix}
                t & 0\\
                0 & s
            \end{psmallmatrix}=ts^{-1} \quad\text{and}\quad \calpha(t)=\begin{psmallmatrix}
                t & 0\\
                0 & t^{-1}
            \end{psmallmatrix}\quad\text{for any }t\in F^\times.
        \]
        In this case, $\Omega\cong\ZZ$ is generated by $\begin{psmallmatrix}
            0 & 1\\
            \varpi & 0\\
        \end{psmallmatrix}$ and therefore $W_{\aff}=\langle s_0,s_1\rangle=\{w\in\widetilde{W}\ | \ \det(w)\in\cO^\times\}$ is a normal subgroup of $\widetilde{W}$ of infinite index. 
    \end{enumerate}
\end{example}

Some of the behaviour observed in the previous example holds in much greater generality. For example, $\Omega$ is an abelian group, and it has finite order if and only if $G$ is a simple group. In that case, $\Omega$ is in bijection with the centre of complex dual group $G^\vee(\CC)$ of $G$. In particular, if $G$ is simply connected, then $\Omega$ is trivial, while $\Omega$ has the largest size within the isogeny class when $G$ is adjoint. On the other hand, $W_{\aff}$ only depends on the isogeny class, and therefore only on the root system of $G$.


\subsection{The Bruhat-Tits building and parahoric subgroups}

It is possible to push idea further and construct the Bruhat-Tits building $\cB(G)$, a polysimplial space associated to $G$ that contains $\cA(G,T)$ for any $F$-split maximal torus. This is achieved by gluing together the apartments of all $F$-split maximal tori of $G$ and then gluing them according to some equivalence relation. %More concretely, we define $$\cB(G)=\cA(G,T)\times G/\sim$$ 
An important property of the building $\cB(G)$ is that it carries a $G$-action satisfying the following:
\begin{enumerate}
    \item It extends the action of $N_G(T)$ on $\cA(G,T)$ for each $F$-split maximal torus $T$.
    \item The stabilizer of $\cA(G,T)$ is $N_G(T)$ for each $F$-split maximal torus $T$.
    \item The stabilizer of any facet $c$ of the building is a (maybe disconnected) open compact subgroup of $G$.
    \item The action is strongly transitive on the set $\{(\mathcal{C},\cA)\ |\ \mathcal{C} \text{ is an alcove inside the apartment }\cA\}$. 
    \item For any pair $(\mathcal{C},\cA)$ as above, its stabilizer acts on $\cB(G)$ as the group $\Omega$. In other words
    $$\mathrm{Stab}_G(\mathcal{C},\cA)/T(\cO)=(N\cap\mathrm{Stab}_G(\mathcal{C}))/T(\cO)=\mathrm{Stab}_N(\mathcal{C})/T(\cO)\cong\Omega$$
     
\end{enumerate}

\begin{example}
    The Bruhat-Tits building of $G=\SL_2(\QQ_p)$ or $G=\PGL_2(\QQ_p)$ is an infinite tree all of whose vertex have degree $p+1$. Each infinite line inside the building is an apartment corresponding to a distinct $F$-split maximal torus of $G$. Consider the apartment $\cA(G,T)$, where $T$ is the group of diagonal matrices, and let $\Delta=\{\alpha\}$ be the simple root as above. Then $\mathcal{C}_0$ is the segment between the vertices $0$ and $\calpha/2$.

    If $G=\SL_2(\QQ_p)$, then 
    $$K_1:=\mathrm{Stab}(0)=\SL_2(\ZZ_p),\ K_2:=\mathrm{Stab}(\calpha/2)=\begin{psmallmatrix}
        \ZZ_p & p^{-1}\ZZ_p\\
        p\ZZ_p & \ZZ_p
    \end{psmallmatrix}, \text{ and }I:=\mathrm{Stab}(\mathcal{C}_0)=\begin{psmallmatrix}
        \ZZ_p^\times & \ZZ_p\\
        p\ZZ_p & \ZZ_p^\times
    \end{psmallmatrix}$$
    are all connected, open compact subgroups of $\SL_2(\QQ_p)$ not conjugate to each other. $K_0$ and $K_1$ are the unique maximal compact subgroups of $\SL_2(\QQ_p)$ -- in particular, the stabilizer of any vertex of the building is conjugate to either $K_0$ or $K_1$. The subgroup $I$ is called the \textbf{Iwahori subgroup}, it is conjugate to the stabilizer of any facet in the building and is of fundamental importance in the representation theory of $\SL_2(\QQ_p)$.

    On the other hand, if $G=\PGL_2(\QQ_p)$, then 
    $$K_1:=\mathrm{Stab}(0)=\PGL_2(\ZZ_p),\ K_2:=\mathrm{Stab}(\calpha/2)=\begin{psmallmatrix}
        \ZZ_p & p^{-1}\ZZ_p\\
        p\ZZ_p & \ZZ_p
    \end{psmallmatrix}_{\det\in\ZZ_p^\times}$$
    are both connected open compact subgroups and conjugate in $\PGL_2(F)$ by $\begin{psmallmatrix}
        0 & 1\\
        p & 0
    \end{psmallmatrix}$, so $\PGL_2(F)$ has one unique \textit{connected} maximal compact subgroup up to conjugacy. Correspondingly,     
    $$\mathrm{Stab}(\mathcal{C}_0)=\begin{psmallmatrix}
        \ZZ_p^\times & \ZZ_p\\
        p\ZZ_p & \ZZ_p^\times
    \end{psmallmatrix}_{\det\in\ZZ_p^\times}\bigsqcup\begin{psmallmatrix}
    0 & 1\\
    p & 0
    \end{psmallmatrix}\begin{psmallmatrix}
        \ZZ_p^\times & \ZZ_p\\
        p\ZZ_p & \ZZ_p^\times
    \end{psmallmatrix}_{\det\in\ZZ_p^\times}$$
    is a disconnected open compact subgroup, whose identity component is the Iwahori subgroup.
\end{example}

The example above suggests that the connected components of the stabilizers of facets in the building depend in a subtle way on the group $\Omega$. This is indeed the case, and we discuss this connection now. Since $\Omega=\mathrm{Stab}_G(\mathcal{C}_0)$ and $\mathcal{C}_0$ is bounded by hyperplanes corresponding to $S_{\aff}$, there is a natural homomorphism of groups $$\Omega\longrightarrow\Aut(S_{\aff}).$$ Moreover, all permutations of $S_{\aff}$ induced by $\Omega$ can be easily seen to preserve the affine Dynkin diagram associated to $S_{\aff}$, and if $G$ is simple of adjoint type, then all such automorphisms of $S_{\aff}$ are induced by $\Omega$. This greatly restricts the size of $\Omega$.

Next, fix some \textit{proper} subset $J\subset S_{\aff}$ and consider the \textit{standard} facet 
$$c_J=\{x\in\overline{\mathcal{C}_0}\ |\ \langle\alpha,x\rangle\in\ZZ\text{ for }\alpha\in J\text{ and }\langle\alpha,x\rangle\not\in\ZZ\text{ for }\alpha\in S_{\aff}-J\}.$$
Two facets $c_{J_1}$ and $c_{J_2}$ are conjugate under the action of $G$ if and only if $J_1$ and $J_2$ lie in the same $\Omega$-orbit. Moreover, any facet $c$ in the building is conjugate to $c_J$ for some proper subset $J\subset S_{\aff}$. In other words, there is a bijection between
\[
    \{\Omega-\text{orbits of }J\subsetneq S_{\aff}\}\longleftrightarrow\{G-\text{orbits of facets $c$ in the BT-building}\}
\]
For any facet $c$ of the building of $G$, we let $P_c^+:=\mathrm{Stab}_G(c)$ be the stabilizer of $c$. There is a short exact sequence
$$1\longrightarrow U_c\longrightarrow P_c^+\longrightarrow M_c^+,$$
where $U_c$ is the pro-unipotent radical of $P_c^+$ and $M_c^+$ is the group of $k$-rational points of a (possibly disconnected) reductive group $\mathbf{M}_c^+$ over $k$. 

\begin{definition}
    A \textbf{parahoric subgroup} $P_c$ is the inverse image in $P_c^+$ of the group $M_c$ of $k$-rational points of the identity component $\mathbf{M}_c$ of $\mathbf{M}_c^+$. We shall sometimes denote "parahoric subgroup" to the triple $(U_c,P_c,M_c)$. If $c$ is open in the building, then $(U_c,P_c,M_c)$ is a minimal parahoric subgroup and is called an \textbf{Iwahori subgroup}. The standard Iwahori subgroup corresponds to $J=\emptyset\subsetneq S_{\aff}$.
\end{definition}

Naturally, two parahoric subgroups are conjugate in $G$ if and only if the corresponding facets of the building are in the same $G$-orbit. Thus, all Iwahori subgroups are conjugate in $G$. If $c=c_J$ is a standard facet, then we simply write $(U_J,P_J,M_J)$ for its associated parahoric subgroup, and $P_J$ is generated by the standard Iwahori subgroup and $J$. Thus, there is a bijection between 
\[
    \{\Omega-\text{orbits of }J\subsetneq S_{\aff}\}\longleftrightarrow\{G-\text{conjugacy classes of parahoric subgroups } (U_c,P_c,M_c)\}
\]
Moreover, if the facet $c$ corresponds to $J\subsetneq S_{\aff}$, then
$$P_c^+/P_c\cong\Omega_J=\mathrm{Stab}_\Omega(J).$$
These results can be directly verified for $\SL_2$, $\PGL_2$ and $\GL_2$ using the examples above.

\begin{example}
    Suppose that $G=G_2(F)$. The affine Dynkin diagram of $G_2$ has no symmetries, so $\Omega=1$ and the extended affine weyl group $\widetilde{W}$ is a Coxeter group of type $\tilde{G_2}$. Since $S_{\aff}={s_0,s_1,s_2}$, there are $7$ conjugacy classes of parahoric subgroups, satisfying
    \begin{align}
        M_{\{s_1,s_2\}}=G_2(k),\quad M_{\{s_0,s_1\}}=\SL_3(k), \quad M_{\{s_0,s_2\}}=\SL_2(k)\times\SL_2(k)\\
        \text{\textcolor{red}{what about singletons}}\quad M_{\emptyset}=T(k)=(k^\times)^2.
    \end{align}
\end{example}

\subsection{Parahoric restriction}

Parahoric subgroups are ubiquitous objects in the representation theory of p-adic objects, since it provides a bridge between smooth admissible representations of the p-adic group $G$ and finite dimensional representations of the finite groups of Lie type $M_c$ defined in the previous section. In this section, we explore this important connection that we will exploit in a latter chapter. 

If $(U,P,M)$ is any parahoric subgroup corresponding to a facet $c$ and $(\pi,V)$ is a smooth admissible representation of $G$, the space $V^{U}$ of fixed points under the pro-unipotent radical is naturally a representation of $M$. We can take this idea one step further and define the \textit{parahoric restriction functor}
\begin{equation*}
    \res_c:R(G)\longrightarrow \CC[M]^{M},\quad V\longmapsto \text{(character of) }V^{U},\quad\text{for all }V\in\Irr(G),
\end{equation*}
where $\CC[M]^M$ is the space of class functions of $M$. This is well-defined since the representations are assumed to be admissible. The existence of such a functor is very powerful -- for example, one can then apply the theory of Deligne-Lusztig induction to the setting of representation of $p$-adic groups. 

\begin{definition}
    We say that an irreducible representation $(\pi,V)$ of $G$ is \textit{unipotent} if there is a parahoric subgroup $(P,U,M)$ such that $V^U$ contains a cuspidal unipotent representation of $M$. More generally, a smooth admissible representation of $G$ is unipotent if all irreducible subquotients of $\pi$ are unipotent.
\end{definition}

\textcolor{red}{Are all pairs $(P,E)$ as above types of certain Bernstein blocks?}

\begin{example}
    
\end{example}


\begin{definition}
    Let $(U_c,P_c,M_c)$ be a parahoric subgroup and let $(\tau, E)$ be a cuspidal representation of $M_c$. We then define 
    $$\Irr(G,P_c,E)=\{(\pi,V)\in\Irr(G)\ |\ \text{the $M_c$-module }V^{U_c}\text{ contains the $M_c$-module }E\}$$ 
\end{definition}


\begin{definition}
    An irreducible smooth admissible representation $(\pi,V)$ of $G$ is \textit{unipotent} if there exists some parahoric subgroup $K$ of $G$ such that $V^{U_K}$ contains a cuspidal unipotent representation of the finite group $\overline{K}$.
\end{definition}
We note that if we replace $G$ for $G^F$ and \textit{parahoric} by \textit{parabolic}, then we recover the definition of a unipotent representation in $G^F$. 

The following two results are basic to ensure that unipotent representations behave under the restriction functor above.
\begin{proposition}\label{prop:unipotent_padic}
    Let $(\pi,V)$ be an irreducible admissible representation of $G$. If there is some parahoric subgroup $K$ such that $V^{U_K}$ contains a non-cuspidal unipotent representation of $\overline{K}=K/U_K$, then $(\pi,V)$ is a unipotent representation of $G$.
    %\begin{enumerate}
    %    \item Conversely, if $(\pi,V)$ is a unipotent representation of $G$, then all irreducible $\overline{K}$-subrepresentations of $V^{U_K}$ are unipotent.
    %\end{enumerate}
\end{proposition}
\begin{proof}
    To prove the first claim, suppose that $V^{U_K}$ contains a unipotent irreducible representation $(\sigma,W)$ of $\overline{K}$, and therefore $\Hom_K(W,V^{U_K})\neq 0$. By Proposition \ref{prop:harishchandra}, there is some standard parabolic subgroup $\overline{P}=\overline{U_P}\cdot\overline{L_P}$ of $\overline{K}$ and cuspidal unipotent representation $(\tau,U)$ of $\overline{L_P}$ such that 
    $$\Hom_{K}(W,\Ind_{\overline{P}}^{\overline{K}}U)\neq 0,$$
    where we view the $\overline{K}$ representations as inflated $K$ representations, trivial on $U_K$. By the classification of parahoric subgroups in $G$, it follows that $\bar{P}=H/U_K$, where $H$ is another parahoric subgroup contained in $K$. Moreover, we have the inclusions $U_K\subseteq U_H\subseteq H\subseteq K$ and therefore $\overline{U_P}=U_H/U_K$ and $\overline{L_P}=H/U_H$. Since induction and inflation are commuting operations, it follows that the induced $\overline{K}$-representation $\Ind_{\overline{P}}^{\overline{K}}U$ is isomorphic to the $K$-representation $\Ind_H^K U$, where we view $U$ as the induced $H$-representation trivial on $U_H$, and hence $\Hom_K(W,\Ind_H^K U)\neq0$. Since $W$ is irreducible and $K$ is compact, it also follows that 
    $$\Hom_K(\Ind_H^K U,V^{U_K})=\Hom_H(U,V^{U_K})\neq 0.$$
    Since the representation $U$ is trivial on $U_H$, it follows that the image of $H$-equivariant map $U\to V^{U_K}$ lies inside $V^{U_H}$. Thus, 
    $$\Hom_H(U,V^{U_H})=\Hom_H(U,V^{U_K})\neq 0,$$
    and this concludes the proof.
\end{proof}

Conversely, we would like to show that for any irreducible unipotent representation $(\pi,V)$ of $G$, the irreducible $\overline{P}$-submodules of $V^{P^+}$ are all unipotent. %We first note that by the results of Harish-Chandra, for any such irreducible $\overline{P}$-submodules affording the character $\chi$, there is a parahoric subgroup $Q\subseteq P$ and cuspidal representation 
This is a direct corollary of the following theorem.

\begin{theorem}
    Let $(\pi,V_\pi)$ be an irreducible admissible representation of $G$. Suppose that there is some parahoric $P$ such that $V^{P^+}\neq 0$ contains an irreducible cuspidal representation $(\sigma,V_\sigma)$ of $\overline{P}$. Then, for any parahoric $Q$ such that $V^{Q^+}\neq 0$, any $\overline{Q}$-submodule of $V^{Q^+}$ contains a cuspidal representation $(\tau,V_\tau)$ of $\overline{L}$ for some parahoric subgroup $L\subseteq Q$. Moreover, there is some $g\in G$ such that $L=\prescript{g}{}{P}$ and $\tau=\prescript{g}{}{\sigma}$.
\end{theorem}

\begin{proof}
    From the proof of the previous result implies that any $\overline{Q}$-submodule of $V^{Q^+}$ contains a cuspidal representation $\tau$ of $\overline{L}$ for some parahoric subgroup $L\subseteq Q$. Thus, we may assume without loss of generality that $Q=L$ so that the irreducible $\overline{Q}$-submodule $(\tau,V_\tau)$ of $V^{Q^+}$ is cuspidal. To prove the second statement, we let $E_\tau:V_\pi\rightarrow V_\tau$ be a $Q$-equivariant projection. Then, for any $g\in G$, we have the linear map 
    $$\varphi_g=E_\tau\circ\pi(g^{-1}):V_\sigma\longrightarrow V_\tau,$$
    and for any $h\in P\cap gQg^{-1}$ and $v\in V_\sigma$, we have that 
    $$\varphi_g\circ\sigma(h)(v)=E_\tau(\pi(g^{-1}hg)\pi(g^{-1})v)=\tau(g^{-1}hg)\circ   E_\tau(\pi(g^{-1}))(v)=\tau(g^{-1}hg)\circ\varphi_g(v).$$
Since $\pi$ is an irreducible representation of $G$, there must be some $g\in G$ such that $\pi(g^{-1})V_\sigma\not\subseteq\ker(E_\tau)$, in which case $\varphi_g\neq 0$. The image of $P\cap gQg^{-1}$ inside $\overline{P}$ (resp $\overline{gQg^{-1}}$) is a parabolic subgroup $\overline{P}_{\prescript{g}{}{Q}}$ (resp. $\overline{\prescript{g}{}{Q}}_{P}$) 
\end{proof}

\textcolor{red}{Need to find proofs for this!!}

\begin{example}
    The Iwahori subgroup $I=\langle T(\cO),\cX_\alpha(\cO),\cX_\alpha(\varpi\cO)\ |\ \alpha\in\Phi^+\rangle$, with pro-unipotent radical $U_I=\langle T(1+\varpi\cO),\cX_\alpha(\cO),\cX_\alpha(\varpi\cO)\ |\ \alpha\in\Phi^+\rangle$ and reductive quotient $I/U_I=T(k_F)$ is, up to conjugacy, the unique minimal parahoric subgroup of $G$. Any smooth admissible representation $(\pi,V)$ of $G$ with an Iwahori fixed vector is unipotent. Indeed, $V^{U_I}$ contains the trivial representation of $T(k_F)$, its unique unipotent representation. 
\end{example}

Analogously to the construction of $R(G)$, we define $\Irr_{\textrm{un}}(G)$ to be the set of irreducible unipotent representations of $G$, and $R_{\textrm{un}}(G)$ to be its $\CC$-span. Lemma \ref{prop:unipotent_padic} implies that there is a well-defined \textit{restriction function}
\begin{equation*}
    \res_{\textrm{un}}^K:R_{\textrm{un}}(G)\longrightarrow \CC_{\textrm{un}}[\overline{K}]^{\overline{K}},\quad V\longmapsto \text{(character of) }V^{U_K},\quad\text{for all }V\in\Irr(G).
\end{equation*}
It is also convenient to consider simultaneously all such functions for all conjugacy classes of maximal parahoric subgroups, so we define $\res_{\textrm{un}}^{\textrm{par}}=(\res_{\textrm{un}}^K)_K$. In the previous section, we defined a nonabelian Fourier transform 
$$\FT^K:\CC_{\un}[\overline{K}]^{\overline{K}}\longrightarrow\CC_{\un}[\overline{K}]^{\overline{K}},$$
taking the unipotent characters to the almost characters of $\overline{K}$, having nice geometric properties. We also consider all these maps simultaneously for all conjugacy classes of maximal parahoric subgroups, which we denote as $\FT^{\mathrm{par}}=(\FT^K)_{K \text{ maximal}}$.

It is therefore a natural question to ask whether there exists some function $\FT^\vee:R_{\un}(G)\rightarrow R_{\un}(G)$ such that the square
\[
\begin{tikzcd}
R_{\un}(G) \arrow[r, "\FT^\vee"] \arrow[d, "\res_{\un}^{\mathrm{par}}"] & R_{\un}(G) \arrow[d, "\res_{\un}^{\mathrm{par}}"] \\
\bigoplus_K\CC_{\un}[\overline{K}]^{\overline{K}} \arrow[r, "\FT^{\mathrm{par}}"] & \bigoplus_K\CC_{\un}[\overline{K}]^{\overline{K}}
\end{tikzcd}
\]

\textcolor{red}{This question is now unresolved, but partial progress has been achieved.} To understand it, we first need to look at the Langlands parametrization of unipotent representations. 

\subsection{The Langlands parametrization of unipotent representations}
For this section, we introduce the Weil group $W_F$ of the field $F$ with inertia subgroup $I_F$. Moreover, we set $W_F':=W_F\times\SL_2(\CC)$. %In addition to these groups, Langlands introduced the $L$-group $\prescript{L}{}{G}:=G^\vee\rtimes W_F$ associated to the $p$-adic group $G$, where $G^\vee$ denotes the $\CC$-points of the dual group of $\mathbf{G}$
Under the assumption that $\mathbf{G}$ is a split group, we have the following important definition.
\begin{definition}
    A \textit{Langlands parameter} (or $L$-parameter) for $G$ is a continuous morphism $\varphi:W'_F\rightarrow G^\vee$, where $G^\vee$ denotes the $\CC$-points of the dual group of $\mathbf{G}$, and $\varphi((w,1))$ is semisimple for each $w\in W_F$.
\end{definition}


In its simplest form, the Local Langlands correspondence (LLC) conjectures the existence of a finite to one map between isomorphism classes of smooth admissible complex representations of $G$ and conjugacy classes of Langlands parameters of $G$ satisfying certain nice properties. In particular, the LLC conjectures that the unipotent representations of $G$ correspond to the following Langlands parameters.

\begin{definition}
    An $L$-parameter $\varphi:W_F\times\SL_2(\CC)\rightarrow G^\vee$ is called \textit{unipotent} if $\varphi(w,1)=1$ for any element $w$ of the inertia subgroup $I_F$ of $W_F$. Such parameters are sometimes called \textit{unramified} Langlands parameters.
\end{definition}
\begin{remark}
    For any $L$-parameter $\varphi:W_F'\rightarrow G^\vee$, define the elements $u_\varphi=\varphi(1,\left(\begin{smallmatrix}
        1 & 1\\
        0 & 1
    \end{smallmatrix}\right))$ and $s_\varphi=\varphi(\Frob,\Id)$ that commute. An application of the Jacobson--Morozov theorem implies that an $L$-parameter is determined by $u_\varphi$ and $\varphi|_{W_F}$ up to $G^\vee$-conjugacy. If the $L$-parameter is, in addition, unipotent, then $\varphi|_{W_F}$ is determined by $s_\varphi$. Thus, unipotent $L$-parameters are parametrized by $G^\vee$ conjugacy classes of pairs $(u,s)$ where $u\in G^\vee$ is unipotent, $s\in G^\vee$ is semisimple and they commute. But this is the same as conjugacy classes of elements of $G^\vee$ (by using the Jordan decomposition).
\end{remark}

However, this is a finite-to-one correspondence, so it does not parametrize unipotent representations, but rather $L$-packets of unipotent representations. To get a one to one correspondence, we need to introduce refinements of the $L$-parameters. Given an $L$-parameter $\varphi$, a natural object of interest is the centralizer $Z_{G^\vee}(\varphi)$ of the image of $\varphi$ inside $G^\vee$. Now, $G^\vee$ sits within $G^\vee_{\mathrm{sc}}=(G_{\ad})^\vee\twoheadrightarrow G^\vee\twoheadrightarrow G_{\ad}^\vee$. We then denote $Z_{G^\vee}^1(\varphi)$ to be the inverse image of $Z_{G^\vee}(\varphi)$ under the isogeny $G^\vee_{\mathrm{sc}}\twoheadrightarrow G^\vee$. We then denote by $A_\varphi$, $A^1_\varphi$ to be the component group of $Z_{G^\vee}(\varphi)$ and $Z_{G^\vee}^1(\varphi)$, respectively.

\begin{definition}
    An enhanced Langlands parameter is a pair $(\varphi,\phi)$, where $\varphi:W_F'\rightarrow G^\vee$ is an $L$-parameter and $\phi$ is an irreducible representation of $A_\varphi^1$. 
\end{definition}

Let us introduce some notation. We denote by $\Phi_{\un}(G^\vee)$ the set of unipotent $L$-parameters and $\Phi_{e,\un}(G^\vee)$ the set of all \textit{enhanced} unipotent $L$-parameters. Similarly to $L$-parameters, an enhanced $L$-parameter $(\varphi,\phi)$ is determined by the triple $(u_\varphi,\varphi|_{W_F},\phi)$, and if it is also unipotent, then it is determined by the triple $(u_\varphi,s_\varphi,\phi)$. Moreover, if $x_\varphi=s_\varphi u_\varphi$, then $Z_{G^\vee}(\varphi)=Z_{G^\vee}(x_\varphi)$, and therefore there is a canonical bijection
\begin{align*}
    \Phi_{\mathrm{e,un}}(G^\vee)&\longleftrightarrow G^\vee\backslash\{(x,\phi)\ |\ x\in G^\vee, \phi\in\widehat{A_x^1}\},\\
    (\varphi,\phi)&\longmapsto (s_\varphi u_\varphi,\phi)
\end{align*}
where $A_x^1$ is the component group of $Z^1_{G^\vee}(x)$.

Then the LLC predicts that there is a natural bijection
\begin{align*}
    \LLC_{\un}: G^\vee\backslash\{(x,\phi)\ |\ x\in G^\vee, \phi\in\widehat{A_x^1}\}\longleftrightarrow\Phi_{\mathrm{e,un}}(G^\vee)\longleftrightarrow&\bigsqcup_{G'\in\InnT(G)}\Irr_{\un}(G')\\
    (x,\phi)\hspace{1.5cm}\longmapsto \hspace{1.5cm}&\quad\pi(x,\phi),
\end{align*}
where $G'$ runs over the classes of inner twists of $G$. We say that a pair $(x,\phi)$ is $G'$-relevant if $\pi(x,\phi)\in\Irr_{\un}(G')$, and this property can be checked explicitly. Firstly, we note that there is a canonical bijection 
$$\InnT(G)\longleftrightarrow H^1(F,\mathrm{Inn}(\mathbf{G}^*))\longleftrightarrow\Irr(Z_{G^\vee_{sc}}),\quad G'\longmapsto\zeta_{G'}.$$
Similarly, for any pair $(x,\phi)$, the representation $\phi$ induces a character $\zeta_\phi$ on $Z_{G^\vee_{sc}}$. Then $\pi(x,\phi)$ is $G'$ relevant if and only if $\zeta_\phi=\zeta_{G'}$ and we denote the set of $G'$-relevant parameters by $\Phi_{e,\un}(G')$. 

It is then clear that 
$$\Phi_{e,\un}(G^\vee)=\bigsqcup_{G'\in\InnT(G)}\Phi_{e,\un}(G'),$$
and the LLC predicts that $\Phi_{e,\un}(G')$ parametrizes the set $\Irr_{\un}(G')$ for each $G'\in\InnT(G)$.

\subsubsection{Pure Langlands parameters}

One can also refine $L$-parameters by considering instead pairs $(\varphi,\phi)$, where $\varphi$ is an $L$-parameters and $\phi$ is an irreducible representations of $A_\varphi$. We then set 
$$\Phi_{\mathrm{e,un}}^p(G^\vee)=G^\vee\backslash\{(\varphi,\phi)\ |\ \varphi\text{ unipotent, }\phi\in\widehat{A_\varphi}\},$$
which is in natural bijection with the set 
$$G^\vee\backslash\{(x,\phi)\ |\ x\in G^\vee, \phi\in\widehat{A_x}\},$$
where $A_x$ is the component group of $Z_{G^\vee}(x)$.

In this setting the Local Langlands conjecture predicts a natural bijection
\begin{equation*}
    G^\vee\backslash\{(x,\phi)\ |\ x\in G^\vee,\phi\in\widehat{A_x}\}\longleftrightarrow\bigsqcup_{G'\in\InnT^p(G)}\Irr_{\un}(G'),\quad (x,\phi)\longmapsto\pi(x,\phi)
\end{equation*}
where $G'$ now runs over the classes of \textit{pure} inner twists of $G$. We distinguish between the distinct pure inner twists by looking at characters of $Z_{G^\vee}$.

\begin{example}
    If $\mathbf{G}$ is a simple split \textit{simply connected} algebraic group, then $H^1(F,\mathbf{G}^*)=1$ and therefore there is only one class of pure inner forms of $G$, namely $G$ itself. Correspondingly, $G^\vee=G^\vee_{\ad}$ and $Z_{G^\vee}$ is trivial. Therefore, the above discussion gives a bijection 
    \begin{equation*}
        \LLC_{\un}^p:G^\vee\backslash\{(x,\phi)\ |\ x\in G^\vee,\phi\in\widehat{A_x}\}\longleftrightarrow\Phi_{\mathrm{e,un}}^p(G^\vee)\longleftrightarrow\Irr_{\un}(G^*).
    \end{equation*}
\end{example}

\begin{example}
    If $\mathbf{G}$ is a simple split \textit{adjoint} algebraic group, then $H^1(F,\mathbf{G}^*)=H^1(F,\mathrm{Inn}(\mathbf{G}^*))$ so for each inner twist there is one unique pure inner twist. 
    Therefore, from the previous discussion, unipotent enhanced $L$-parameters are in bijection with the set
    $$G^\vee\backslash\{(x,\phi)\ |\ x\in G^\vee,\phi\in\widehat{A_x}\}, \quad\text{where}\quad A_x=Z_{G^\vee}(x)/Z_{G^\vee}(x)^0,$$
    and we have a one-to-one correspondence 
    \begin{equation*}
        G^\vee\backslash\{(x,\phi)\ |\ x\in G^\vee,\phi\in\widehat{A_x}\}\longleftrightarrow\bigsqcup_{G'\in\InnT(G)}\Irr_{\un}(G'),\quad (x,\phi)\longmapsto\pi(x,\phi)
    \end{equation*}
\end{example}