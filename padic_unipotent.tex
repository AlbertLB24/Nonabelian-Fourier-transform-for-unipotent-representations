\section{Unipotent representations of p-adic groups}
Having discussed the classification of unipotent characters of finite groups of Lie type and Lusztig's nonabelian Fourier transform, in this section we explain in detail the remaining maps and objects from Conjecture \ref{conj:main}. To achieve this, we first need to cover the notion of a parahoric subgroup and of a unipotent representation of a $p$-adic group
%denoted the \textit{dual nonabelian Fourier transform} on unipotent representations of $p$-adic groups. 
Therefore, we now shift our focus completely and turn our attention towards the structural and representation theoretic aspects of $p$-adic groups. 

Throughout, we let $F$ be a nonarchimedean local field with ring of integers $\cO$, uniformizer $\varpi$ and residue field $k$ of cardinality $q$, a power of a prime $p$. Let $\mathbf{G}$ be a connected, almost simple, split algebraic group over $F$ and let $G=\mathbf{G}(F)$ be the group of $F$-rational points. In this section, we will first discuss some results about the structure theory of the group $G$, including the Bruhat--Tits building and the classification of parahoric (or open compact) subgroups of $G$. These results will be instrumental to study and classify the unipotent representations of $G$ \textcolor{red}{refer to definition} inside the category $\Rep(G)$ of smooth admissible complex representations of $G$, the central object of interest in the local Langlands correspondence. 

\iffalse
\subsection{The Bernstein decomposition}\label{sec:Bernstein}

Let $F$ be a nonarchimedean local field with ring of integers $\cO$, uniformizer $\varpi$ and residue field $k$ of cardinality $q$, a power of a prime $p$. Let $\mathbf{G}$ be a connected, almost simple, split algebraic group over $F$ and let $G=\mathbf{G}(F)$. We denote by $\Rep(G)$ the set of smooth admissible complex representations of $G$. We begin this chapter by discussing a fundamental result that instrumental in the study of the category $\Rep(G)$. The starting point is the following well-known fact.

\begin{proposition}
    Let $(\pi,V)$ be an irreducible smooth representation of $G$. Then there exists a parabolic subgroup $P\subseteq G$ with Levi subgroup $M$ and a supercuspidal representation $(\sigma,M)$ of $M$ such that $\pi\hookrightarrow\Ind_P^G\sigma$. Moreover, if $P'$ is another parabolic subgroup with Levi subgroup $M$ and supercuspidal representation $(\sigma',W')$ such that $\pi\hookrightarrow\Ind_{P'}^{G'}\sigma'$, then there exists $g\in G$ such that $M'=gMg^{-1}$ and $\sigma'\cong\prescript{g}{}{\sigma}$.
\end{proposition}
Given an irreducible smooth representation $(\pi,V)$, we denote the $G$-conjugacy class of $(M,(\sigma,W))$ as above the \textit{supercuspidal support} of $(\pi,V)$.

The Bernstein decomposition naturally arises when we study whether two irreducible representations with distinct supercuspidal support can have non-trivial extensions between them. This is indeed possible, but only to a very limited extent.

\begin{lemma}
    Let $(\pi,V), (\pi',V')$ be two irreducible representations with supercuspidal support $(M,\sigma)$, $(M',\sigma')$, respectively. If there is a non-trivial extension between $V$ and $V'$, then there exists $g\in G$ and an unramified character $\chi$ of $M'(F)$ such that $M'=gMg^{-1}$ and $\sigma'\cong\prescript{g}{}{\sigma}\otimes\chi$.
\end{lemma}
If the conclusion of the lemma is satisfied, then we say that the pairs $(M,\sigma)$ and $(M',\sigma')$ are \textit{inertially equivalent}, and we denote the equivalence by $\sim$ and the inertial equivalence class by $[M,\sigma]_G$. Finally, we let $\mathfrak{J}(G)$ be the set of inertial equivalence classes. 

If $[M,\sigma]\in\mathfrak{J}(G)$, then we denote $\Rep(G)_{[M,\sigma]}$ the full subcategory of $\Rep(G)$ whose objects are representations $(\pi,V)$ satisfying that all for any irreducible subquotient $\pi'$ of $\pi$, there is a parabolic subgroup $P'$ with Levi subgroup $M'$ and supercuspidal representation $\sigma'$ of $M'$ such that $\pi'\hookrightarrow\Ind_{P'}^{G'}$ and $(M',\sigma')\in[M,\sigma]_G$.

These results are summarized in the following theorem.

\begin{theorem}[Bernstein decomposition]
    We have an equivalence of categories
    \begin{equation}
        \Rep(G)\cong \prod_{[M,\sigma]\in\mathfrak{J}(G)}\Rep(G)_{[M,\sigma]}
    \end{equation}
    and each full subcategory $\Rep(G)_{[M,\sigma]}$ is indecomposable.
\end{theorem}
\fi

\subsection{The apartment of a split maximal torus}
To simplify the exposition, we will assume throughout that $G$ is a split $p$-adic group over $F$, even though it is not strictly necessary.
%Before continuing with representation theoretic aspects of $p$-adic groups, we first shift towards a more structural focus. As in the previous section, let $\mathbf{G}$ be a connected, almost simple, split algebraic group over $F$ and let $G=\mathbf{G}(F)$.
For any split maximal torus $T$ of $G$ over $F$, there is a natural perfect pairing 
$$\langle\cdot,\cdot\rangle:X^*(T)\times X_*(T)\longrightarrow\ZZ,$$
where $X^*(T)=\Hom(T,F^\times)$ and $X_*(T)=\Hom(F^\times,T)$ are its character and cocharacter lattice, respectively, and the pairing is obtained by composition. Let $\Phi(G,T)\subset X^*(T)$ be the set of roots associated to $T$, with the corresponding set of coroots $\Phi^\vee(G,T)\subset X_*(T)$. Similarly to the previous section, a choice of a Borel subgroup $B$ of $G$ containing $T$ is equivalent to the choice of simple roots $\Delta=\{\alpha_1,\ldots,\alpha_r\}\subset\Phi(G,T)$, which we now fix throughout. We also let $\alpha_0$ be the highest root of $\Phi(G,T)$ with respect to $\Delta$. In addition, the group $B$ together with the normalizer $N:=N_G(T)$ form a $BN$-pair with corresponding Weyl group $W=N(F)/T(F)$. 

A natural object arising in the representation theory of $G$ is the apartment $\cA(G,T):=X_*(T)\otimes_\ZZ\RR$, a real vector space spanned by the simple coroots. Moreover, $\cA(G,T)$ has the structure of a simplicial complex given by the hyperplanes
$$H_{\alpha,n}=\{x\in\cA(G,T)\ |\ \langle\alpha,x\rangle=n\},\quad\text{for each }\alpha\in\Phi(G,T)^+\text{ and } n\in\ZZ.$$
Whenever the torus $T$ is clear from context, we will omit it from the notation. The complexes on the apartment are called \textit{facets}, and the facets of largest dimension (equivalently, the open facets) are called \textit{alcoves}. Our choice of simple roots $\Delta$ determines a canonical alcove
$$\mathcal{C}_0=\{x\in\cA\ |\ \langle\alpha,x\rangle>0, \alpha\in\Delta\ \text{ and } \langle\alpha_0,x\rangle<1\},$$
commonly referred to as the \textit{fundamental alcove}.

Another important property of the apartment is that it carries a natural action of the group $N$ by affine transformations. To state some of its properties, we define, for each root $\alpha\in\Phi(G,T)$ a map (not a group homomorphism!) $w_\alpha:\CC^\times\rightarrow N$ given by $w_\alpha(t)=x_\alpha(t)x_{-\alpha}(-t^{-1})x_\alpha(t)$. The action of $N$ on $\cA$ satisfies
\begin{itemize}
    \item For any $\alpha\in\Phi$ and $\lambda\in F$, the element $\calpha(\lambda)\in T\subset N$ acts on $\cA$ by a translation $-\nu_p(\lambda)\calpha$.
    \item For any $\alpha\in\Phi$, the element $w_\alpha(t)\in N$ acts on $\cA$ by a reflection along $H_{\alpha,-\nu_p(t)}$. When $t\in\cO$, this coincides with the natural action of $W$ on $\cA$.
    \item This action preserves the simplicial structure of the apartment and is transitive on the set of alcoves of $\cA$.
    \item The kernel of this action is $T(\cO)$.
\end{itemize}
Therefore the \textit{extended Weyl group}
$$\widetilde{W}:=N(F)/T(\cO)\cong W\ltimes X_*(T)$$
acts faithfully on the apartment $\cA$ and transitively on the set of alcoves. We denote by $w_{\alpha,n}$ the unique element in $\widetilde{W}$ acting on $\mathcal{A}$ by a reflection on the hyperplane $H_{\alpha,n}$. 

In general, however, this action is not simple on the set of alcoves and the group $\Omega=\{w\in\widetilde{W}\ |\ w(\mathcal{C}_0)=\mathcal{C}_0\}$ is non-trivial. These groups fit in a \textbf{splitting} short exact sequence
$$1\longrightarrow W_{\aff}\longrightarrow \widetilde{W}\longrightarrow \Omega\longrightarrow 1,$$
where $(W_{\aff},S_{\aff})$ is an affine Coxeter group generated by the simple reflections $\tilde{s}_0:=w_{\alpha_0,1}$, $s_i=w_{\alpha_i,0},\ i=1,\ldots, r$ along the walls of the fundamental alcove $\mathcal{C}_0$ and acting simply transitively on the set of alcoves of $\cA$. The group $W_{\aff}$ is the \textit{affine Weyl group} associated to the group $G$. The Weyl groups $W$, $\widetilde{W}$ and $W_{\aff}$ are independent of $T$, up to isomorphism.

\begin{example}
    \begin{enumerate}
        \item Let $G=\SL_2(F)$ and $T$ the set of diagonal matrices. Then $\Phi(G,T)=\{\pm\alpha\}$ where 
        \[
            \alpha\begin{psmallmatrix}
                t & 0\\
                0 & t^{-1}
            \end{psmallmatrix}=t^2 \quad\text{and}\quad \calpha(t)=\begin{psmallmatrix}
                t & 0\\
                0 & t^{-1}
            \end{psmallmatrix}\quad\text{for any }t\in F^\times,
        \]
        so $X^*(T)=\frac{\alpha}{2}\ZZ$ and $X_*(T)=\calpha\ZZ$. Moreover, we have that 
        $$N=T\cup\begin{psmallmatrix}
            0 & 1\\
            -1 & 0
        \end{psmallmatrix}T\quad\text{and}\quad w_\alpha(t)=\begin{psmallmatrix}
            0 & t\\
            -t^{-1} & 0
        \end{psmallmatrix},\ t \in F^\times.$$

        The apartment $\cA(\SL_2(F),T)$ is a one-dimensional real vector space whose hyperplanes $H_{\alpha,n}$ are the points $\frac{n}{2}\calpha$. It is easy to check that $\Omega=\{1\}$ so that $\widetilde{W}=W_{\aff}$ is generated by $s_0=\begin{psmallmatrix}
            0 & \varpi^{-1}\\
            -\varpi & 0
        \end{psmallmatrix}$ and $s_1=\begin{psmallmatrix}
            0 & 1\\
            -1 & 0
        \end{psmallmatrix}$.

        \item Let $G=\PGL_2(F)$ and $T$ the set of diagonal matrices. Then $\Phi(G,T)=\{\pm\alpha\}$ where 
        \[
            \alpha\begin{psmallmatrix}
                t & 0\\
                0 & 1
            \end{psmallmatrix}=t \quad\text{and}\quad \calpha(t)=\begin{psmallmatrix}
                t & 0\\
                0 & t^{-1}
            \end{psmallmatrix}=\begin{psmallmatrix}
                t^2 & 0\\
                0 & 1
            \end{psmallmatrix}\quad\text{for any }t\in F^\times,
        \]
        so $X^*(T)=\alpha\ZZ$ and $X_*(T)=\frac{\calpha}{2}\ZZ$. Similarly, 
        $$N=T\cup\begin{psmallmatrix}
            0 & 1\\
            1 & 0
        \end{psmallmatrix}T,\quad w_\alpha(t)=\begin{psmallmatrix}
            0 & t\\
            -t^{-1} & 0
        \end{psmallmatrix},\ t \in F^\times$$
        and the apartment $\cA(\SL_2(F),T)$ is a one-dimensional real vector space whose hyperplanes $H_{\alpha,n}$ are the points $\frac{n}{2}\calpha$. This time, however, $\Omega=\{1,\begin{psmallmatrix}
        0 & 1\\
        p & 0
        \end{psmallmatrix}\}$ is non-trivial, and 
        $$W_{\aff}=\left\langle s_0=\begin{psmallmatrix}
            0 & \varpi^{-1}\\
            \varpi & 0
        \end{psmallmatrix},s_1=\begin{psmallmatrix}
            0 & 1\\
            1 & 0
        \end{psmallmatrix}\right\rangle=\{w\in\widetilde{W}\ |\ \nu(\det(w))\text{ is even}\}$$ 
        is an index $2$ normal subgroup of $\widetilde{W}$.

        \item Let $G=\GL_2(F)$ and $T$ the set of diagonal matrices. Then $\Phi(G,T)=\{\pm\alpha\}$ where 
        \[
            \alpha\begin{psmallmatrix}
                t & 0\\
                0 & s
            \end{psmallmatrix}=ts^{-1} \quad\text{and}\quad \calpha(t)=\begin{psmallmatrix}
                t & 0\\
                0 & t^{-1}
            \end{psmallmatrix}\quad\text{for any }t\in F^\times.
        \]
        In this case, $\Omega\cong\ZZ$ is generated by $\begin{psmallmatrix}
            0 & 1\\
            \varpi & 0\\
        \end{psmallmatrix}$ and therefore $W_{\aff}=\langle s_0,s_1\rangle=\{w\in\widetilde{W}\ | \ \det(w)\in\cO^\times\}$ is a normal subgroup of $\widetilde{W}$ of infinite index. 
    \end{enumerate}
\end{example}

Some of the behaviour observed in the previous example holds in much greater generality. For example, $\Omega$ is an abelian group, and it has finite order if and only if $G$ is a simple group. In that case, $\Omega$ is in bijection with the centre of complex dual group $G^\vee(\CC)$ of $G$ and with the length preserving automorphisms of the affine Dynkin diagram \textcolor{red}{(cite Iwahori)}. In particular, if $G$ is simply connected, then $\Omega$ is trivial, while $\Omega$ has the largest size within the isogeny class when $G$ is adjoint. On the other hand, $W_{\aff}$ only depends on the isogeny class, and therefore only on the root system of $G$.

\subsection{The Bruhat-Tits building and parahoric subgroups}

When we work with the apartment $\cA(G,T)$, we are making an arbitrary choice for the torus. It is possible, however, to push this idea further and construct the Bruhat-Tits building $\cB(G)$, a \textit{canonical} polysimplial space associated to $G$ that contains $\cA(G,T)$ for any $F$-split maximal torus. This is achieved by gluing together the apartments of all $F$-split maximal tori of $G$ according to some natural equivalence relation. An important property of the building $\cB(G)$ is that it carries a $G$-action satisfying the following:
\begin{enumerate}
    \item It extends the action of $N_G(T)$ on $\cA(G,T)$ for each $F$-split maximal torus $T$.
    \item The stabilizer of $\cA(G,T)$ is $N_G(T)$ for each $F$-split maximal torus $T$.
    \item The stabilizer of any facet $c$ of the building is a (maybe disconnected) open compact subgroup of $G$.
    \item The action is strongly transitive on the set $\{(\mathcal{C},\cA)\ |\ \mathcal{C} \text{ is an alcove inside the apartment }\cA\}$. 
    \item For any pair $(\mathcal{C},\cA)$ as above, its stabilizer acts on $\cB(G)$ as the group $\Omega$. In other words
    $$\mathrm{Stab}_G(\mathcal{C},\cA)/T(\cO)=(N\cap\mathrm{Stab}_G(\mathcal{C}))/T(\cO)=\mathrm{Stab}_N(\mathcal{C})/T(\cO)\cong\Omega.$$
\end{enumerate}

The study of the action of $G$ on its Bruhat--Tits building $\cB(G)$ is an invaluable tool to study the structure of $G$. Many results about $G$ that are hard to prove directly become transparent when studying the building. One such problem is the classification of open maximal compact subgroups of $G$.

Given a facet $c$ of the building of $G$, we are interested in the stabilizer $K_c^+:=\mathrm{Stab}_G(c)$. By the properties listed above, this is a (possibly disconnected) open compact subgroup of $G$. All maximal open compact subgroups of $G$ arise this way. These stabilizers behave well under the conjugation in $G$ since
$$\mathrm{Stab}_G(g\cdot c)=g K_c^+ g^{-1}\quad\text{for all facets $c\subset\cB(G)$ and }g\in G.$$

\begin{example}
    The Bruhat-Tits building of $G=\SL_2(\QQ_p)$ or $G=\PGL_2(\QQ_p)$ is an infinite tree all of whose vertices have degree $p+1$. Each infinite line inside the building is an apartment corresponding to a distinct $F$-split maximal torus of $G$. Consider the apartment $\cA(G,T)$, where $T$ is the group of diagonal matrices, and let $\Delta=\{\alpha\}$ be the simple root as above. Then $\mathcal{C}_0$ is the segment between the vertices $0$ and $\calpha/2$.

    If $G=\SL_2(\QQ_p)$, then 
    $$K_{\{0\}}^+=\SL_2(\ZZ_p),\ K_{\{\calpha/2\}}^+=\begin{psmallmatrix}
        \ZZ_p & p^{-1}\ZZ_p\\
        p\ZZ_p & \ZZ_p
    \end{psmallmatrix}, \text{ and }\cI:=K_{\mathcal{C}_0}^+=\begin{psmallmatrix}
        \ZZ_p^\times & \ZZ_p\\
        p\ZZ_p & \ZZ_p^\times
    \end{psmallmatrix}$$
    are all connected, open compact subgroups of $\SL_2(\QQ_p)$ not conjugate to each other. Since any vertex in the building lies in the $G$-orbit of $0$ or $\calpha/2$, $K_0$ and $K_1$ are the unique maximal compact subgroups of $\SL_2(\QQ_p)$ up to conjugation. The subgroup $\cI$ is called the \textbf{Iwahori subgroup}, it is conjugate to the stabilizer of any facet in the building and is of fundamental importance in the representation theory of $\SL_2(\QQ_p)$.

    On the other hand, if $G=\PGL_2(\QQ_p)$, then 
    $$K_{\{0\}}^+=\mathrm{Stab}(0)=\PGL_2(\ZZ_p),\ K_{\{\calpha/2\}}^+=\mathrm{Stab}(\calpha/2)=\begin{psmallmatrix}
        \ZZ_p & p^{-1}\ZZ_p\\
        p\ZZ_p & \ZZ_p
    \end{psmallmatrix}_{\det\in\ZZ_p^\times}$$
    are both connected open compact subgroups and conjugate in $\PGL_2(F)$ by $\begin{psmallmatrix}
        0 & 1\\
        p & 0
    \end{psmallmatrix}$, so $\PGL_2(F)$ has one unique \textit{connected} maximal compact subgroup up to conjugacy. But it has another class of maximal compact subgroups, namely     
    $$K_{\mathcal{C}_0}^+=\begin{psmallmatrix}
        \ZZ_p^\times & \ZZ_p\\
        p\ZZ_p & \ZZ_p^\times
    \end{psmallmatrix}_{\det\in\ZZ_p^\times}\bigsqcup\begin{psmallmatrix}
    0 & 1\\
    p & 0
    \end{psmallmatrix}\begin{psmallmatrix}
        \ZZ_p^\times & \ZZ_p\\
        p\ZZ_p & \ZZ_p^\times
    \end{psmallmatrix}_{\det\in\ZZ_p^\times},$$
    a disconnected open compact subgroup. We define the Iwahori subgroup $\cI$ to be identity connected component of $K_{\mathcal{C}_0}^+$.
\end{example}

The example above suggests that the connected components of the stabilizers of facets in the building depend in a subtle way on the group $\Omega$. This is indeed the case, and we discuss this connection now. Since $\Omega=\mathrm{Stab}_{\widetilde{W}}(\mathcal{C}_0)$ and $\mathcal{C}_0$ is bounded by hyperplanes corresponding to $S_{\aff}$, there is a natural homomorphism
$$\Omega\longrightarrow\Aut(S_{\aff}).$$
All permutations of $S_{\aff}$ induced by $\Omega$ preserve the affine Dynkin diagram associated to $S_{\aff}$, and if $G$ is simple of adjoint type, then all such automorphisms of $S_{\aff}$ are induced by $\Omega$. This greatly restricts the size of $\Omega$.

Next, fix some \textit{proper} subset $J\subset S_{\aff}$ and consider the \textit{standard} facet 
$$c_J=\{x\in\overline{\mathcal{C}_0}\ |\ \langle\alpha,x\rangle\in\ZZ\text{ for }\alpha\in J\text{ and }\langle\alpha,x\rangle\not\in\ZZ\text{ for }\alpha\in S_{\aff}-J\}.$$
Two facets $c_{J_1}$ and $c_{J_2}$ are conjugate under the action of $G$ if and only if $J_1$ and $J_2$ lie in the same $\Omega$-orbit. Moreover, any facet $c$ in the building is conjugate to $c_J$ for some proper subset $J\subset S_{\aff}$. In other words, there is a bijection between
\[
    \{\Omega-\text{orbits of }J\subsetneq S_{\aff}\}\longleftrightarrow\{G-\text{orbits of facets $c$ in the BT-building}\},
\]
satisfying that for any two $J_1\subseteq J_2$, their corresponding facets satisfy $c_{J_2}\subset \overline{c_{J_1}}$.

The groups $K_c^+$ will be of great importance since they bridge the representation theory of $p$-adic groups with the that of finite groups of Lie type. To explain this connection, we note that there is a short exact sequence
$$1\longrightarrow U_c\longrightarrow K_c^+\longrightarrow \overline{K}_c^+\longrightarrow 1,$$
where $U_c$ is the pro-unipotent radical of $K_c^+$ and $\overline{K}_c^+$ is the group of $k$-rational points of a (possibly disconnected) reductive group $\mathbf{\overline{K}}_c^+$ over $k$. To avoid disconnected quotients, it is convenient to introduce the notion of a \textit{parahoric subgroup}.

\begin{definition}
    A \textbf{parahoric subgroup} $K_c$ is the inverse image in $K_c^+$ of the group $\overline{K}_c$ of $k$-rational points of the identity component $\mathbf{\overline{K}}_c$ of $\mathbf{\overline{K}}_c^+$. We shall sometimes denote ``parahoric subgroup" to the triple $(K_c,U_c,\overline{K}_c)$. If $c=c_J$ is a standard facet, then we simply write $(K_J,U_J,\overline{K}_J)$ for its associated parahoric subgroup.
\end{definition}

Naturally, two parahoric subgroups are conjugate in $G$ if and only if the corresponding facets of the building are in the same $G$-orbit. Thus, there is an inclusion-preserving bijection between 
\[
    \{\Omega-\text{orbits of }J\subsetneq S_{\aff}\}\longleftrightarrow\{G-\text{conjugacy classes of parahoric subgroups } (K,U_K,\overline{K})\}
\]

In particular, if $c$ is open in the building, then $(K_c,U_c,\overline{K}_c)$ is a minimal parahoric subgroup. These is called an \textbf{Iwahori subgroups}, and they are all conjugate in $G$. The standard Iwahori subgroup $\cI$ corresponds to $J=\emptyset\subsetneq S_{\aff}$.

Moreover, if the facet $c$ corresponds to $J\subsetneq S_{\aff}$, then
$$K_c^+/K_c\cong\Omega_J=\mathrm{Stab}_\Omega(J).$$
These results can be directly verified for $\SL_2$, $\PGL_2$ and $\GL_2$ using the examples above.

\begin{example}\label{example:G2_quotients}
    Suppose that $G=G_2(F)$. The affine Dynkin diagram of $G_2$ has no symmetries, so $\Omega=1$ and the extended affine weyl group $\widetilde{W}$ is a Coxeter group of type $\tilde{G_2}$. Since $S_{\aff}=\{s_0,s_1,s_2\}$, there are $7$ conjugacy classes of parahoric subgroups, satisfying
    \begin{align*}
        \overline{K}_{\{s_1,s_2\}}=G_2(k),\quad \overline{K}_{\{s_0,s_1\}}=\SL_3(k), \quad \overline{K}_{\{s_0,s_2\}}=(\SL_2(k)\times\SL_2(k))/\mu_2\\
        \overline{K}_{\{s_1\}}=\overline{K}_{\{s_2\}}=\overline{K}_{\{s_0\}}=\GL_2(k)\quad\text{and}\quad \overline{K}_{\emptyset}=T(k)=(k^\times)^2.
    \end{align*}
\end{example}

As we mentioned earlier, parahoric subgroups are of great important for they provide a bridge between smooth admissible representations of $G$ and finite dimensional representations of the finite groups of Lie type $\overline{K}_c$. In the next subsection, we carefully describe this bridge and how it helps us classify certain representations of $G$.

\iffalse
\subsection{Types for Bernstein blocks and Hecke algebras}
In Section \ref{sec:Bernstein} we stated the Bernstein decomposition, a fundamental result in the complex representation theory of $p$-adic groups. The upshot of this result a priori is clear -- one can restrict attention to each block individually and study the irreducible objects in each block instead of the entire category $\Rep(G)$. In this section, we briefly introduce the notion of types and their corresponding Hecke algebra, which help us understand each individual Bernstein block. We give precise results for so-called \textit{depth-zero Bernstein blocks} which will be required later on. We begin with the definition of a \textit{type}.
\begin{definition}
    Let $[M,\sigma]\in\mathfrak{J}(G)$ be a pair parametrizing a Bernstein block $\Rep(G)_{[M,\sigma]}$. A pair $(K,\rho)$ consisting of an open compact subgroup $K$ of $G$ and a smooth irreducible representation $\rho$ of $K$ is called a $[M,\sigma]$-type if, for any $(\pi,V)\in\Irr(G)$, the following two conditions are equivalent:
    \begin{itemize}
        \item The representation $(\pi,V)$ lies in the Bernstein block $\Rep(G)_{[M,\sigma]}$.
        \item The restriction of $\pi$ to $K$ contains $\rho$; in other words, $\Hom_K(\rho,\pi|_K)\neq 0$.
    \end{itemize}
\end{definition}

Associated to every pair $(K,\rho)$, where $K$ is a compact open subgroup of $G$ and $(\rho,W)$ is an irreducible smooth representation of $K$, one can associate the \textit{Hecke algebra} 
$$\cH(G,K,\rho):=\End_G(\cInd_K^G\rho),$$
with composition of functions as the product.
Alternatively, one can show that the Hecke algebra $\cH(G,K,\rho)$ can be seen as the $\CC$-vector space of functions $f:G\rightarrow\End_\CC(W)$ satisfying
\begin{itemize}
    \item $f(k_1gk_2)=\rho(k_1)\circ f(g)\circ\rho(k_2)$ for any $k_1,k_2\in K$ and $g\in G$.
    \item the support of $f$ is compact,
\end{itemize}
together with multiplication given by \textit{convolution} defined by
\[
    (f_1*f_2)(g)=\sum_{x\in G(F)/K}f_1(x)f_2(x^{-1}g).
\]

The importance of the theory of types in studying individual Bernstein blocks is highlighted in the following theorem.
\begin{theorem}
    Let $(K,\rho)$ be an $[M,\sigma]$-type. Then the Bernstein block $\Rep(G)_{[M,\sigma]}$ is equivalent to the category of right unital $\cH(G,K,\rho)$-modules; i.e.
    $$\Rep(G)_{[M,\sigma]}\simeq\cH(G,K,\rho)\textrm{ - mod}.$$
\end{theorem}
Of course, such a result is only useful if one can
\begin{enumerate}
    \item construct types for the Bernstein blocks we are interested in,
    \item understand the structure of the corresponding Hecke algebras and
    \item describe the irreducible unital right modules of the Hecke algebra.
\end{enumerate}

In most cases, one can answer the three questions giving rise to beautiful, deep and interesting mathematics. One of the main aims of this document is to try to answer these questions. Let us answer these questions first in a particularly simple case.

\begin{example}\label{example:supercuspidal_type}
    Suppose that $G$ is a simple group and that $x\in\cB(G)$ is a vertex in building (facet of minimal dimension) and let $(K_x,U_x,\overline{K}_x)$ be the corresponding parahoric subgroup. If $\sigma$ is an irreducible smooth representation of $K_x^+$ that is trivial on $U_x$ and such that $\sigma|_{K_x}$ is a cuspidal representation of $\overline{K}_x$, then 
    $$\pi:=\cInd_{K_x^+}^G\sigma$$
    is a supercuspidal representation of $G$, and $(K_x^+,\sigma)$ is a $[G,\pi]$-type. Moreover, by Schur's lemma, we know that 
    $$\cH(G,K_x^+,\sigma)=\End_G(\cInd_{K_x^+}^G\sigma)=\End_G(\pi)=\CC$$
    is a $1$-dimensional vector space. This implies that $\Rep(G)_{[G,\pi]}$ has a unique irreducible object, which of course is $\pi=\cInd_{K_x^+}^G\sigma$. Moreover, $\pi$ has no non-trivial extensions, so
    $$\Rep(G)_{[G,\pi]}=\{\pi,\pi\oplus\pi,\pi\oplus\pi\oplus\pi,\ldots\}.$$
\end{example}




\textcolor{red}{Before we finish this section, we state the answer of the first question for \textit{depth-zero blocks}.}

\begin{definition}
    An irreducible smooth representation $(\pi,V)$ of $G$ has \textit{depth-zero} if there is some parahoric subgroup $(K,U_K,\overline{K})$ such that $V^{U_K}\neq 0$.
\end{definition}    

For the remainder of the section, assume for simplicity that $G$ is simple. Suppose that $M$ is a Levi subgroup of $G$ and that $\sigma$ is a depth-zero supercuspidal representation $\sigma$ of $M$. A well-known result of Moy and Prasad states that there is a vertex $x\in\cB(M)\subseteq\cB(G)$ with corresponding stabilizer $K^{(M)+}_x$, parahoric subgroup $(K^{(M)}_x,U^{(M)}_x,\overline{K}^{(M)}_x)$ and a cuspidal representation $\tilde\tau$ of $\overline{K}^{(M)}_xZ(G)$ such that 
$$\sigma\cong\cInd_{K^{(M)}_xZ(G)}^G\tilde\tau.$$
\fi



\subsection{Parahoric restriction and unipotent representations}
In this section, we finally focus our attention in representation theoretic aspects of the $p$-adic group $G$, using the techniques developed in previous ones. The main object of study in the local Langlands correspondence is the category of smooth admissible complex representations of $G$, denoted by $\Rep(G)$. Let $\Irr(G)$ be the set of irreducible representations in $\Rep(G)$. This is a large, and we are interested in a particular subset of known as \textit{unipotent} irreducible representations of $G$ and denoted by $\Irr_{\un}(G)$, which we now define using parahoric subgroups.

Consider a smooth admissible representation $(\pi,V)$ of $G$ and let $(K,U_K,\overline{K})$ be a parahoric subgroup corresponding to a facet $c$. The space $V^{U_K}$ of fixed points under the pro-unipotent radical is naturally a representation of $\overline{K}=K/U_K$. We can take this idea one step further and define the \textit{parahoric restriction functor}
\begin{equation}\label{eqn:par_restriction}
    \res_K:R(G)\longrightarrow R(\overline{K}),\quad V\longmapsto V^{U_K},\quad\text{for all }V\in\Irr(G),
\end{equation}
where $R(G)$ is the $\CC$-span of $\Irr(G)$ and $R(\overline{K})$ is the $\CC$-span of the irreducible representations of $\overline{K}$. This is well-defined since the representations are assumed to be admissible. The existence of such a functor is very powerful -- we can then apply the techniques of representation theory of finite groups of Lie such as Deligne--Lusztig induction in the setting of $p$-adic groups. Let us begin with a natural definition.

\begin{definition}
    Let $(K,U_K,\overline{K})$ be a parahoric subgroup and $(\sigma, E)$ be a cuspidal representation of $\overline{K}$. Define 
    $$\Irr(G,K,E)=\{(\pi,V)\in\Irr(G)\ |\ \text{the $\overline{K}$-module }V^{U_K}\text{ contains the $\overline{K}$-module }E\}.$$ 
\end{definition}

The definition of a unipotent representation is now straightforward.

\begin{definition}
    We say that an irreducible representation $(\pi,V)$ of $G$ is \textit{unipotent} if there is a parahoric subgroup $(K,U_K,\overline{K})$ such that $V^{U_K}$ contains a cuspidal unipotent representation of $\overline{K}$; that is, if $(\pi,V)\in\Irr(G,K,E)$ for some pair $(K,E)$ where $E$ is unipotent.
\end{definition}

Therefore, the set of unipotent representations up to isomorphism is
$$\Irr_{\un}(G)=\bigcup_{\substack{J\subsetneq S_{\aff}\\ E \text{ cusp. unip. $\overline{K}_J$-rep}}}\Irr(G,K_J,E).$$
We say that each set $\Irr(G,K_J,E)$ generates a distinct \textit{block} of representations of $G$.

We note that if we replace the $p$-adic group $G$ for the finite group of Lie type $G^F$ and \textit{parahoric} by \textit{parabolic}, then we recover the definition of a unipotent representation in $G^F$. 

\begin{example}\label{example:unip_blocks}
    Let $G$ be a split reductive $p$-adic group with split maximal torus $T$. Let $\cI$ be an Iwahori subgroup with pro-unipotent radical $\cI^+$. Then the reductive quotient $\cI/\cI^+$ is isomorphic to $T(k)$. Thus, all irreducible representations of $\cI/\cI^+$ are $1$-dimensional and the only unipotent representation is the trivial one. Therefore, the irreducible \textit{Iwahori-spherical} representations
    $$\Irr(G,\cI,\mathbf{1})=\{(\pi,V)\in\Irr(G)\ |\ V^{\cI}\neq 0\}$$
    are all unipotent. It is a known fact that this set coincides with the set of irreducible subrepresentations of $\cInd_B^G\chi$, where $\chi$ is an unramified character of $T$. We say that $\Irr(G,\cI,\mathbf{1})$ generates the \textit{principal block}.
    \iffalse
    \begin{enumerate}
        \item 
        \item Let $(K,U_K,\overline{K})$ be a maximal parahoric subgroup corresponding to a vertex of the building associated to $G$ and let $(\sigma,E)$ be a cuspidal (not necessarily unipotent) representation of $\overline{K}$ viewed as a representation of $K$ by inflation. Then the compactly induced $(\pi,V):=\cInd_K^G(\sigma,E)$ is an irreducible supercuspidal representation and by Frobenius reciprocity
        $$(\pi,V)\in\Irr(G,K,E).$$
        In fact, as we shall observe later, we have that $\Irr(G,K,E)=\{(\pi,V)\}$ and consequently \textcolor{red}{(potentially mention type theory briefly)} the block 
        $$\Rep(G)_{[G,\pi]}=\{\pi,\pi\oplus\pi,\pi\oplus\pi\oplus\pi,\ldots\}.$$
    \end{enumerate}
    \fi    
\end{example}

\begin{example}\label{rem:An_unipotent}
    For $n\geq 1$, reductive groups over finite fields of type $A_n$ have no irreducible cuspidal unipotent representations. Therefore, if $G$ is a reductive $p$-adic group of type $A_n$ and $J\subseteq S_{\aff}$ is non-empty, then $\overline{K}_J$ has no cuspidal unipotent representations. This implies that the set of irreducible unipotent representations of $G$
    $$\Irr_{\un}(G)=\Irr(G,\cI,\mathbf{1})$$
    coincides with the irreducible Iwahori-spherical representations of $G$.
\end{example}

\begin{example}
    Let $G$ be a $p$-adic group of type $G_2$. For $J\subsetneq S_{\aff}$, $\overline{K}_J$ has unipotent cuspidal representations only if $J=\emptyset$ and $J=J_0=\{s_1,s_2\}$. In the first case, we obtain the principal block $\Irr(G,\cI,\mathbf{1})$. In the second case, $\overline{K}_J$ has four unipotent characters, so 
    $$\Irr_un(G)=\Irr(G,\cI,\mathbf{1})\sqcup\bigsqcup_{\rho\in\{1,-1,\theta,\theta^2\}}\Irr(G,K_{J_0},G_2[\rho]).$$
    However, these last four blocks are very simple since they have a unique irreducible unipotent representation, namely $\Irr(G,K_{J_0},G_2[\rho])=\{\cInd_{K_{J_0}}^G G_2[\rho]\}$ for each $\rho\in\{1,-1,\theta,\theta^2\}$.
\end{example}

The next step is to ensure that unipotent representations behave under the parahoric restriction functor \eqref{eqn:par_restriction}. The following two results ensure this is indeed the case.
\begin{proposition}\label{prop:unipotent_padic}
    Let $(\pi,V)$ be an irreducible admissible representation of $G$. If there is some parahoric subgroup $(K,U_K,\overline{K})$ such that $V^{U_K}$ contains a (potentially non-cuspidal) unipotent representation of $\overline{K}$, then $(\pi,V)$ is a unipotent representation of $G$.
\end{proposition}
\begin{proof}
    By assumption, $V^{U_K}$ contains a unipotent irreducible representation $(\tau,W)$ of $\overline{K}$ so $\Hom_K(W,V^{U_K})\neq 0$. By Proposition \ref{prop:harishchandra}, there is some standard parabolic subgroup $\overline{P}=\overline{U_P}\cdot\overline{L_P}$ of $\overline{K}$ and cuspidal unipotent representation $(\sigma,E)$ of $\overline{L_P}$ such that 
    $$\Hom_{K}(W,\Ind_{\overline{P}}^{\overline{K}}E)\neq 0,$$
    where we view the $\overline{K}$ representations as inflated $K$ representations, trivial on $U_K$. By the classification of parahoric subgroups in $G$, it follows that $\overline{P}=H/U_K$, where $(H,U_H,\overline{H})$ is another parahoric subgroup contained in $K$. Moreover, we have the inclusions $U_K\subseteq U_H\subseteq H\subseteq K$ and therefore $\overline{U_P}=U_H/U_K$ and $\overline{L_P}=\overline{H}=H/U_H$. Since induction and inflation are commuting operations, it follows that
    $$\mathrm{Inf}_{\overline{K}}^K\Ind_{\overline{P}}^{\overline{K}}E\cong\Ind_{H}^K\mathrm{Inf}_{\overline{H}}^H E$$
    and hence $\Hom_K(W,\Ind_H^K E)\neq0$. Since $W$ is irreducible and $K$ is compact, it also follows that 
    $$\Hom_K(\Ind_H^K E,V^{U_K})=\Hom_H(E,V^{U_K})\neq 0.$$
    Since the representation $E$ is trivial on $U_H$, the image of any $H$-equivariant map $E\to V^{U_K}$ lies inside $V^{U_H}$. Thus, 
    $$\Hom_H(E,V^{U_H})=\Hom_H(E,V^{U_K})\neq 0,$$
    and this concludes the proof.
\end{proof}

Conversely, we would like to show that for any irreducible unipotent representation $(\pi,V)$ of $G$, the irreducible $\overline{K}$-submodules of $V^{U_K}$ are all unipotent, for any parahoric subgroup $(K,U_K,\overline{K}).$ This is a direct corollary of the following theorem.

\begin{theorem}\label{thm:unip_restriction}
    Suppose $I\subsetneq S_{\aff}$ and that $V^{U_I}$ contains the cuspidal unipotent representation $\sigma$ of $\overline{K}_I$. If $J\subsetneq S_{\aff}$ with $V^{U_J}\neq 0$, and $J$ is minimal with respect to this property, then there is $\omega\in\Omega$ such that $I=\omega J$, and $V^{U_J}$ consists of copies of $\sigma^{\omega}$. Moreover, if $G$ is exceptional, then $J=I$.
\end{theorem}
\begin{proof}
    See Moy-Prasad for a complete account for general cuspidal representations and not necessarily unipotent and Reeder's paper for a sketch in the unipotent setting.
\end{proof}

\begin{cor}
    Let $(\pi,V)$ be a unipotent representation of $G$ and let $(H,U_H,\overline{H})$ be a parahoric subgroup. Then the $\overline{H}$-irreducible components of $V^{U_H}$ are all unipotent.
\end{cor}
\begin{proof}
    Since $(\pi,V)$ is unipotent, $(\pi,V)\in\Irr(G,K_J,E)$ for some $J\subsetneq S_{\aff}$ and cuspidal unipotent representation $E$ of $\overline{K}_J$.
    Let $(\tau,W)$ be a $\overline{H}$-irreducible component of $\pi^{U_H}$. By conjugating if necessary, we may assume that $(H,U_H,\overline{H})$ is a standard parahoric subgroup. Analogously to the proof of Proposition \ref{prop:unipotent_padic}, there is some $I\subsetneq S_{\aff}$ such that $(K_I,U_I,\overline{K}_I)$ is contained in $K$ and $\tau$ is a subrepresentation of $\Ind_{\overline{K_I}}^{\overline{H}}\sigma$. By Theorem \ref{thm:unip_restriction}, $I$ is the same $\Omega$-orbit as $J$ and $\sigma$ is cuspidal unipotent. By Proposition $\ref{prop:harishchandra}$, this implies that $(\tau,W)$ is also unipotent.
\end{proof}

\begin{cor}\label{cor:disjoint_reps}
    For any two pairs $(K,E)$, $(K',E')$ of a parahoric subgroup and a cuspidal unipotent representation of the reductive quotient, $\Irr(G,K,E)$ and $\Irr(G,K',E')$ are either disjoint or equal.
\end{cor}

\iffalse
\begin{theorem}
    Let $(\pi,V_\pi)$ be an irreducible admissible representation of $G$. Suppose that there is some parahoric $P$ such that $V^{P^+}\neq 0$ contains an irreducible cuspidal representation $(\sigma,V_\sigma)$ of $\overline{P}$. Then, for any parahoric $Q$ such that $V^{Q^+}\neq 0$, any $\overline{Q}$-submodule of $V^{Q^+}$ contains a cuspidal representation $(\tau,V_\tau)$ of $\overline{L}$ for some parahoric subgroup $L\subseteq Q$. Moreover, there is some $g\in G$ such that $L=\prescript{g}{}{P}$ and $\tau=\prescript{g}{}{\sigma}$.
\end{theorem}

\begin{proof}
    From the proof of the previous result implies that any $\overline{Q}$-submodule of $V^{Q^+}$ contains a cuspidal representation $\tau$ of $\overline{L}$ for some parahoric subgroup $L\subseteq Q$. Thus, we may assume without loss of generality that $Q=L$ so that the irreducible $\overline{Q}$-submodule $(\tau,V_\tau)$ of $V^{Q^+}$ is cuspidal. To prove the second statement, we let $E_\tau:V_\pi\rightarrow V_\tau$ be a $Q$-equivariant projection. Then, for any $g\in G$, we have the linear map 
    $$\varphi_g=E_\tau\circ\pi(g^{-1}):V_\sigma\longrightarrow V_\tau,$$
    and for any $h\in P\cap gQg^{-1}$ and $v\in V_\sigma$, we have that 
    $$\varphi_g\circ\sigma(h)(v)=E_\tau(\pi(g^{-1}hg)\pi(g^{-1})v)=\tau(g^{-1}hg)\circ   E_\tau(\pi(g^{-1}))(v)=\tau(g^{-1}hg)\circ\varphi_g(v).$$
Since $\pi$ is an irreducible representation of $G$, there must be some $g\in G$ such that $\pi(g^{-1})V_\sigma\not\subseteq\ker(E_\tau)$, in which case $\varphi_g\neq 0$. The image of $P\cap gQg^{-1}$ inside $\overline{P}$ (resp $\overline{gQg^{-1}}$) is a parabolic subgroup $\overline{P}_{\prescript{g}{}{Q}}$ (resp. $\overline{\prescript{g}{}{Q}}_{P}$) 
\end{proof}

\textcolor{red}{Need to find proofs for this!!}
\fi

Analogously to the construction of $R(G)$, we define $R_{\textrm{un}}(G)$ to be the $\CC$-span of the irreducible unipotent representations $\Irr_{\un}(G)$. Lemma \ref{prop:unipotent_padic} and Theorem \ref{thm:unip_restriction} implies that for each parahoric subgroup $(K,U_K,\overline{K})$ there is a well-defined \textit{restriction function}
\begin{equation*}
    \res_{\un}^K:R_{\textrm{un}}(G)\longrightarrow R_{\un}(\overline{K}),\quad V\longmapsto V^{U_K},\quad\text{for all }V\in\Irr(G).
\end{equation*}
It is also convenient to consider simultaneously all such functions for all conjugacy classes of maximal parahoric subgroups, so we define $\res_{\textrm{un}}^{\textrm{par}}=(\res_{\textrm{un}}^K)_K$. These are the vertical maps in the diagram \ref{conj:main}.


\iffalse In the previous section, we defined a nonabelian Fourier transform 
$$\FT^K:R_{\un}(\overline{K})\longrightarrow R_{\un}(\overline{K}),$$
taking the unipotent characters to the almost characters of $\overline{K}$, having nice geometric properties. We also consider all these maps simultaneously for all conjugacy classes of maximal parahoric subgroups, which we denote as $\FT^{\mathrm{par}}=(\FT^K)_{K \text{ maximal}}$.
\fi


\iffalse
\subsection{Parahoric restriction for unipotent supercuspidal representations}

Let $G$ be the simple $p$-adic group over $F$. In this section, we investigate the parahoric restriction of supercuspidal unipotent representations of $G$ (if any) with respect to maximal parahoric subgroups. A well-known result of Moy and Prasad states that any supercuspidal unipotent representation $(\pi,V)$ of $G$ is obtained by compactly inducting an irreducible smooth representation $(\rho,E)$ of $K_x^+$, where $x\in\cB(G)$ is a vertex, such that $\rho|_{K_x}$ is the inflation of a cuspidal representation of $\overline{K}_x$. By conjugating if necessary, we may assume that $x$ lies in the closure of the fundamental alcove $\mathcal{C}_0$. Explicitly, 
$$\pi\cong\cInd_{K_x^+}^G\rho,$$
so by Frobenius reciprocity we have that 
\begin{equation*}
    \Hom_{K_x}(\rho|_{K_x},\pi^{U_x})\supseteq\textcolor{red}{\text{this should be equality }}\Hom_{K_x^+}(\rho,\pi^{U_x})=\Hom_{K_x^+}(\rho,\pi)=\Hom_G(\cInd_{K_x^+}^G\rho,\pi)\cong\CC,
\end{equation*}
so $(\pi,V)\in\Irr(G,K_x,E)$. If $J=\{\alpha\in S_{\alpha}\ |\ \langle\alpha,x\rangle=0\}$, then $K_x=K_J$ and by cuspidality $J$ is a minimal subset of $S_{\aff}$, up to the action of $\Omega$, such that $\pi^{U_J}\neq 0$. Now let $I\subsetneq S_{\aff}$ be another subset such that $V^{U_I}\neq 0$. If $\pi^{U_I}$ contains an irreducible cuspidal representation of $\overline{K}_I$ then $I$ is also minimal with respect to $V^{U_I}\neq 0$ and by Theorem \ref{thm:unip_restriction}, $I$ and $J$ are in the same $\Omega$-orbit. If $\pi^{U_I}$ does not contain any irreducible cuspidal representation, then by \ref{prop:harishchandra}, there is some $J'\subset I$ such that $\pi^{U_{J'}}$ contains a cuspidal representation of $\overline{K}_{J'}$ so $J$ and $J'$ lie in the same $\Omega$-orbit, but this is a contradiction since $K_J$ is a maximal parahoric subgroup of $G$. We have thus shown:

\begin{lemma}
    Let $(\pi,V)$ be a supercuspidal unipotent representation of $G$. Then there is one unique $\Omega$-orbit $[J]$ of subsets of $S_{\aff}$, all of which are maximal such that $\pi^{U_I}\neq 0$ if and only if $I\in[J]$.
\end{lemma}



\vspace{3cm}

Suppose $G$ has type $G_2$ with simple reflections $S_{\aff}=\{s_0,s_1,s_2\}$. We note that $\Omega=\{1\}$ so $\Omega$-orbits are all singletons. By combining Example \ref{example:G2_quotients} and Remark \ref{rem:An_unipotent}, given $J\subsetneq S_{\aff}$, the reductive quotient $\overline{K}_J$ has cuspidal unipotent representations if and only if $J=J_0:=\{s_1,s_2\}$ or $J=\emptyset$.

In the first case, $K_0:=K_{J_0}$ is the stabilizer of the origin in the apartment $\cA(G,T)$ and $\overline{K_0}=G_2(\FF_q)$ has $4$ cuspidal unipotent representations labelled $G_2[1],G_2[-1],G_2[\theta]$ and $G_2[\theta^2]$, where $\theta$ is a primitive third root of unity. For any of these representations $\sigma$, Example \ref{example:unip_blocks} shows that the compactly induced representation $\pi=\cInd_{K_0}^G\sigma$ is irreducible and supercuspidal, and 
$$\Irr(G,K_0,\sigma)=\{\pi\}.$$
In the second case, $K_\emptyset=\cI$ is the standard Iwahori subgroup (stabilizer of the fundamental alcove) and the only cuspidal unipotent representation of $I/U_I$ is the trivial character. Therefore,
\[
    \Irr_{\un}(G)=\Irr(G,\cI,\mathbf{1})\ \bigcup\ \{\cInd_{K_0}^GG_2[1],\cInd_{K_0}^GG_2[-1],\cInd_{K_0}^GG_2[\theta],\cInd_{K_0}^GG_2[\theta^2]\}.
\]
In the next section, we shall describe a natural way to parametrize this family. We shall now investigate the parahoric restriction of these representations with respect to the \textit{maximal parahoric subgroups} $K_0$, $K_1:=K_{\{\alpha_0,\alpha_2\}}$ and $K_2:=K_{\{\alpha_0,\alpha_1\}}$. 

Firstly, consider the case $\pi=\cInd_{K_0}^G\sigma$ for a cuspidal unipotent representation $\sigma$ of $G_2(\FF_q)$. By Frobenius reciprocity, it follows that $\pi^{U_{K_0}}=\sigma\neq0$ and therefore by Theorem \ref{thm:unip_restriction}, the set $J_0=\{\alpha_1,\alpha_2\}$ is minimal with respect to the property that $V^{U_J}\neq 0$. Suppose for a contradiction that $V^{U_{J_i}}\neq 0$ for $i=1$ or $i=2$, where $J_1:=\{\alpha_0,\alpha_2\}$ and $J_2:=\{\alpha_0,\alpha_1\}$. Since $J_1$ or $J_2$ cannot be minimal with respect to the same property, then $V^{U_{\{\alpha_0\}}}\neq0$. But $\overline{K}_{\{\alpha_0\}}$ has no cuspidal unipotent representations, so $V^{K_{\emptyset}}=V^{I}\neq0$, a contradiction to Corollary \ref{cor:disjoint_reps}.

\fi
\subsection{The Langlands parametrization of unipotent representations}

In this section, we give an overview on the Langlands parametrization of unipotent representations achieved by Lusztig in his celebrated paper of 1995. It relates the unipotent representations of $G$ with certain geometric structures of its complex dual group $G^\vee$. This is a complex reductive group, whose root datum is dual to that of $G$. Here are some examples for classical groups.
\begin{center}
    \begin{tabular}{c|cccccc}
        
        $G$ & $\GL_n(F)$ & $\SL_n(F)$ & $\PGL_n(F)$ & $\Sp_{2n}(F)$ & $\mathrm{PSp}_{2n}(F)$ & $\SO_{2n+1}(F)$\\\hline
        $G^\vee$ & $\GL_n(\CC)$ & $\PGL_n(\CC)$ & $\SL_n(\CC)$ & $\SO_{2n+1}(\CC)$ & $\mathrm{Spin}_{2n+1}(\CC)$ & $\Sp_{2n}(\CC)$\\
    \end{tabular}    
\end{center}

This parametrization is widely used and is key to define the dual nonabelian fourier transform $\FT^\vee_{\text{ell}}$ on a certain space of unipotent representations of $G$. To state this result, we need to consider the representations not only of $G$, but also of its \textit{pure inner twists} $\InnT^p(G)$. We will not discuss the precise definition here; instead it will suffice to know that there is a canonical bijection
\begin{equation}\label{eqn:pureinnertwists}
    \InnT^p(G)\longleftrightarrow Z(G^\vee),
\end{equation}
where the unique split form corresponds to the trivial character. In particular, if $G$ is simply connected, $G^\vee$ is adjoint so it has trivial centre and $G$ has no pure inner twists other than itself. The three example we will visit later are all simply connected, so no pure inner twists will arise. The main result of this section is the following.

\begin{theorem}[The arithmetic-geometric correspondence]\label{thm:correspondence}
    There is an natural bijection between the sets
    \begin{align*}
        G^\vee\backslash\{(x,\phi)\ |\ x\in G^\vee, \phi\in\widehat{A_x}\}&\longleftrightarrow\bigsqcup_{G'\in\InnT^p(G)}\Irr_{\un}(G')\\
        (x,\phi)&\longmapsto (\pi_{(x,\phi)},V_{(x,\phi)}),
    \end{align*}
    where $A_x=Z_{G^\vee}(x)/Z_{G^\vee}(x)^0$ is the component group of the centralizer of $x$ in $G^\vee$.
\end{theorem}

This is a remarkable result, which we now study in some detail, fits naturally within the local Langlands correspondence and the theory of Langlands parameters \textcolor{red}{(overview of this in chapter 4 Beth and Dan)}. Firstly, given a pair $(x,\phi)$, where $x\in G^\vee$ and $\phi\in\widehat{A_x}$, one can easily determine the inner twist $G'$ acting on the representation $(\pi_{(x,\phi)},V_{(x,\phi)})$. We can $\phi$ as a character on $Z_{G^\vee}(x)$ and we can restrict it to get a character of $Z(G^\vee)$. Tracing back through the bijection in \eqref{eqn:pureinnertwists} we obtain the corresponding pure inner twist $G'$.

Secondly, given a pair $(x,\phi)$ parametrizing a representation of $G$, we would like to know inside which unipotent block does $(\pi_{(x,\phi)},V_{(x,\phi)})$ lie. This is a much harder question to answer than the previous one, and its solution delves deep into the proof of the arithmetic-geometric correspondence. For the remaining of this section, we give an overview of some of the ideas involved which will of use for us later. The main one is the use of \textit{Hecke algebras} to classify irreducible representation of an individual block. This approach has the added advantage that it reduces the problem of classifying infinite dimensional irreducible representation of $G$ of some block to classifying simple finite dimensional modules over the Hecke algebra. 

Suppose that $(\pi,V)\in\Irr(G,K,E)$ is an irreducible unipotent representation of $G$, where $(E,\sigma)$ is a cuspidal unipotent representation of $\overline{K}$. By definition, $V^{U_K}$ is a finite dimensional $\overline{K}$-representation, so 
$$\Hom_G(\cInd_{K}^G \sigma,\pi)=\Hom_K(\sigma,\pi)=\Hom_K(\sigma,\pi^{U_K})$$
is a non-zero finite-dimensional module over the algebra $\cH(G,K,\sigma):=\End_G(\cInd_{K}^G \sigma)$ acting by precomposition, denoted the \textit{Hecke algebra} associated to the pair $(K,\sigma)$.
\begin{theorem}
    There is a bijection between 
    \begin{equation*}
        \Irr(G,K,E)\longleftrightarrow\text{ simple right }\cH(G,K,\sigma)-\text{modules}, \quad V\longmapsto\Hom_K(E,V^{U_K}).
    \end{equation*}
\end{theorem}
This is a foundational result in the study of representations of $p$-adic groups, which can be vastly generalized. Still, this does not solve the problem stated above. A pair $(x,\phi)$ parametrizes a simple module of some unipotent Hecke algebra $\cH(G,K,\sigma)$, but which one? The answer is provided by the generalized Springer correspondence (which is something I am learning currently). 
\begin{example}[Classical Springer correspondence]
    As a particular example of this correspondence, we state a criterion for $(x,\phi)$ to parametrize a simple module for the principal Hecke algebra $\cH(G,\cI,\mathbf{1})$ and therefore of Iwahori-spherical representations. This is the content of the classical Springer correspondence. Let $\cB:=G^\vee/B^\vee$ the flag variety of $G^\vee$ and let $\cB^x$ the fixed points under the action of conjugation of $x$. The space $\cB$ naturally parametrizes the set of Borel subgroups of $G^\vee$, while $\cB^x$ parametrizes the set of Borel subgroups \textit{containing} $x$. Now, $\cB^x$ admits a natural action of the group $A_x$ by conjugation. This induces an $A_x$-action on the singular cohomology complex $H^*(\cB^x,\CC)$. Then $(x,\phi)$ corresponds to an Iwahori-spherical representation if and only if the $\phi$-isotropic subspace on $H^*(\cB^x,\CC)$ is non-trivial. Using this, it can be shown that $(x,\mathbf{1})$ always parametrizes an Iwahori-spherical representation of the split form for any $x\in G^\vee$. It is possible to construct explicitly the Hecke algebra module from $(x,\phi)$, but we shall not visit this here.
\end{example}

\iffalse

Firstly, we briefly discuss the results of Kazhdan--Lusztig on the parametrization of Iwahori-spherical representation when $G$ is a $p$-adic reductive group of \textit{adjoint} type. Throughout, let $G^\vee$ be complex dual group of $G$.

We recall that the irreducible Iwahori-spherical representations are in bijection with the irreducible modules of $\cH_\cI=\cH(G,\cI,\mathbf{1})$. Let $\cB$ be the variety of Borel subgroups of $G^\vee$ and let 
$$\mathcal{Z}=\{(B,u,B')\in\cB\times G^\vee\times\cB:u\in B\cap B'\text{ unipotent}\}$$
be the Steinberg variety of $G$, playing a main role in the representation theory of $\cH_\cI$.
Importantly, $G^\vee\times\CC^\times$ acts on $\mathcal{Z}$ by
$$(g,\lambda)(B,u,B')=(gBg^{-1},gu^{\lambda^{-1}}g^{-1},gB'g^{-1}).$$
This action gives rise to the $K$-group $K^{G^\vee\times\CC^\times}(\mathcal{Z})$, which is naturally a $\CC[z,z^{-1}]$-module and satisfies
\begin{equation}\label{eqn:Kgroup}
    K^{G^\vee\times\CC^\times}(\mathcal{Z})\otimes_{\CC[z,z^{-1}]}\CC_q\cong\cH(G,\cI,q).
\end{equation}

Thus, we want to construct the $K^{G^\vee\times\CC^\times}(\mathcal{Z})$-modules and then specialize to $\cH$-modules via \eqref{eqn:Kgroup}. This is performed most naturally with Borel-Moore homology. 

Let $t\in G^\vee$ be semisimple and let $u\in G^\vee$ be unipotent such that $tut^{-1}=u^q$ and let $\cB^{t,u}\subset\cB$ be the subvariety of Borel subgroups containing $t$ and $u$. Then it turns out that $H_*(\cB^{t,u},\CC)$ is naturally a $K^{G^\vee\times\CC^\times}(\mathcal{Z})$-module, usually reducible. Since these constructions are compatible with conjugation by elements of $G^\vee$, the group $Z_{G^\vee}(t,u)$ acts on $H_*(\cB^{t,u},\CC)$ by $K^{G^\vee\times\CC^\times}(\mathcal{Z})$-intertwiners. In fact, the neutral component of $Z_{G^\vee}(t,u)$ acts trivially, so we may regard it as an action of the component group $\pi_0(Z_{G^\vee}(t,u))$. This action can be used to decompose $H_*(\cB^{t,u},\CC)$ as follows:

For each irreducible representation $\rho$ of $\pi_0(Z_{G^\vee}(t,u))$ appearing in $H_*(\cB^{t,u},\CC)$, the space 
$$K_{t,u,\rho}:=\Hom_{\pi_0(Z_{G^\vee}(t,u))}(\rho,H_*(\cB^{t,u},\CC))$$
is a nonzero $K^{G^\vee\times\CC^\times}(\mathcal{Z})$-module, called standard. The data $(t,u,\rho)$ are called \textit{Kazhdan--Lusztig triples} for $(G^\vee,q)$.

\begin{theorem}
    Under the assumption that $G^\vee$ is simply connected, we have that
    \begin{enumerate}
        \item For each Kazhdan--Lusztig triple $(t,u,\rho)$, the $\cH$-module $K_{t,u,\rho}$ has a unique irreducible quotient $L_{t,u,\rho}$.
        \item Every irreducible $\cH$-module if of the form $L_{t,u,\rho}$ for some Kazhdan--Lusztig triple.
        \item If $(t',u',\rho')$ is another triple, then $L_{t,u,\rho}\cong L_{t',u',\rho'}$ if and only if there is some $g\in G$ such that $t'=gtg^{-1}$, $u'=gug^{-1}$ and $\rho'=\rho\circ\mathrm{Ad}(g^{-1})$.
    \end{enumerate}
\end{theorem}

The above theorem is a major result and has many interesting consequences. However, the definition of a Kazhdan--Lusztig triple is slightly awkward since the pair $(t,u)$ does not commute, and consequently the classification of these triples up to $G$-conjugacy seems hard. Thankfully, this situation can be remedied by considering \textit{Kazhdan--Lusztig triples for $(G^\vee,1)$}. These are defined analogously to the Kazhdan--Lusztig triples for $(G,q)$ but replacing $1$ for $q$ throughout. In particular, the semisimple and unipotent part do commute.


\begin{lemma}\label{lem:parametrization}
    Let $G$ be a $p$-adic reductive group over a field $F$ of residue cardinality $q$ and let $G^\vee$ be its complex dual. There exists a bijection
    \begin{align*}
        \{\text{Kazhdan--Lusztig triples for }(G,1)\}/G&\longleftrightarrow\Irr(\cH(G,\cI,q))\\
        (t,u,\rho)\quad\quad &\longmapsto\quad L_{t_q,u,\rho_q},
    \end{align*}
    where and $(t_q,u,\rho_q)$ are obtained from $(t,u,\rho)$ in a prescribed way.
\end{lemma}

We recall that Kazhdan--Lusztig triples for $(G^\vee,1)$ are defined to be tuples $(t,u,\rho)$ such that $\rho$ is 
\textit{an irreducible character of $\pi_0(Z_{G^\vee}(tu))$ appearing in $H_*(\cB(t,u),\CC)$}. This begs the question: if $\rho$ does not satisfy this condition, does the triple $(t,u,\rho)$ parametrize a (not Iwahori-spherical) representation of $G$?

This question was studied and completely resolved by Lusztig in his celebrated paper of 1995. He showed that, in order to get a bijection with all pairs $(t,u,\rho)$ without technical conditions on $\rho$, one needs to consider a wider family of representations. Firstly, one needs to consider not only representations of $G$, but also of all of its \textit{pure inner twists}. We let $\InnT^p(G)$ be the set of pure inner twists of $G$. A well known result states that there is a canonical bijection between the sets
\begin{align}\label{eqn:pure_bijection}
    \InnT^p(G)\longleftrightarrow H^1(F,\mathbf{G}^*)&\longleftrightarrow\Irr(Z_{G^\vee}),\\
    G'&\longmapsto\zeta_{G'}
\end{align}
For instance, if $G$ is a simply connected $p$-adic group, then $Z_{G^\vee}=\{1\}$ and therefore $G$ has no pure inner twists other than itself. Secondly, one needs to consider all unipotent representations, and not just the Iwahori-spherical. The following theorem contains this information.

\begin{theorem}[The arithmetic-geometric correspondence]
    There is an explicit bijection between the sets
    \begin{equation*}
        \bigcup_{G'\in\InnT^p(G)}\Irr_{\un}(G')\longleftrightarrow\mathcal{T}(\sqrt{q})\longleftrightarrow\mathcal{T}(1),
    \end{equation*}
    where $\mathcal{T}(v_0)$ is set containing all triples $(s,u,\rho)$ such that
    \begin{itemize}
        \item $t\in G^\vee$ is semisimple,
        \item $u\in G^\vee$ is unipotent satisfying $tut^{-1}=u^{v_0^2}$,
        \item $\rho$ is an irreducible representation of the group of components of the centralizer group $Z_{G^\vee}(t,u)$.
    \end{itemize}
\end{theorem}

For the remaining of the section, we explain how this result fits within the modern framework of the local Langlands correspondence. Let $W_F$ be the Weyl group of the field $F$ with inertia subgroup $I_F$. Moreover, we set $W_F':=W_F\times\SL_2(\CC)$. 

Under the assumption that $\mathbf{G}$ is a split group, we have the following important definition.
\begin{definition}
    A \textit{Langlands parameter} (or $L$-parameter) for $G$ is a continuous morphism $\varphi:W'_F\rightarrow G^\vee$, where $G^\vee$ denotes the $\CC$-points of the dual group of $\mathbf{G}$, and $\varphi((w,1))$ is semisimple for each $w\in W_F$.
\end{definition}


In its simplest form, the Local Langlands correspondence (LLC) conjectures the existence of a finite to one map between isomorphism classes of smooth admissible complex representations of $G$ and conjugacy classes of Langlands parameters of $G$ satisfying certain nice properties. Using Theorem \ref{thm:correspondence}, we will see that the the unipotent representations of $G$ and its pure inner twists correspond to the following Langlands parameters.

\begin{definition}
    An $L$-parameter $\varphi:W_F\times\SL_2(\CC)\rightarrow G^\vee$ is called \textit{unipotent} if $\varphi(w,1)=1$ for any element $w$ of the inertia subgroup $I_F$ of $W_F$. Such parameters are sometimes called \textit{unramified} Langlands parameters and we denote this set by $\Phi_{\un}(G^\vee)$.
\end{definition}
\begin{remark}
    For any $L$-parameter $\varphi:W_F'\rightarrow G^\vee$, define the commuting elements $u_\varphi=\varphi(1,\left(\begin{smallmatrix}
        1 & 1\\
        0 & 1
    \end{smallmatrix}\right))$ and $s_\varphi=\varphi(\Frob,\Id)$. An application of the Jacobson--Morozov theorem implies that an $L$-parameter is determined by $u_\varphi$ and $\varphi|_{W_F}$ up to $G^\vee$-conjugacy. If the $L$-parameter is, in addition, unipotent, then $\varphi|_{W_F}$ is determined by $s_\varphi$. Thus, unipotent $L$-parameters are parametrized by $G^\vee$ conjugacy classes of pairs $(u,s)$ where $u\in G^\vee$ is unipotent, $s\in G^\vee$ is semisimple and they commute. But this is the same as conjugacy classes of elements of $G^\vee$ (by using the Jordan decomposition). This should be reminiscent of the parametrization of Iwahori-spherical representations in Lemma \ref{lem:parametrization}.
\end{remark}

However, under the LLC correspondence, unramified $L$-parameters do not parametrize unipotent representations, but rather $L$-packets of unipotent representations. To get a one to one correspondence, we need to introduce refinements of the $L$-parameters. Given an $L$-parameter $\varphi$, a natural object of interest is the component group $A_\varphi$ of centralizer $Z_{G^\vee}(\varphi)$ of the image of $\varphi$ inside $G^\vee$. We remark that when $\varphi$ is unipotent, it is determined by the commuting elements $s_\varphi$ and $u_\varphi$ and therefore $Z_{G^\vee}(\varphi)=Z_{G^\vee}(s_\varphi u_\varphi)$. This object is completely analogous to the centralizer $Z_{G^\vee}(t,u)$, considered by Kazhdan and Lusztig in the setting of representations of Hecke algebras. 

\begin{definition}
    An \textit{enhanced pure Langlands parameter} is a pair $(\varphi,\phi)$, where $\varphi:W_F'\rightarrow G^\vee$ is an $L$-parameter and $\phi$ is an irreducible representation of $A_\varphi$. 
\end{definition}

Let us introduce some important notation. Define
$$\Phi_{\mathrm{e,un}}^p(G^\vee)=G^\vee\backslash\{(\varphi,\phi)\ |\ \varphi\text{ unipotent, }\phi\in\widehat{A_\varphi}\},$$
which by the previous paragraph is in natural bijection with the set 
$$G^\vee\backslash\{(x,\phi)\ |\ x\in G^\vee, \phi\in\widehat{A_x}\},$$
where $A_x$ is the component group of $Z_{G^\vee}(x)$.

In this setting the Local Langlands conjecture predicts a natural bijection
\begin{align*}
    \LLC^p_{\un}: G^\vee\backslash\{(x,\phi)\ |\ x\in G^\vee, \phi\in\widehat{A_x}\}\longleftrightarrow\Phi^p_{\mathrm{e,un}}(G^\vee)\longleftrightarrow&\bigsqcup_{G'\in\InnT^p(G)}\Irr_{\un}(G')\\
    (x,\phi)\hspace{1.5cm}\longmapsto \hspace{1.5cm}&\quad\pi(x,\phi),
\end{align*}
where $G'$ runs over the classes of \textit{pure} inner twists of $G$. 

We distinguish between the distinct pure inner twists by looking at characters of $Z_{G^\vee}$. By \eqref{eqn:pure_bijection}, each pure inner twist $G'$ naturally corresponds to some character $\zeta_{G'}$ of $Z_{G^\vee}$. Similarly, for any pure enhanced $L$-parameter $(\varphi,\phi)$, the representation $\phi$ induces a character $\zeta_\phi$ on $Z_{G^\vee_{sc}}$. We say that a pair $(\varphi,\phi)$ is $G'$-relevant if $\zeta_\phi=\zeta_{G'}$, in which case $\pi(x_\varphi,\phi)\in\Irr_{\un}(G')$ if $\varphi$ is unipotent, and we denote the set of $G'$-relevant pure enhanced unipotent $L$-parameters by $\Phi^p_{e,\un}(G)$.
It is then clear that 
$$\Phi_{e,\un}(G^\vee)=\bigsqcup_{G'\in\InnT(G)}\Phi_{e,\un}(G'),$$
and the LLC predicts that $\Phi_{e,\un}(G')$ parametrizes the set $\Irr_{\un}(G')$ for each $G'\in\InnT(G)$.

\begin{example}
    If $\mathbf{G}$ is a simple split \textit{simply connected} algebraic group, then $H^1(F,\mathbf{G}^*)=1$ and therefore there is only one class of pure inner forms of $G$, namely $G$ itself. Correspondingly, $G^\vee=G^\vee_{\ad}$ and $Z_{G^\vee}$ is trivial. Therefore, the above discussion gives a bijection 
    \begin{equation*}
        \LLC_{\un}^p:G^\vee\backslash\{(x,\phi)\ |\ x\in G^\vee,\phi\in\widehat{A_x}\}\longleftrightarrow\Phi_{\mathrm{e,un}}^p(G^\vee)\longleftrightarrow\Irr_{\un}(G^*).
    \end{equation*}
\end{example}

\begin{example}
    If $\mathbf{G}$ is a simple split \textit{adjoint} algebraic group, then $H^1(F,\mathbf{G}^*)=H^1(F,\mathrm{Inn}(\mathbf{G}^*))$ so for each inner twist there is one unique pure inner twist. 
    Therefore, from the previous discussion, unipotent enhanced $L$-parameters are in bijection with the set
    $$G^\vee\backslash\{(x,\phi)\ |\ x\in G^\vee,\phi\in\widehat{A_x}\}, \quad\text{where}\quad A_x=Z_{G^\vee}(x)/Z_{G^\vee}(x)^0,$$
    and we have a one-to-one correspondence 
    \begin{equation*}
        G^\vee\backslash\{(x,\phi)\ |\ x\in G^\vee,\phi\in\widehat{A_x}\}\longleftrightarrow\bigsqcup_{G'\in\InnT(G)}\Irr_{\un}(G'),\quad (x,\phi)\longmapsto\pi(x,\phi)
    \end{equation*}
\end{example}
\fi

\subsection{Unipotent conjugacy classes of complex simple groups}\label{subsec:unipotent_classes}

In the previous paragraph we stated the arithmetic-geometric correspondence (also known as the unramified local Langlands correspondence), which reduces the classification of unipotent representations of $G$ to the classification of conjugacy classes of $G^\vee$ and the structure of the component group of their centralizers. To understand these, we first need to study the classification of unipotent conjugacy classes of $G^\vee$, an interesting problem on its own right that uncovers rich geometric structure inside $G^\vee$. 

Define $\mathcal{U}$ to be the set of unipotent elements of $G^\vee$. This can be seen to be a closed irreducible subvariety of $G^\vee$ of dimension $\dim G^\vee-\rk G^\vee$. If $u\in \mathcal{U}$ is a unipotent element, its conjugacy class $C(u)\subset G^\vee$ is the orbit of $u$ under the conjugation action of $G^\vee$ on itself. Standard results in the structure theory of unipotent elements inside complex reductive groups state that $G^\vee$ has finitely many conjugacy classes of unipotent elements, and that each class $C$ is a locally closed subvariety of $G^\vee$. Moreover, its closure $\overline{C}$ is the union of (finitely many) unipotent conjugacy classes. In particular, there is one unique unipotent conjugacy class $C_{\reg}$ of maximal dimension such that $C_{\reg}$ is open and dense. Such unipotent elements are called \textit{regular}, and $\dim Z_{G^\vee}(u)=\rk G^\vee$ for any $u\in C_{\reg}$. The boundary of $C_{\reg}$ has dimension $\dim G^\vee-\rk G^\vee-2$ and contains a unique dense unipotent conjugacy class $C_{subreg}$ of \textit{subregular} unipotent elements such that 
$$\overline{C_{subreg}}=\overline{C_{\reg}}-C_{\reg}=\mathcal{U}-C_{\reg}.$$
Similarly, $\dim_{Z_{G^\vee}}(u)=\rk G^\vee+2$ for any $u\in C_{subreg}$. At the other end, there is the trivial class consisting of $\{1\}$, and this is the only closed conjugacy class. There is one further "canonical orbit", the set of \textit{minimal} unipotent elements $C_{\min}$, with the property that they are contained in the closure of every unipotent conjugacy class except for $\{1\}$.  

Beyond these four classes, the structure of $\mathcal{U}$ for a general simple complex group can be complicated. To study it, one can define a partial ordering on the set of unipotent conjugacy classes given by 
$$C\leq C'\quad\text{if and only if}\quad\overline{C}\subseteq\overline{C'}.$$
One can then picture this partial order in a diagram, called a \textit{Hasse diagram}, and one has the following generic picture.
\begin{figure}[!ht]
    \centering
    \includegraphics{Unipotent structure.png}
\end{figure}

\newpage
\subsection{The dual nonabelian Fourier transform}
At this stage, it is hopefully a natural question to ask whether there exists so linear involution $\FT^\vee:R_{\un}(G)\rightarrow R_{\un}(G)$ such that the square
\[
\begin{tikzcd}
R_{\un}(G) \arrow[r,dashed, "\FT^\vee"] \arrow[d, "\res_{\un}^{\mathrm{par}}"] & R_{\un}(G) \arrow[d, "\res_{\un}^{\mathrm{par}}"] \\
\bigoplus_K R_{\un}(\overline{K}) \arrow[r, "(\FT_{\overline{K}})_K"] & \bigoplus_K R_{\un}(\overline{K})
\end{tikzcd}
\]
commutes, where the sum is taken over all maximal parahoric subgroups $K$. This is an open question in general, and a very difficult one. A significant amount of work has been done trying to define such a map with nice properties. The first important observation is that while the above map is hoped to exist for simply connected groups $G$, representations of pure inner twists of $G$ should interact with those of $G$, so a modification is required. More concretely, we consider instead the commutative square
\[
\begin{tikzcd}
\bigoplus_{G'}R_{\un}(G') \arrow[r,dashed, "\FT^\vee"] \arrow[d, "\res_{\un}^{\mathrm{par}}"] & \bigoplus_{G'}R_{\un}(G') \arrow[d, "\res_{\un}^{\mathrm{par}}"] \\
\bigoplus_{G'}\bigoplus_{K'} R_{\un}(\overline{K'}) \arrow[r, "\FT^{\mathrm{par}}"] & \bigoplus_{G'}\bigoplus_{K'} R_{\un}(\overline{K'}),
\end{tikzcd}
\]
where the outer sum is taken over all pure inner twists of $G$ and the inner sum is taken over all maximal \textbf{open compact} subgroups of $G$, not necessarily connected. Their connected components at the identity are parahoric subgroups and they coincide with them when $G$ is simply connected. It is also important to highlight the fact that $\FT^{\mathrm{par}}$ is generalization of Lusztig's nonabelian transform for finite groups of Lie type. For example, if a maximal open compact subgroup $K$ of $G$ is disconnected, there is some other maximal open compact subgroup $K'$ of a pure inner twist $G'$ of $G$ in such a way that $\FT^{\mathrm{par}}$ acts on $R_{\un}(K)\oplus R_{\un}(K')$ but not on each individual term. We will not pursue this further since $G$ will always be simply connected in our examples. \textcolor{red}{cite the paper of Beth}

So far, there has not been a well-defined construction for $\FT^\vee$ on the space $\oplus_{G'}R_{\un}(G')$. However, such a construction for $\FT^\vee$ has been achieved on a certain subspace called the \textit{elliptic representation space} of unipotent representations of $G$, which we now describe. For each unipotent conjugacy class, consider some element $u\in G^\vee$ in that class and let $\Gamma_u$ be reductive part of $Z_{G^\vee}(u)$. The set of elliptic pairs associated to $u$ are defined as
$$\mathcal{Y}(\Gamma_u)_{\text{ell}}=\{(s,h)\ |\ s,h\in\Gamma_u \text{ semisimple, }sh=hs \text{ and }Z_{\Gamma_u}(s,h) \text{ is finite}\}.$$
For each class of elliptic pairs $(s,h)\in\Gamma_u\backslash\mathcal{Y}(\Gamma_u)_{\text{ell}}$ up to $\Gamma_u$-conjugacy, we define the virtual representation 
$$\Pi(u,s,h):=\sum_{\phi\in\widehat{A_{su}}}\phi(h)\pi(su,\phi).$$
\begin{definition}
    The elliptic unipotent representation space $\mathcal{R}_{\un,\text{ell}}^p(G)$ of $G$ is defined as the $\CC$-subspace of $\bigoplus_{G'}R_{\un}(G')$ spanned by the set $\{\Pi(u,s,h)\ |\ u\in G^\vee,\ (s,h)\in\Gamma_u\backslash\mathcal{Y}(\Gamma_u)_{\text{ell}}\}.$
\end{definition}

On the space $\mathcal{R}_{\un,\text{ell}}^p(G)$ we can define the \textit{dual} Fourier transform in a natural way.

\begin{definition}\label{def:ellFourier}
    The dual elliptic nonabelian Fourier transform is the linear map satisfying
    \begin{equation*}
        \mathrm{FT}^\vee_{\text{ell}}:\mathcal{R}_{\un,\text{ell}}^p(G)\longrightarrow\mathcal{R}_{\un,\text{ell}}^p(G)\quad \Pi(u,s,h)\longmapsto\Pi(u,h,s)\quad\text{for all}\quad (s,h)\in\Gamma_u\backslash\mathcal{Y}(\Gamma_u)_{\text{ell}},\ u\in G^\vee\text{ unipotent.}
    \end{equation*}
\end{definition}

\begin{example}
    Let $G=G_2(F)$. Its dual complex group $G_2(\CC)$ has $5$ unipotent classes, but only the regular class (labelled by $G_2$) with trivial component group, and the subregular one (labelled by $G_2(a_1)$) with component group isomorphic to $S_3$, have elliptic pairs. The only elliptic pair for the regular class $G_2$ is $(1,1)$, so
    $$\Pi(G_2,1,1)=\pi(G_2,\mathbf{1}).$$
    Instead, the subregular class $G_2(a_1)$ has $8$ elliptic pairs, giving
    \begin{align*}
            \Pi(G_2(a_1),1,1)&=\pi(G_2(a_1),\mathbf{1})+\pi(G_2(a_1),\varepsilon)+2\pi(G_2(a_1),\mathbf{r})\\
            \Pi(G_2(a_1),1,g_2)&=\pi(G_2(a_1),\mathbf{1})-\pi(G_2(a_1),\varepsilon)\\
            \Pi(G_2(a_1),1,g_3)&=\pi(G_2(a_1),\mathbf{1})+\pi(G_2(a_1),\varepsilon)-\pi(G_2(a_1),\mathbf{r})\\
            \Pi(G_2(a_1),g_2,1)&=\pi(G_2(a_1)g_2,\mathbf{1})+\pi(G_2(a_1)g_2,\varepsilon)\\
            \Pi(G_2(a_1),g_2,g_2)&=\pi(G_2(a_1)g_2,\mathbf{1})-\pi(G_2(a_1)g_2,\varepsilon)\\
            \Pi(G_2(a_1),g_3,1)&=\pi(G_2(a_1)g_3,\mathbf{1})+\pi(G_2(a_1)g_3,\theta)+\pi(G_2(a_1),\theta^2)\\
            \Pi(G_2(a_1),g_3,g_3)&=\pi(G_2(a_1)g_3,\mathbf{1})+\theta^2\pi(G_2(a_1)g_3,\theta)+\theta\pi(G_2(a_1)g_3,\theta^2)\\
            \Pi(G_2(a_1),g_3,g_3^{-1})&=\pi(G_2(a_1)g_3,\mathbf{1})+\theta\pi(G_2(a_1)g_3,\theta)+\theta^2\pi(G_2(a_1)g_3,\theta^2).
    \end{align*}

\end{example}

Definition \ref{def:ellFourier} allows us to state the main conjecture of this document. It was stated in the Introduction \textcolor{red}{relabel here}, but we restate it for completeness.

\begin{conjecture}
    Let $G$ be a simple $p$-adic group. Then the diagram 
    \[
        \begin{tikzcd}
        \mathcal{R}_{\un,\text{ell}}^p(G) \arrow[r, "\FT^\vee_{\text{ell}}"] \arrow[d, "\res_{\un}^{\mathrm{par}}"] & \mathcal{R}_{\un,\text{ell}}^p(G) \arrow[d, "\res_{\un}^{\mathrm{par}}"] \\
        \bigoplus_{G'}\bigoplus_{K'} R_{\un}(\overline{K'}) \arrow[r, "\FT^{\mathrm{par}}"] & \bigoplus_{G'}\bigoplus_{K'} R_{\un}(\overline{K'}),
        \end{tikzcd}
    \]
    commutes, up to certain some roots of unity.
\end{conjecture}
A simple yet important observation is that the conjecture can be verified one \textit{unipotent conjugacy class of $G^\vee$ at a time} since the virtual representations $\Pi(u,s,h)$ and the dual elliptic nonabelian Fourier transform preserve the unipotent part of the parametrization. 

By far, the hardest part in the proof of this conjecture is the explicit calculation of the parahoric restriction map $\res_{\un}^{\mathrm{par}}:\mathcal{R}_{\un,\text{ell}}^p(G)\longrightarrow \oplus_{G'}\oplus_{K'} R_{\un}(\overline{K'})$. The issue is that representations in $R_{\un}(G')$ are labelled in terms of geometric data, while representations in $R_{\un}(\overline{K'})$ are described using Lusztig's labels. The relation between them is deep and requires the generalized Springer correspondence. We devote the next chapter to explore this connection. 

\iffalse
\begin{example}
    Let $\mathbf{G}$ be a simple algebraic group of type $G_2$, and let $G=G(F)$. Then $G$ is both simply connected and adjoint so it has no pure inner twists other than itself. In addition, $G$ has three maximal parahoric subgroups of types $K_0$, $K_1$ and $K_2$, with reductive quotients of type $G_2$, $A_2$ and $A_1+\tilde{A_1}$, respectively. Thus, commutativity of the above square is equivalent to the commutativity of
    \[
        \begin{tikzcd}
        \mathcal{R}_{\un,\text{ell}}^p(G) \arrow[r, "\FT^\vee_{\text{ell}}"] \arrow[d, "\res_{\un}^{\mathrm{par}}"] & \mathcal{R}_{\un,\text{ell}}^p(G) \arrow[d, "\res_{\un}^{K_i}"] \\
        R_{\un}(\overline{K_i}) \arrow[r, "\FT^{K_i}"] & R_{\un}(\overline{K_i}),
        \end{tikzcd}
    \]
    for $i=0,1,2$. Moreover, if $u\in G^\vee$ is unipotent, then $\mathcal{Y}(\Gamma_u)$ is non-empty if and only if $u=u_{\text{reg}}$ is regular and $\Gamma_u=\{(1,1)\}$, or $u=G_2(a_1)$ is subregular and $\Gamma_u=\{(1,1),(1,g_2),(1,g_3),(g_2,1),(g_2,g_2),(g_3,1),(g_3,g_3),(g_3,g_3')\}$.
    Therefore, $\mathcal{R}_{\un,\text{ell}}^p(G)$ is $9$-dimensional, spanned by
    \begin{equation*}
        \begin{cases}
            \Pi(u_{\text{reg}},1,1)&=\pi(u_{\text{reg}},\mathbf{1})\\
            \Pi(G_2(a_1),1,1)&=\pi(G_2(a_1),\mathbf{1})+\pi(G_2(a_1),\varepsilon)+2\pi(G_2(a_1),\mathbf{r})\\
            \Pi(G_2(a_1),1,g_2)&=\pi(G_2(a_1),\mathbf{1})-\pi(G_2(a_1),\varepsilon)\\
            \Pi(G_2(a_1),1,g_3)&=\pi(G_2(a_1),\mathbf{1})+\pi(G_2(a_1),\varepsilon)-\pi(G_2(a_1),\mathbf{r})\\
            \Pi(G_2(a_1),g_2,1)&=\pi(G_2(a_1)g_2,\mathbf{1})+\pi(G_2(a_1)g_2,\varepsilon)\\
            \Pi(G_2(a_1),g_2,g_2)&=\pi(G_2(a_1)g_2,\mathbf{1})-\pi(G_2(a_1)g_2,\varepsilon)\\
            \Pi(G_2(a_1),g_3,1)&=\pi(G_2(a_1)g_3,\mathbf{1})+\pi(G_2(a_1)g_3,\theta)+\pi(G_2(a_1),\theta^2)\\
            \Pi(G_2(a_1),g_3,g_3)&=\pi(G_2(a_1)g_3,\mathbf{1})+\theta^2\pi(G_2(a_1)g_3,\theta)+\theta\pi(G_2(a_1)g_3,\theta^2)\\
            \Pi(G_2(a_1),g_3,g_3^{-1})&=\pi(G_2(a_1)g_3,\mathbf{1})+\theta\pi(G_2(a_1)g_3,\theta)+\theta^2\pi(G_2(a_1)g_3,\theta^2).
        \end{cases}
    \end{equation*}

    When $i=1,2$ and the finite group $\overline{K_i}$ is of type $A_2$ or $A_1+\tilde{A_1}$, then $\mathrm{FT}^{K_i}$ is the identity map, and therefore it is enough to show that 
    $$\res_{\un}^{K_i}(\pi(u,s,h))=\res_{\un}^{K_i}(\Pi(u,h,s))$$
    for all $\Pi(u,s,h)$ spanning $\mathcal{R}_{\un,\text{ell}}^p(G)$. This is obvious for all cases except for $\pi(u,s,h)=\pi(G_2(a_1),1,g_2),\pi(G_2(a_1),1,g_3)$.


\end{example}
\fi


\iffalse
\subsubsection{Enhanced Langlands parameters}

Now, $G^\vee$ sits within $G^\vee_{\mathrm{sc}}=(G_{\ad})^\vee\twoheadrightarrow G^\vee\twoheadrightarrow G_{\ad}^\vee$. We then denote $Z_{G^\vee}^1(\varphi)$ to be the inverse image of $Z_{G^\vee}(\varphi)$ under the isogeny $G^\vee_{\mathrm{sc}}\twoheadrightarrow G^\vee$. We denote by $A^1_\varphi$ to be the component group of $Z_{G^\vee}^1(\varphi)$, respectively. In some contexts, the following refinement of Langlands parameters are required.

\begin{definition}
    An enhanced Langlands parameter is a pair $(\varphi,\phi)$, where $\varphi:W_F'\rightarrow G^\vee$ is an $L$-parameter and $\phi$ is an irreducible representation of $A_\varphi^1$. 
\end{definition}

Similarly to $L$-parameters, an enhanced $L$-parameter $(\varphi,\phi)$ is determined by the triple $(u_\varphi,\varphi|_{W_F},\phi)$, and if it is also unipotent, then it is determined by the triple $(u_\varphi,s_\varphi,\phi)$. Moreover, if $x_\varphi=s_\varphi u_\varphi$, then $Z_{G^\vee}(\varphi)=Z_{G^\vee}(x_\varphi)$, and therefore there is a canonical bijection
\begin{align*}
    \Phi_{\mathrm{e,un}}(G^\vee)&\longleftrightarrow G^\vee\backslash\{(x,\phi)\ |\ x\in G^\vee, \phi\in\widehat{A_x^1}\},\\
    (\varphi,\phi)&\longmapsto (s_\varphi u_\varphi,\phi)
\end{align*}
where $A_x^1$ is the component group of $Z^1_{G^\vee}(x)$.

The LLC then predicts that enhanced Langlands parameters parametrize the unipotent representations of all \textit{inner twists} of $G$, a set denoted by $\InnT(G)$ and which is sometimes more natural to study. Explicitly, the LLC predicts there is a bijection between the sets
\begin{align*}
    \LLC_{\un}: G^\vee\backslash\{(x,\phi)\ |\ x\in G^\vee, \phi\in\widehat{A_x^1}\}\longleftrightarrow\Phi_{\mathrm{e,un}}(G^\vee)\longleftrightarrow&\bigsqcup_{G'\in\InnT(G)}\Irr_{\un}(G')\\
    (x,\phi)\hspace{1.5cm}\longmapsto \hspace{1.5cm}&\quad\pi(x,\phi),
\end{align*}
where $G'$ runs over the classes of inner twists of $G$. We say that a pair $(x,\phi)$ is $G'$-relevant if $\pi(x,\phi)\in\Irr_{\un}(G')$, and this property can be checked explicitly. Firstly, we note that there is a canonical bijection 
$$\InnT(G)\longleftrightarrow H^1(F,\mathrm{Inn}(\mathbf{G}^*))\longleftrightarrow\Irr(Z_{G^\vee_{sc}}),\quad G'\longmapsto\zeta_{G'}.$$
Similarly, for any pair $(x,\phi)$, the representation $\phi$ induces a character $\zeta_\phi$ on $Z_{G^\vee_{sc}}$. Then $\pi(x,\phi)$ is $G'$ relevant if and only if $\zeta_\phi=\zeta_{G'}$ and we denote the set of $G'$-relevant parameters by $\Phi_{e,\un}(G')$. 

It is then clear that 
$$\Phi_{e,\un}(G^\vee)=\bigsqcup_{G'\in\InnT(G)}\Phi_{e,\un}(G'),$$
and the LLC predicts that $\Phi_{e,\un}(G')$ parametrizes the set $\Irr_{\un}(G')$ for each $G'\in\InnT(G)$.
\fi


