\section{The p-adic group of type \texorpdfstring{$F_4$}{PDFstring}}

Let $G=F_4(F)$ be a $p$-adic of type $F_4$ with simple roots $\alpha_1,\alpha_2$ (short roots) and $\alpha_3,\alpha_4$ (long roots). Let $S_{\aff}=\{\alpha_1,\alpha_2,\alpha_3,\alpha_4,-\alpha_0\}$ be the set of affine simple roots where $\alpha_0$ is the highest root. 

Assuming Reeder's tables in [Re2000], we present a discrepancy with conjecture 1.2 in [ACR2023]. Since $G$ is both an adjoint and simply connected group, it has a unique inner twist and pure inner twist (itself) so 
$$\mathcal{R}_{\un,\text{ell}}^p(G)=\overline{R}_{\un}(G).$$
By Lemma 9.5 in [ACR2023], this space is equipped with the Euler-Poincare pairing $EP_G$ and the combinations
$$\{\Pi(u,s,h)\ |\ u\in F_4(\CC)\text{ unipotent, }(s,h)\in\Gamma_u\backslash\mathcal{Y}(\Gamma_u)\},$$
where $u$ ranges over representatives of conjugacy classes of $F_4(\CC)$, is an orthogonal basis. Since the non-abelian Fourier transform
$$\mathrm{FT}_{\text{ell}}^\vee:\mathcal{R}_{\un,\text{ell}}^p(G)\longrightarrow\mathcal{R}_{\un,\text{ell}}^p(G)$$ 
preserves the subspaces spanned by $\{\Pi(u,s,h)\ |\ (s,h)\in\Gamma_u\backslash\mathcal{Y}(\Gamma_u)\}$ where $u$ is fixed, Conjecture 1.2 can be verified one unipotent class at a time. One can check that this space has dimension $31$, and $21$ of these come from the unipotent class $F_4(a_3)$, whose centralizer has component group isomorphic to $S_4$. To find the discrepancy, we shall fix a unipotent element $u$ of $F_4(\CC)$ in this class. 

On the parahoric side, conjugacy classes of maximal open compact subgroups of $G$ coincide with conjugacy classes of maximal parahoric subgroups since $G$ is simply connected. Thus, there are five conjugacy classes of maximal compact subgroups $K_0,\ldots,K_4$ of types $F_4, B_4, A_3\times \tilde{A}_1, A_2\times\tilde{A}_2,A_1\times C_3$ respectively, where $K_i$ is the parahoric subgroup corresponding to the set of simple affine roots given by $S_{\aff}-\{\alpha_i\}$. In particular, since $G$ has no pure inner twists other than itself, $\mathcal{C}(G)_{\text{cpt},\un}=\oplus_{i=0}^4 R_{\un}(\overline{K}_i)$ and the non-abelian Fourier transform 
$$\mathrm{FT}_{\text{cpt},\un}:\mathcal{C}(G)_{\text{cpt},\un}\longrightarrow\mathcal{C}(G)_{\text{cpt},\un}$$
is simply the direct sum of Lusztig's non-abelian Fourier transform in each of the direct summands.

Using Reeder's tables, I have verified that the conjecture holds for the unipotent classes labelled $F_4, F_4(a_1)$ and  $F_4(a_2)$. The unipotent class labelled $B_3$ is not covered in Mark's tables. Finally, take a unipotent element $u$ in the conjugacy class labelled by $F_4(a_3)$ and consider the parahoric subgroup $K_1$ of type $B_4$. We shall show that, assuming the tables are correct, the parahoric restriction of $\Pi(u,1,1)$ with respect to $K_1$ is not $\text{FT}_{K_1}$ stable, contradicting the conjecture. As Mark, I label the irreducible representations of $S_4$ by partitions of $4$: $(4)$ is the trivial rep, $(22)$ is the two dimensional rep, $(31)$ and $(211)$ are three-dimensional and $(1111)$ is the sign representation. Thus, 
$$\Pi(u,1,1)=\pi(u,1,(4))+3\pi(u,1,(31))+3\pi(u,1,(211))+2\pi(u,1,(22))+\pi(u,1,(1111)).$$
The first four representations are Iwahori-spherical, while $\pi(u,1,(1111))$ is supercuspidal. In particular, its parahoric restriction to $K_1$ is zero. The irreducible unipotent representations in the principal series of a finite group of Lie type of type $B_4$ are labelled by bipartitions $(\alpha,\beta)$ such that $|\alpha|+|\beta|=4$. According to Mark's tables, one obtains the parahoric restrictions with respect to $K_1$:
\begin{align*}
    \pi(u,1,(4))&\longrightarrow [\cdot,1^4]+[\cdot,211]+2[1,1^3]+[1^3,1]+[1,21]+[11,11]+[11,2]+[2,11],\\
    \pi(u,1,(31))&\longrightarrow[\cdot,22]+[\cdot,31]+[1,21]+[2,2],\\
    \pi(u,1,(211))&\longrightarrow[\cdot,4],\\
    \pi(u,1,(22))&\longrightarrow[1,1^3]+[2,11].
\end{align*}

The unipotent representations of $\overline{K}_1$ form families of size $1$ or $4$. For families of size $1$ there is nothing to check. The families of size $4$ are
\begin{align*}
    &\{[1,1^3],[1^4,\cdot],[\cdot,211], \theta_1\},\\
    &\{[11,11],[1^3,1],[\cdot,22], \theta_2\},\\
    &\{[2,11],[211,\cdot],[\cdot,31], \theta_3\},\\
    &\{[2,2],[22,\cdot],[1,3], \theta_4\},\\
    &\{[3,1],[31,\cdot],[\cdot,4], \theta_5\}.
\end{align*}
where the first representation in each family is the unique special one. Also, $\theta_1,\ldots,\theta_5$ are the five representations coming from the the other series of unipotent representations of $\overline{K}_1$.
Lusztig's non-abelian transform acts on each of these families by the transformation
$$\begin{pmatrix}
    1/2 & 1/2 & 1/2 & 1/2 \\
    1/2 & 1/2 & -1/2 & -1/2 \\
    1/2 & -1/2 & 1/2 & -1/2 \\
    1/2 & -1/2 & -1/2 & 1/2 
\end{pmatrix}$$
If we examine carefully the expression for $\Pi(u,1,1)$ and the parahoric restrictions, we note that the representations (with multiplicity) from each of the five families are
\begin{align*}
    &4[1,1^3]+[\cdot,211],\\
    &[11,11]+[1^3,1]+3[\cdot,22],\\
    &3[2,11]+3[\cdot,31]\\
    &3[2,2],\\
    &3[\cdot,4].
\end{align*}

Only the third combination is $\FT_{K_1}$-stable. The other $4$ all fail to be stable. However, if we believe the conjecture to be true and therefore Mark's tables to contain typos, it is not too difficult to spot possible ad-hoc fixes. For example, if we replace $[\cdot,22]$ by $[22,\cdot]$ then the second and fourth lines would be stable. Similarly, if we replace $[\cdot,4]$ by $[\cdot,211]$, then the first and last lines would be stable. However, I doubt this is one of the mistakes because one calculates the parahoric restriction of $\pi(u,1,(211))$ by looking at the $\tilde{W}$-module $\varepsilon\otimes H(\mathcal{B}^u)^{(211)}$. The Springer correspondence gives that the action of $W$ on the top cohomology group corresponds to the principal series representation $\phi_{1,12}'$ of $F_4(\FF_q)$, so the top cohomology is one dimensional. Mark states that the restriction of $\pi(u,1,(211))$ is irreducible, which would imply that the remaining cohomology groups are all trivial. However, $[\cdot,211]$ is $3$-dimensional so it could not arise in the parahoric restriction of $\pi(u,1,(211))$. 

Another ad-hoc fix for this is to replace both $[\cdot,211]$ and $[\cdot,4]$ for $[1^4,\cdot]$. But I find it hard to believe this is the mistake. This fix would affect the parahoric restriction of $\pi(u,1,(4))$ which is a problem because using Mark's tables the conjecture holds for the elliptic pair $(1,g_3)$, so the mistake would propagate to the restriction of $\pi(u,g_3,\mathbf{1})$.
Indeed, the parahoric restriction of 
$$\Pi(u,1,g_3)=\pi(u,1,(4))-\pi(u,1,(22))+\pi(u,1,(1111))$$
is $\FT_{K_1}$-stable and coincides with the parahoric restriction of 
$$\Pi(u,g_3,1)=\pi(u,g_3,\mathbf{1})+\pi(u,g_3,\theta)+\pi(u,g_3,\theta^2).$$
This is a simple check since $\pi(u,1,(1111)),\pi(u,g_3,\theta)$ and $\pi(u,g_3,\theta^2)$ are all supercuspidal, so their parahoric restriction to $K_1$ is zero.



Finding discrepancies in the families of size $4$ for the remaining elliptic pairs is harder because one needs to calculate parahoric restrictions of the representations $\pi(u,g_2,\varepsilon\otimes\mathbf{1}), \pi(u,g_2',r)$ and $\pi(u,g_4,-1)$, which are non-Iwahori and not supercuspidal representations and lie in the block whose type is given by $(K_{\{\alpha_2,\alpha_3\}},\theta)$, where $K_{\alpha_2,\alpha_3}$ is the parahoric subgroup of type $B_2$ and $\theta$ is the unique cuspidal unipotent representation of its reductive quotient. I am currently learning how to do this since it is not covered in Mark's paper. However, it is easy to find discrepancies in the families of size $1$, since these are always fixed by $\FT_{K_1}$. For example, $\{[\cdot,1^4]\}$ is the Steinberg representation of $\overline{K}_8$ and forms a family by itself. The multiplicity of this representation of the parahoric restriction of $\Pi(u,1,h), h\in\{1,g_2,g_2',g_3,g_4\}$ with respect to $K_1$ is always equal to $1$ according to Mark's tables. By the theory of unrefined minimal $K$-types of Moy--Prasad (please correct me if I am wrong here), the parahoric restriction of the three representations above with respect to $K_1$ cannot contain principal series representations. Thus, we can read the multiplicity of $[\cdot,1^4]$ in the parahoric restriction of $\Pi(u,h,1)$ for $h\in\{1,g_2,g_2',g_3,g_4\}$, which equals $1,2,0,1$ and $0$, respectively. So the conjecture would not hold for the elliptic pairs $(1,g_2),(1,g_2')$ or $(1,g_4)$ simply by looking at the Steinberg representation.


If the parahoric restrictions can be programmed in GAP, then it should not be hard to check these tables, but I have not learned how to do this yet either. Of course, any comment about any of these problems is very welcome and appreciated!

\vspace{1.4cm}
\textbf{REFERENCES}

[Re2000] refers to Mark Reeder's paper "Formal degrees and $L$-packets of unipotent discrete series representations of exceptional $p$-adic groups".

[AMC2023] refers to Aubert-Ciubotaru-Romano paper "A nonabelian Fourier transform for tempered
unipotent representations".


\newpage


