\section{Unipotent representations for p-adic groups and the dual nonabelian Fourier transform}

Let $\mathbf{G}$ be a connected, almost simple, split algebraic group over some nonarchimedean local field $F$ with finite residue field $k_F$ and let $G=\mathbf{G}(F)$. Let $\Rep(G)$ be the category of admissible smooth complex representations of $G$ and let $\Irr(G)$ be the subset of the irreducible representations. We define $R(G)$ to be $\CC$-span of $\Irr(G)$ (i.e. the complexification of the Grothendieck group of the abelian category of smooth $G$-representations).

If $K$ is any parahoric subgroup (these are the stabilizers under the action of $G$ on the Bruhat-Tits building) with pro-unipotent radical $U_K$, the quotient $\overline{K}=K/U_K$ is naturally isomorphic to a reductive group over $\kappa_F$, and there is a \textit{restriction functor}
\begin{equation*}
    \res^K:R(G)\longrightarrow \CC[\overline{K}]^{\overline{K}},\quad V\longmapsto \text{(character of) }V^{U_K},\quad\text{for all }V\in\Irr(G),
\end{equation*}
where $\CC[H]$ is the $\CC$-vector space of $\CC$-valued functions on a group $H$ and $\CC[H]^H$ is the subspace of functions preserved under conjugation (i.e. class functions). This is well-defined since the representations are assumed to be admissible. 

\begin{remark}
    Suppose that $K$ is the stabilizers of a point $x$ in the Bruhat-Tits building in some apartment $\mathcal{A}(G,T,F)$ corresponding to some maximal torus $T$ of $G$ with roots $\Phi$ and root subgroups $\cX_\alpha$. By tracing the action of $G$ on the building, one can show that 
    $$K=\langle T(\cO),\cX_\alpha(\varpi^{\lceil-\langle\alpha,x\rangle\rceil}\cO)\ |\ \alpha\in\Phi\rangle\quad\text{and}\quad U_K=\langle T(1+\varpi\cO),\cX_\alpha(\varpi^{\lceil\epsilon-\langle\alpha,x\rangle\rceil})\ |\ \alpha\in\Phi\rangle$$
    for some sufficiently small $\epsilon>0$. It then follows that 
    $U_K$ is a pro-unipotent normal subgroup of $K$ and $\overline{K}=K/U_K$ is isomorphic to the $k_F$-points of a reductive group with maximal torus $T(k_F)$ and roots $\Phi_K=\{\alpha\in\Phi\ |\ \langle\alpha,x\rangle\in\ZZ\}$. Thus, one can apply the ideas of Deligne--Lusztig for the representations of $\overline{K}$.
\end{remark}

\begin{definition}
    An irreducible smooth admissible representation $(\pi,V)$ of $G$ is \textit{unipotent} if there exists some parahoric subgroup $K$ of $G$ such that $V^{U_K}$ contains a cuspidal unipotent representation of the finite group $\overline{K}$.
\end{definition}
We note that if we replace $G$ for $G^F$ and \textit{parahoric} by \textit{parabolic}, then we recover the definition of a unipotent representation in $G^F$. 

The following two results are basic to ensure that unipotent representations behave under the restriction functor above.
\begin{proposition}\label{prop:unipotent_padic}
    Let $(\pi,V)$ be an irreducible admissible representation of $G$. If there is some parahoric subgroup $K$ such that $V^{U_K}$ contains a non-cuspidal unipotent representation of $\overline{K}=K/U_K$, then $(\pi,V)$ is a unipotent representation of $G$.
    %\begin{enumerate}
    %    \item Conversely, if $(\pi,V)$ is a unipotent representation of $G$, then all irreducible $\overline{K}$-subrepresentations of $V^{U_K}$ are unipotent.
    %\end{enumerate}
\end{proposition}
\begin{proof}
    To prove the first claim, suppose that $V^{U_K}$ contains a unipotent irreducible representation $(\sigma,W)$ of $\overline{K}$, and therefore $\Hom_K(W,V^{U_K})\neq 0$. By Proposition \ref{prop:harishchandra}, there is some standard parabolic subgroup $\overline{P}=\overline{U_P}\cdot\overline{L_P}$ of $\overline{K}$ and cuspidal unipotent representation $(\tau,U)$ of $\overline{L_P}$ such that 
    $$\Hom_{K}(W,\Ind_{\overline{P}}^{\overline{K}}U)\neq 0,$$
    where we view the $\overline{K}$ representations as inflated $K$ representations, trivial on $U_K$. By the classification of parahoric subgroups in $G$, it follows that $\bar{P}=H/U_K$, where $H$ is another parahoric subgroup contained in $K$. Moreover, we have the inclusions $U_K\subseteq U_H\subseteq H\subseteq K$ and therefore $\overline{U_P}=U_H/U_K$ and $\overline{L_P}=H/U_H$. Since induction and inflation are commuting operations, it follows that the induced $\overline{K}$-representation $\Ind_{\overline{P}}^{\overline{K}}U$ is isomorphic to the $K$-representation $\Ind_H^K U$, where we view $U$ as the induced $H$-representation trivial on $U_H$, and hence $\Hom_K(W,\Ind_H^K U)\neq0$. Since $W$ is irreducible and $K$ is compact, it also follows that 
    $$\Hom_K(\Ind_H^K U,V^{U_K})=\Hom_H(U,V^{U_K})\neq 0.$$
    Since the representation $U$ is trivial on $U_H$, it follows that the image of $H$-equivariant map $U\to V^{U_K}$ lies inside $V^{U_H}$. Thus, 
    $$\Hom_H(U,V^{U_H})=\Hom_H(U,V^{U_K})\neq 0,$$
    and this concludes the proof.
\end{proof}

Conversely, we would like to show that for any irreducible unipotent representation $(\pi,V)$ of $G$, the irreducible $\overline{P}$-submodules of $V^{P^+}$ are all unipotent. %We first note that by the results of Harish-Chandra, for any such irreducible $\overline{P}$-submodules affording the character $\chi$, there is a parahoric subgroup $Q\subseteq P$ and cuspidal representation 
This is a direct corollary of the following theorem.

\begin{theorem}
    Let $(\pi,V_\pi)$ be an irreducible admissible representation of $G$. Suppose that there is some parahoric $P$ such that $V^{P^+}\neq 0$ contains an irreducible cuspidal representation $(\sigma,V_\sigma)$ of $\overline{P}$. Then, for any parahoric $Q$ such that $V^{Q^+}\neq 0$, any $\overline{Q}$-submodule of $V^{Q^+}$ contains a cuspidal representation $(\tau,V_\tau)$ of $\overline{L}$ for some parahoric subgroup $L\subseteq Q$. Moreover, there is some $g\in G$ such that $L=\prescript{g}{}{P}$ and $\tau=\prescript{g}{}{\sigma}$.
\end{theorem}

\begin{proof}
    From the proof of the previous result implies that any $\overline{Q}$-submodule of $V^{Q^+}$ contains a cuspidal representation $\tau$ of $\overline{L}$ for some parahoric subgroup $L\subseteq Q$. Thus, we may assume without loss of generality that $Q=L$ so that the irreducible $\overline{Q}$-submodule $(\tau,V_\tau)$ of $V^{Q^+}$ is cuspidal. To prove the second statement, we let $E_\tau:V_\pi\rightarrow V_\tau$ be a $Q$-equivariant projection. Then, for any $g\in G$, we have the linear map 
    $$\varphi_g=E_\tau\circ\pi(g^{-1}):V_\sigma\longrightarrow V_\tau,$$
    and for any $h\in P\cap gQg^{-1}$ and $v\in V_\sigma$, we have that 
    $$\varphi_g\circ\sigma(h)(v)=E_\tau(\pi(g^{-1}hg)\pi(g^{-1})v)=\tau(g^{-1}hg)\circ   E_\tau(\pi(g^{-1}))(v)=\tau(g^{-1}hg)\circ\varphi_g(v).$$
Since $\pi$ is an irreducible representation of $G$, there must be some $g\in G$ such that $\pi(g^{-1})V_\sigma\not\subseteq\ker(E_\tau)$, in which case $\varphi_g\neq 0$. The image of $P\cap gQg^{-1}$ inside $\overline{P}$ (resp $\overline{gQg^{-1}}$) is a parabolic subgroup $\overline{P}_{\prescript{g}{}{Q}}$ (resp. $\overline{\prescript{g}{}{Q}}_{P}$) 
\end{proof}

\textcolor{red}{Need to find proofs for this!!}

\begin{example}
    The Iwahori subgroup $I=\langle T(\cO),\cX_\alpha(\cO),\cX_\alpha(\varpi\cO)\ |\ \alpha\in\Phi^+\rangle$, with pro-unipotent radical $U_I=\langle T(1+\varpi\cO),\cX_\alpha(\cO),\cX_\alpha(\varpi\cO)\ |\ \alpha\in\Phi^+\rangle$ and reductive quotient $I/U_I=T(k_F)$ is, up to conjugacy, the unique minimal parahoric subgroup of $G$. Any smooth admissible representation $(\pi,V)$ of $G$ with an Iwahori fixed vector is unipotent. Indeed, $V^{U_I}$ contains the trivial representation of $T(k_F)$, its unique unipotent representation. 
\end{example}

Analogously to the construction of $R(G)$, we define $\Irr_{\textrm{un}}(G)$ to be the set of irreducible unipotent representations of $G$, and $R_{\textrm{un}}(G)$ to be its $\CC$-span. Lemma \ref{prop:unipotent_padic} implies that there is a well-defined \textit{restriction function}
\begin{equation*}
    \res_{\textrm{un}}^K:R_{\textrm{un}}(G)\longrightarrow \CC_{\textrm{un}}[\overline{K}]^{\overline{K}},\quad V\longmapsto \text{(character of) }V^{U_K},\quad\text{for all }V\in\Irr(G).
\end{equation*}
It is also convenient to consider simultaneously all such functions for all conjugacy classes of maximal parahoric subgroups, so we define $\res_{\textrm{un}}^{\textrm{par}}=(\res_{\textrm{un}}^K)_K$. In the previous section, we defined a nonabelian Fourier transform 
$$\FT^K:\CC_{\un}[\overline{K}]^{\overline{K}}\longrightarrow\CC_{\un}[\overline{K}]^{\overline{K}},$$
taking the unipotent characters to the almost characters of $\overline{K}$, having nice geometric properties. We also consider all these maps simultaneously for all conjugacy classes of maximal parahoric subgroups, which we denote as $\FT^{\mathrm{par}}=(\FT^K)_{K \text{ maximal}}$.

It is therefore a natural question to ask whether there exists some function $\FT^\vee:R_{\un}(G)\rightarrow R_{\un}(G)$ such that the square
\[
\begin{tikzcd}
R_{\un}(G) \arrow[r, "\FT^\vee"] \arrow[d, "\res_{\un}^{\mathrm{par}}"] & R_{\un}(G) \arrow[d, "\res_{\un}^{\mathrm{par}}"] \\
\bigoplus_K\CC_{\un}[\overline{K}]^{\overline{K}} \arrow[r, "\FT^{\mathrm{par}}"] & \bigoplus_K\CC_{\un}[\overline{K}]^{\overline{K}}
\end{tikzcd}
\]

\textcolor{red}{This question is now unresolved, but partial progress has been achieved.} To understand it, we first need to look at the Langlands parametrization of unipotent representations. 

\subsection{The Langlands parametrization of unipotent representations}
For this section, we introduce the Weil group $W_F$ of the field $F$ with inertia subgroup $I_F$. Moreover, we set $W_F':=W_F\times\SL_2(\CC)$. %In addition to these groups, Langlands introduced the $L$-group $\prescript{L}{}{G}:=G^\vee\rtimes W_F$ associated to the $p$-adic group $G$, where $G^\vee$ denotes the $\CC$-points of the dual group of $\mathbf{G}$
Under the assumption that $\mathbf{G}$ is a split group, we have the following important definition.
\begin{definition}
    A \textit{Langlands parameter} (or $L$-parameter) for $G$ is a continuous morphism $\varphi:W'_F\rightarrow G^\vee$, where $G^\vee$ denotes the $\CC$-points of the dual group of $\mathbf{G}$, and $\varphi((w,1))$ is semisimple for each $w\in W_F$.
\end{definition}


In its simplest form, the Local Langlands correspondence (LLC) conjectures the existence of a finite to one map between isomorphism classes of smooth admissible complex representations of $G$ and conjugacy classes of Langlands parameters of $G$ satisfying certain nice properties. In particular, the LLC conjectures that the unipotent representations of $G$ correspond to the following Langlands parameters.

\begin{definition}
    An $L$-parameter $\varphi:W_F\times\SL_2(\CC)\rightarrow G^\vee$ is called \textit{unipotent} if $\varphi(w,1)=1$ for any element $w$ of the inertia subgroup $I_F$ of $W_F$. Such parameters are sometimes called \textit{unramified} Langlands parameters.
\end{definition}
\begin{remark}
    For any $L$-parameter $\varphi:W_F'\rightarrow G^\vee$, define the elements $u_\varphi=\varphi(1,\left(\begin{smallmatrix}
        1 & 1\\
        0 & 1
    \end{smallmatrix}\right))$ and $s_\varphi=\varphi(\Frob,\Id)$ that commute. An application of the Jacobson--Morozov theorem implies that an $L$-parameter is determined by $u_\varphi$ and $\varphi|_{W_F}$ up to $G^\vee$-conjugacy. If the $L$-parameter is, in addition, unipotent, then $\varphi|_{W_F}$ is determined by $s_\varphi$. Thus, unipotent $L$-parameters are parametrized by $G^\vee$ conjugacy classes of pairs $(u,s)$ where $u\in G^\vee$ is unipotent, $s\in G^\vee$ is semisimple and they commute. But this is the same as conjugacy classes of elements of $G^\vee$ (by using the Jordan decomposition).
\end{remark}

However, this is a finite-to-one correspondence, so it does not parametrize unipotent representations, but rather $L$-packets of unipotent representations. To get a one to one correspondence, we need to introduce refinements of the $L$-parameters. Given an $L$-parameter $\varphi$, a natural object of interest is the centralizer $Z_{G^\vee}(\varphi)$ of the image of $\varphi$ inside $G^\vee$. Now, $G^\vee$ sits within $G^\vee_{\mathrm{sc}}=(G_{\ad})^\vee\twoheadrightarrow G^\vee\twoheadrightarrow G_{\ad}^\vee$. We then denote $Z_{G^\vee}^1(\varphi)$ to be the inverse image of $Z_{G^\vee}(\varphi)$ under the isogeny $G^\vee_{\mathrm{sc}}\twoheadrightarrow G^\vee$. We then denote by $A_\varphi$, $A^1_\varphi$ to be the component group of $Z_{G^\vee}(\varphi)$ and $Z_{G^\vee}^1(\varphi)$, respectively.

\begin{definition}
    An enhanced Langlands parameter is a pair $(\varphi,\phi)$, where $\varphi:W_F'\rightarrow G^\vee$ is an $L$-parameter and $\phi$ is an irreducible representation of $A_\varphi^1$. 
\end{definition}

Let us introduce some notation. We denote by $\Phi_{\un}(G^\vee)$ the set of unipotent $L$-parameters and $\Phi_{e,\un}(G^\vee)$ the set of all \textit{enhanced} unipotent $L$-parameters. Similarly to $L$-parameters, an enhanced $L$-parameter $(\varphi,\phi)$ is determined by the triple $(u_\varphi,\varphi|_{W_F},\phi)$, and if it is also unipotent, then it is determined by the triple $(u_\varphi,s_\varphi,\phi)$. Moreover, if $x_\varphi=s_\varphi u_\varphi$, then $Z_{G^\vee}(\varphi)=Z_{G^\vee}(x_\varphi)$, and therefore there is a canonical bijection
\begin{align*}
    \Phi_{\mathrm{e,un}}(G^\vee)&\longleftrightarrow G^\vee\backslash\{(x,\phi)\ |\ x\in G^\vee, \phi\in\widehat{A_x^1}\},\\
    (\varphi,\phi)&\longmapsto (s_\varphi u_\varphi,\phi)
\end{align*}
where $A_x^1$ is the component group of $Z^1_{G^\vee}(x)$.

Then the LLC predicts that there is a natural bijection
\begin{align*}
    \LLC_{\un}: G^\vee\backslash\{(x,\phi)\ |\ x\in G^\vee, \phi\in\widehat{A_x^1}\}\longleftrightarrow\Phi_{\mathrm{e,un}}(G^\vee)\longleftrightarrow&\bigsqcup_{G'\in\InnT(G)}\Irr_{\un}(G')\\
    (x,\phi)\hspace{1.5cm}\longmapsto \hspace{1.5cm}&\quad\pi(x,\phi),
\end{align*}
where $G'$ runs over the classes of inner twists of $G$. We say that a pair $(x,\phi)$ is $G'$-relevant if $\pi(x,\phi)\in\Irr_{\un}(G')$, and this property can be checked explicitly. Firstly, we note that there is a canonical bijection 
$$\InnT(G)\longleftrightarrow H^1(F,\mathrm{Inn}(\mathbf{G}^*))\longleftrightarrow\Irr(Z_{G^\vee_{sc}}),\quad G'\longmapsto\zeta_{G'}.$$
Similarly, for any pair $(x,\phi)$, the representation $\phi$ induces a character $\zeta_\phi$ on $Z_{G^\vee_{sc}}$. Then $\pi(x,\phi)$ is $G'$ relevant if and only if $\zeta_\phi=\zeta_{G'}$ and we denote the set of $G'$-relevant parameters by $\Phi_{e,\un}(G')$. 

It is then clear that 
$$\Phi_{e,\un}(G^\vee)=\bigsqcup_{G'\in\InnT(G)}\Phi_{e,\un}(G'),$$
and the LLC predicts that $\Phi_{e,\un}(G')$ parametrizes the set $\Irr_{\un}(G')$ for each $G'\in\InnT(G)$.

\subsubsection{Pure Langlands parameters}

One can also refine $L$-parameters by considering instead pairs $(\varphi,\phi)$, where $\varphi$ is an $L$-parameters and $\phi$ is an irreducible representations of $A_\varphi$. We then set 
$$\Phi_{\mathrm{e,un}}^p(G^\vee)=G^\vee\backslash\{(\varphi,\phi)\ |\ \varphi\text{ unipotent, }\phi\in\widehat{A_\varphi}\},$$
which is in natural bijection with the set 
$$G^\vee\backslash\{(x,\phi)\ |\ x\in G^\vee, \phi\in\widehat{A_x}\},$$
where $A_x$ is the component group of $Z_{G^\vee}(x)$.

In this setting the Local Langlands conjecture predicts a natural bijection
\begin{equation*}
    G^\vee\backslash\{(x,\phi)\ |\ x\in G^\vee,\phi\in\widehat{A_x}\}\longleftrightarrow\bigsqcup_{G'\in\InnT^p(G)}\Irr_{\un}(G'),\quad (x,\phi)\longmapsto\pi(x,\phi)
\end{equation*}
where $G'$ now runs over the classes of \textit{pure} inner twists of $G$. We distinguish between the distinct pure inner twists by looking at characters of $Z_{G^\vee}$.

\begin{example}
    If $\mathbf{G}$ is a simple split \textit{simply connected} algebraic group, then $H^1(F,\mathbf{G}^*)=1$ and therefore there is only one class of pure inner forms of $G$, namely $G$ itself. Correspondingly, $G^\vee=G^\vee_{\ad}$ and $Z_{G^\vee}$ is trivial. Therefore, the above discussion gives a bijection 
    \begin{equation*}
        \LLC_{\un}^p:G^\vee\backslash\{(x,\phi)\ |\ x\in G^\vee,\phi\in\widehat{A_x}\}\longleftrightarrow\Phi_{\mathrm{e,un}}^p(G^\vee)\longleftrightarrow\Irr_{\un}(G^*).
    \end{equation*}
\end{example}

\begin{example}
    If $\mathbf{G}$ is a simple split \textit{adjoint} algebraic group, then $H^1(F,\mathbf{G}^*)=H^1(F,\mathrm{Inn}(\mathbf{G}^*))$ so for each inner twist there is one unique pure inner twist. 
    Therefore, from the previous discussion, unipotent enhanced $L$-parameters are in bijection with the set
    $$G^\vee\backslash\{(x,\phi)\ |\ x\in G^\vee,\phi\in\widehat{A_x}\}, \quad\text{where}\quad A_x=Z_{G^\vee}(x)/Z_{G^\vee}(x)^0,$$
    and we have a one-to-one correspondence 
    \begin{equation*}
        G^\vee\backslash\{(x,\phi)\ |\ x\in G^\vee,\phi\in\widehat{A_x}\}\longleftrightarrow\bigsqcup_{G'\in\InnT(G)}\Irr_{\un}(G'),\quad (x,\phi)\longmapsto\pi(x,\phi)
    \end{equation*}
\end{example}

\subsection{The unipotent elliptic space and the dual nonabelian Fourier transform}
With the Langlands parametrization, it is then possible to define a dual fourier transform on a certain subspace of $\bigoplus_{G'\in\InnT^p(G)}R_{\un}(G')$, which we now describe. We first fix some unipotent element $u\in G^\vee$ up to $G^\vee$-conjugacy and we denote $\Gamma_u$ the reductive part of $Z_{G^\vee}(u)$. We then consider the space of elliptic pairs 
$$\mathcal{Y}(\Gamma_u)_{\text{ell}}=\{(s,h)\ |\ s,h\in\Gamma_u \text{ semisimple, }sh=hs \text{ and }Z_{G^\vee}(s,h) \text{ is finite}\}$$
up to $\Gamma_u$-conjugacy. Then for each $(s,h)\in\Gamma_u\backslash\mathcal{Y}(\Gamma_u)_{\text{ell}}$, we define the virtual representation 
$$\pi(u,s,h):=\sum_{\phi\in\widehat{A_{su}}}\phi(h)\pi(su,\phi).$$
\begin{definition}
    The elliptic unipotent representation space $\mathcal{R}_{\un,\text{ell}}^p(G)$ of $G$ is defined as the $\CC$-subspace of $\bigoplus_{G'\in\InnT^p(G)}R_{\un}(G')$ spanned by the set $\{\pi(u,s,h)\ |\ u\in G^\vee,\ (s,h)\in\Gamma_u\backslash\mathcal{Y}(\Gamma_u)_{\text{ell}}\}.$
\end{definition}

On the space $\mathcal{R}_{\un,\text{ell}}^p(G)$ we can define the \textit{dual} Fourier transform in a natural way.

\begin{definition}
    The dual elliptic nonabelian Fourier transform is the linear map satisfying
    \begin{equation*}
        \mathrm{FT}^\vee_{\text{ell}}:\mathcal{R}_{\un,\text{ell}}^p(G)\longrightarrow\mathcal{R}_{\un,\text{ell}}^p(G)\quad \pi(u,s,h)\longmapsto\pi(u,h,s)\quad\text{for all}\quad (s,h)\in\Gamma_u\backslash\mathcal{Y}(\Gamma_u)_{\text{ell}},\ u\in G^\vee\text{ unipotent.}
    \end{equation*}
\end{definition}

With this linear map, we then have the diagram 
\[
    \begin{tikzcd}
    \mathcal{R}_{\un,\text{ell}}^p(G) \arrow[r, "\FT^\vee_{\text{ell}}"] \arrow[d, "\res_{\un}^{\mathrm{par}}"] & \mathcal{R}_{\un,\text{ell}}^p(G) \arrow[d, "\res_{\un}^{\mathrm{par}}"] \\
    \bigoplus_{G'}\bigoplus_{K'}\CC_{\un}[\overline{K'}]^{\overline{K'}} \arrow[r, "\FT^{\mathrm{par}}"] & \bigoplus_{G'}\bigoplus_{K'}\CC_{\un}[\overline{K'}]^{\overline{K'}},
    \end{tikzcd}
\]
and a natural question to ask is whether this square commutes. We first show this is indeed the case when $\mathbf{G}$ is a simple algebraic group of type $G_2$.

\subsection{Type \texorpdfstring{$G_2$}{PDFstring}}


\begin{example}
    Let $\mathbf{G}$ be a simple algebraic group of type $G_2$, and let $G=G(F)$. Then $G$ is both simply connected and adjoint so it has no pure inner twists other than itself. In addition, $G$ has three maximal parahoric subgroups of types $K_0$, $K_1$ and $K_2$, with reductive quotients of type $G_2$, $A_2$ and $A_1+\tilde{A_1}$, respectively. Thus, commutativity of the above square is equivalent to the commutativity of
    \[
        \begin{tikzcd}
        \mathcal{R}_{\un,\text{ell}}^p(G) \arrow[r, "\FT^\vee_{\text{ell}}"] \arrow[d, "\res_{\un}^{\mathrm{par}}"] & \mathcal{R}_{\un,\text{ell}}^p(G) \arrow[d, "\res_{\un}^{K_i}"] \\
        \CC_{\un}[\overline{K_i}]^{\overline{K_i}} \arrow[r, "\FT^{K_i}"] & \CC_{\un}[\overline{K_i}]^{\overline{K_i}},
        \end{tikzcd}
    \]
    for $i=0,1,2$. Moreover, if $u\in G^\vee$ is unipotent, then $\mathcal{Y}(\Gamma_u)$ is non-empty if and only if $u=u_{\text{reg}}$ is regular and $\Gamma_u=\{(1,1)\}$, or $u=u_{sr}$ is subregular and $\Gamma_u=\{(1,1),(1,g_2),(1,g_3),(g_2,1),(g_2,g_2),(g_3,1),(g_3,g_3),(g_3,g_3')\}$.
    Therefore, $\mathcal{R}_{\un,\text{ell}}^p(G)$ is $9$-dimensional, spanned by
    \begin{equation*}
        \begin{cases}
            \pi(u_{\text{reg}},1,1)&=\pi(u_{\text{reg}},\mathbf{1})\\
            \pi(u_{sr},1,1)&=\pi(u_{sr},\mathbf{1})+\pi(u_{sr},\varepsilon)+2\pi(u_{sr},\mathbf{r})\\
            \pi(u_{sr},1,g_2)&=\pi(u_{sr},\mathbf{1})-\pi(u_{sr},\varepsilon)\\
            \pi(u_{sr},1,g_3)&=\pi(u_{sr},\mathbf{1})+\pi(u_{sr},\varepsilon)-\pi(u_{sr},\mathbf{r})\\
            \pi(u_{sr},g_2,1)&=\pi(u_{sr}g_2,\mathbf{1})+\pi(u_{sr}g_2,\varepsilon)\\
            \pi(u_{sr},g_2,g_2)&=\pi(u_{sr}g_2,\mathbf{1})-\pi(u_{sr}g_2,\varepsilon)\\
            \pi(u_{sr},g_3,1)&=\pi(u_{sr}g_3,\mathbf{1})+\pi(u_{sr}g_3,\theta)+\pi(u_{sr},\theta^2)\\
            \pi(u_{sr},g_3,g_3)&=\pi(u_{sr}g_3,\mathbf{1})+\theta^2\pi(u_{sr}g_3,\theta)+\theta\pi(u_{sr}g_3,\theta^2)\\
            \pi(u_{sr},g_3,g_3^{-1})&=\pi(u_{sr}g_3,\mathbf{1})+\theta\pi(u_{sr}g_3,\theta)+\theta^2\pi(u_{sr}g_3,\theta^2).
        \end{cases}
    \end{equation*}

    When $i=1,2$ and the finite group $\overline{K_i}$ is of type $A_2$ or $A_1+\tilde{A_1}$, then $\mathrm{FT}^{K_i}$ is the identity map, and therefore it is enough to show that 
    $$\res_{\un}^{K_i}(\pi(u,s,h))=\res_{\un}^{K_i}(\pi(u,h,s))$$
    for all $\pi(u,s,h)$ spanning $\mathcal{R}_{\un,\text{ell}}^p(G)$. This is obvious for all cases except for $\pi(u,s,h)=\pi(u_{sr},1,g_2),\pi(u_{sr},1,g_3)$


\end{example}