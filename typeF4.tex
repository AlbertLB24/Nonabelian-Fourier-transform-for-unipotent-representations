\section{The p-adic group of type \texorpdfstring{$F_4$}{PDFstring}}

In this chapter, we let $G=F_4(F)$ be a simple $p$-adic group of type $F_4$, whose extended Dynkin diagram is
\[
    0 \relbar 4 \relbar 3 \xRightarrow{\quad} 2 \relbar 1.
\]
Let us also fix a set of simple affine roots $S_{\aff}=\{\beta_1,\beta_2,\beta_3,\beta_4,\beta_0\}$, where $\beta_1,\beta_2$ are the short simple roots. Since the extended Dynkin diagram has no non-trivial automorphisms, the fundamental group $\Omega(F_4)$ is trivial, and therefore $G=F_4(F)$ is both a simply connected and adjoint $p$-adic group. This is an important fact that simple groups of type $F_4$ share with simple groups of type $G_2$ that simplifies many results and calculations. 

Since $G$ is simply connected, it has no pure inner twists other than itself. Moreover, the set of maximal open compact subgroups of $G$ agree with the set of maximal parahoric subgroups of $G$, and their conjugacy classes can be parametrized by proper maximal subsets $J\subsetneq S_{\aff}$. It is then easy to see that there are five maximal open compact subgroups $\{K_0,K_4,K_3,K_2,K_1\}$ of $G$ up to conjugacy, where $K_i$ corresponds to $J_i:=S_{\aff}\setminus\{\alpha_i\}$ for $0\leq i\leq 4$. The reductive quotients $\overline{K}_0,\overline{K}_4,\overline{K}_3,\overline{K}_2,\overline{K}_1$ are finite reductive groups of type $F_4, A_1\times C_3, A_2\times\tilde{A}_2, A_3\times\tilde{A}_1,B_4$, respectively. In particular,
\[
    \mathcal{C}(G)_{\cpt,\un}=\bigoplus_{i=0}^4 R_{\un}(\overline{K}_i)
\]
and $\FT^{\mathrm{par}}=(\FT^{K_i})_i$ is Lusztig's Fourier transform on $R_{\un}(\overline{K}_i)$ on each coordinate.

The aim of this chapter is to prove that Conjecture \ref{conj:main} holds for $G$. In other words, we prove that 

\begin{theorem}\label{thm:F4}
    Let $G$ be a simple p-adic group of type $F_4$. Then the diagram 
    \[
        \begin{tikzcd}
            \mathcal{R}_{\un,\text{ell}}^p(G) \arrow[r, "\FT^\vee_{\text{ell}}"] \arrow[d, "\res_{\un}^{\mathrm{par}}"] & \mathcal{R}_{\un,\text{ell}}^p(G) \arrow[d, "\res_{\un}^{\mathrm{par}}"] \\
            \bigoplus_{i=0}^5 R_{\un}(\overline{K}_i) \arrow[r, "(\FT^{K_i})_i"] & \bigoplus_{i=0}^5 R_{\un}(\overline{K}_i),
        \end{tikzcd}
    \]
    commutes.
\end{theorem}

Firstly, we need to compute the basis $\{\Pi(u,s,h)\ |\ u\in G^\vee, (s,h)\in\Gamma_u\backslash\mathcal{Y}(\Gamma_u)_{\mathrm{ell}}\}$ of the space $\mathcal{R}_{\un,\text{ell}}^p(G)$. Here $u$ is a unipotent element of $G^\vee=F_4(\CC)$, the complex simple reductive group of type $F_4$, and $\mathcal{Y}(\Gamma_u)$ is the set of elliptic pairs of $\Gamma_u$ (see Section \ref{subsec:unipotent_elliptic} for the Definitions). Thus, the first step is to classify all such pairs for all unipotent conjugacy classes of $F_4(\CC)$. Thankfully, this is a well-known classification given in Table \ref{tbl:F4_pairs}. In particular, $\mathcal{R}_{\un,\text{ell}}^p(G)$ is a $31$-dimensional vector space.

\begin{table}[ht]
    \centering
    \begin{tabular}{|c|c|c|c|c|c|}
        \hline
        Unipotent & $\Gamma_u^0$ & $A_u$ & $|\Gamma_u\backslash\mathcal(\Gamma_u)_{\text{ell}}|$ & $\mathcal{F}_un$ & $\Gamma_{\mathcal{F}_u}$ \\\hline
        $F_4$ & $1$ & $1$ & $1$ & $\phi_{1,24}$ & $1$ \\\hline
        $F_4(a_1)$ & $1$ & $S_2$ & $4$ & $\phi_{4,13}$ & $C_2$ \\\hline
        $F_4(a_2)$ & $1$ & $S_2$ & $4$ & $\phi_{9,10}$ & $1$ \\\hline
        $B_3$ & $\text{PGL}(2)$ & $1$ & $1$ & $\phi_{8,9}''$ & $1$ \\\hline
        $F_4(a_3)$ & $1$ & $S_4$ & $21$ & $\phi_{12,4}$ & $S_4$\\\hline
    \end{tabular}
    \caption{Elliptic pairs for $F_4$}
    \label{tbl:F4_pairs}
\end{table}

The main difficulty in proving Theorem \ref{thm:F4} lies in the explicit computation of the restriction map 
$$\res^{\mathrm{par}}_{\un}:\mathcal{R}_{\un,\mathrm{ell}}^p(G)\longrightarrow\bigoplus_{i=0}^5 R_{\un}(\overline{K}_i).$$

We note that the conjecture can be verified one unipotent class $u\in G^\vee$ and one parahoric subgroup $K_i$ at a time. Thus, the proof of Theorem \ref{thm:F4} will consist on a case-by-case analysis of the distinct options.

\subsection{Restriction to hyperspecial parahoric \texorpdfstring{$K_0\longrightarrow F_4$}{PDFstring}}

