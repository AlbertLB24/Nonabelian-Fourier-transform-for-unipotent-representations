\section{Unipotent representations for finite groups of Lie type}
%The statement of \textcolor{red}{Conjecture from introduction} is at heart a statement about the representation theory of a p-adic group $G$.

Despite the fact that the statement of \textcolor{red}{\ref{conj:main}} is at heart a result about the representation theory of a p-adic group $G$, in this first section we restrict our attention to structural and representation theoretic results of finite groups of Lie type. The aim is to understand Lusztig's non-abelian fourier transform $\FT^K:R_{\un}(\overline{K})\longrightarrow R_{\un}(\overline{K})$ on the space of unipotent representations of the reductive parahoric quotient $\overline{K}$ appearing at the bottom of the diagram in \ref{conj:main}. To achieve this, we first need to take a step back and look at general theory of finite groups of Lie type and the classification of their representations. Most of the material in this first section can be found in the book by \textcolor{red}{Carter}, a great reference on the subject. 

To introduce finite groups of Lie type, we first need to briefly look at Frobenius maps. Throughout this chapter, as it is common in the literature, $G$ will denote an algebraic group over an algebraically closed field of positive characteristic. In later sections, $G$ will always denote a $p$-adic group!
%In later sections, we will explain structural and representation theoretic results of $p$-adic groups to understand the vertical arrows


\subsection{Frobenius maps}

Let $k$ be an algebraically closed field of characteristic $p>0$ and let $G$ be a connected linear algebraic group over $k$, which we assume in addition to be reductive. A well-known fact states that $G$ is isomorphic to a closed subgroup of $\GL_n(k)$ for some $n$. The structure of $G$ can be understood to a large extent by looking at its maximal connected solvable subgroups of $G$, denoted as Borel subgroups. If we fix some Borel subgroup $B$, any maximal torus $T$ in $B$ is also a maximal torus in $G$, and it determines a set of roots $\Phi=\Phi(G,T)\subset X(T)$. The choice of the Borel $B$ containing $T$ corresponds to a choice of positive roots $\Phi^+$ and therefore of an integral basis $\Delta\subseteq\Phi^+$. Moreover, the subgroups $B,N:=N_G(T)$ satisfy the axioms of a $BN$-pair, as described by Tits, whose corresponding Weyl group is $W=N/T=\langle w_{\alpha_i}\ |\ \alpha_i\in\Delta\rangle$, generated by the reflections along the simple roots in $\Delta$.

For some power $q=p^e, e\geq 1$, consider the map $F_q:\GL_n(k)\rightarrow\GL_n(K),\ (a_{ij})\mapsto(a_{ij}^q)$, which is known to be a rational group homomorphism. A homomorphism $F:G\rightarrow G$ is called a \textit{standard Frobenius map} if there exists an embedding $i:G\hookrightarrow\GL_n(k)$ and some $q=p^e$ such that
$$i(F(g))=F_q(i(g))\quad\text{for all }g\in G.$$
More generally, a homomorphism $F:G\rightarrow G$ is a \textit{Frobenius map} if some power of $F$ is a standard Frobenius map. If $F:G\rightarrow G$ is a Frobenius map, we are interested in structure and representation theory of the \textbf{finite} group $G^F$ of fixed points under the Frobenius map. Many properties are inherited from $G$, but some change significantly.

One can show that $G$ contains $F$-stable Borel subgroups, and that inside any $F$-stable Borel there are $F$-stable maximal tori. Thus, we may assume that the Borel subgroup $B$ and maximal torus $T$ fixed in the previous paragraph are $F$-stable. Under these assumptions, the Frobenius map acts on the simple roots by permuting the corresponding the root spaces. Thus, $F$ corresponds to some permutation $\rho$ of $\Delta$ satisfying
$$F(\cX_\alpha)=\cX_{\rho(\alpha)}\quad\text{ for all }\alpha\in\Delta.$$
Moreover, one can easily check that $\rho$ is in fact a symmetry of the Dynkin diagram, and these can be completely classified. When $\rho$ is trivial we say that $G^F$ is a \textit{split} group and \textit{non-split} otherwise. In this document, we will mainly focus on split groups.  

For each orbit $J\subseteq\Delta$ of $\rho$, let $w_J\in W_J=\{w_{\alpha_i}\ |\ \alpha_i\in J\}$ be the unique element such that $w_J(J)=-J$. It is the longest element of $W_J$ and it satisfies that $w_J^2=1$. In then follows that the group $G^F$ has a natural $BN$-pair given by the groups $B^F$ and $N^F$, whose Weyl group is
$$N^F/T^F=(N/T)^F=W^F=\langle w_J\ |\ J\subseteq\Delta\text{ is an orbit of }\rho\rangle.$$
When $G^F$ is split, $\rho$ acts trivially and all orbits are singletons, so $W^F\cong W$.

\begin{example}
    Let $G=\GL_n(k)$ and let $F:G\rightarrow G$ be a Frobenius map. If $F=F_q$, then $\rho$ is trivial and $G^F=\GL_n(\FF_q)$. Alternatively, consider the map $\gamma:\GL_n(k)\rightarrow\GL_n(k)$ given by $A\mapsto Q_n^{-1}(A^T)^{-1}Q_n$, where $Q_n$ is the matrix with $1$ in the anti-diagonal and $0$ everywhere else. Then the map $F':=F_q\circ\gamma$ squares to $F_{q^2}$ so it is a Frobenius map. Then we obtain the non-split group $$\GL_n(k)^{F'}=\mathrm{GU}_n(\FF_q)=\{g\in\GL_n(\FF_{q^2})|\bar{A}^TQ_nA=Q_n\}.$$
\end{example}

It is a key structural fact of $G^F$ than any two $F$-stable Borel subgroups in $G$ are conjugate under $G^F$. However, it is not the case that any two $F$-stable maximal tori in $G$ are $G^F$-conjugate. A classification of these classes is a fundamental problem in the structure theory of $G^F$, with a beautiful solution. While it is true that any $F$-stable Borel subgroup contains an $F$-stable maximal torus, the converse might not be true. Any $F$-stable maximal torus that is contained in an $F$-stable Borel subgroup is called \textit{maximally split}, and it can be shown that any two $F$-stable maximally split tori are also conjugate under $G^F$.

Remarkably, one can easily determine the $G^F$-conjugacy classes of $F$-stable maximal tori by looking at the Weyl group. To state this result, we first introduce the notion of $F$-conjugacy classes in $W$. Given two $w_1,w_2\in W$, we say that they are $F$-conjugate if there is some $x\in W$ such that $F(x)w_1x^{-1}=w_2$. Note that if $F$ acts on $W$ trivially, then the $F$-conjugacy classes are the standard conjugacy classes.

\begin{proposition}
    There is a bijection between
    \begin{align*}
        \{G^F\text{-conjugacy classes of $F$-stable maximal tori}\}&\longrightarrow\{F\text{-conjugacy classes of }W\}\\
        T'=\prescript{g}{}{T}&\longmapsto \pi(g^{-1}F(g))
    \end{align*}
\end{proposition}

From now, we will write $T_1$ for a maximally split $F$-stable maximal torus and $T_w$ for any $F$-stable torus obtained from $T_1$ by conjugating by some element $g\in G$ such that $\pi(g^{-1}F(g))=w$. By the previous result, these objects are uniquely defined up to $G^F$-conjugation.

\subsection{Deligne--Lusztig characters and unipotent representations}

In their groundbreaking paper from 1976, Deligne and Lusztig attached to each pair $(T,\theta)$ of $F$-stable maximal torus $T$ and character $\theta$ of $T^F$, a virtual character $R_{T,\theta}$ of the group $G^F$. These virtual characters were constructed using the action of $G^F$ on certain $\ell$-adic cohomology groups associated to certain Deligne--Lusztig varieties. We shall not consider the explicit definition of the characters, but we will rather recall without proof some important properties. 

\begin{enumerate}
    \item If the pair $(T',\theta')$ is obtained from $(T,\theta)$ by conjugation on some element of $G^F$ in the natural way, then $R_{T,\theta}=R_{T',\theta'}$.
    \item If $T_1$ is a maximally split torus inside some $F$-stable Borel $B$, then $R_{T_1,\theta}=\theta_{B^F}^{G^F}$, where $\theta_{B^F}^{G^F}$ is the character of the parabolically induced representation $\Ind_{B^F}^{G^F}\theta$.
    %\item The value $R_{T,\theta}(u)$ is independent of $\theta$ if $u$ is unipotent. We write $Q_T(u)$ for this common value. The function $Q_T:\{u\in G^F\mathrm{ unipotent}\}\rightarrow\CC$ is called the \textit{Green function}.
    \item The orthogonality relations: For any two pairs $(T,\theta)$ and $(T',\theta')$, we have that 
    $$(R_{T,\theta},R_{T',\theta'})=|\{w\in W(T,T')^F\ |\ \prescript{w}{}{\theta'}=\theta\}|.$$ In particular, if $T,T'$ are not $G^F$-conjugate, then $(R_{T,\theta},R_{T',\theta'})=0$.
    \item We say that $(T,\theta)$ is in general position if $\Stab_{W(T)^F}(\theta)$ is trivial. Thus, if $(T,\theta)$ is in general position, the orthogonality relations imply that one of $\pm R_{T,\theta}$ is an irreducible character.
    \item For any $F$-stable maximal torus $T$,
    \begin{equation*}
        (R_{T,\theta},1)=\begin{cases}
            1 \text{ if } \theta=1\\
            0 \text{ if } \theta\neq1.
        \end{cases}
    \end{equation*}
    \item The dimension of $R_{T,\theta}$ equals
    $$\dim R_{T,\theta}=\varepsilon_G\varepsilon_T|G^F:T^F|, \quad \varepsilon_G\varepsilon_T\in\{\pm1\}.$$
\end{enumerate}

It is possible to define an action of $G$ on the pairs $(T,\theta)$ extending the action of $G^F$. It acts on the tori in the obvious way, but care is required to define the action on the characters $\theta$. We say that two pairs in the same $G$-orbit are \textit{geometrically conjugate}. For example, all $(T,\mathbf{1})$ are geometrically conjugate. Similarly, we say that two characters $\chi_1,\chi_2$ are \textit{geometrically conjugate} if there are geometrically conjugate pairs $(T,\theta)$ and $(T',\theta')$ such that 
$$(\chi_1,R_{T,\theta})\neq0\quad\text{and}\quad(\chi_2,R_{T',\theta'})\neq 0.$$

\begin{enumerate}[start=7]
    \item If $(T,\theta)$ and $(T',\theta')$ are not geometrically conjugate, then $R_{T,\theta}$ and $R_{T',\theta'}$ do not share any irreducible component in common.
    \item If $G$ has centre $Z$ and semisimple rank $l$, there are $|Z^F|q^l$ geometric conjugacy classes of pairs $(T,\theta)$.
\end{enumerate}

Let's give a couple of examples for the decomposition of the Deligne--Lusztig characters, where the properties above can be easily verified.

\begin{example}
    Suppose first that $G=\GL_2(k)$ and $F=F_q:G\rightarrow G$ is the standard Frobenius. Then $G^F=\GL_2(\FF_q)$ and $G$ has two $F$-stable tori up to $G^F$ conjugation, namely
    \begin{equation*}
        T_1=\left\{\begin{pmatrix}
            a & 0\\
            0 & b\\
        \end{pmatrix}\ |\ a,b\in k^*\right\}\quad\text{and}\quad T_w=\left\{\begin{pmatrix}
            a & b\\
            ub & a\\
        \end{pmatrix}\ |\ a,b\in k,\ a^2-ub^2\neq0\right\},
    \end{equation*}
    where $u\in\FF_q^\times$ is a non-square. 
    Then $T_1^F\cong \FF_q^\times\times\FF_q^\times$ while $T_2^F\cong\FF_{q^2}^\times$. Now, if $\theta=\theta_1\otimes\theta_2$ is a character of $T_1^F$, then 
    $$R_{T_1,\theta}={\theta_{B^F}}^{G^F}=\begin{cases}
        \overline{\theta}\otimes(1\oplus\St) &\text{ if } \theta_1=\theta_2\\
        \text{irreducible principal series} & \text{ if }\theta_1\neq\theta_2,
    \end{cases}$$
    where $\overline{\theta}$ is the unique extension of $\theta$ to all of $\GL_2(\FF_q)$ (this is only possible if $\theta_1=\theta_2)$.
    On the other hand, suppose that $\theta'$ is a character of $T_w^F$. Then
    $$R_{T_w,\theta'}=\begin{cases}
        \overline{\theta}\otimes(1\ominus\St) &\text{ if } \theta'^q=\theta'\\
        \text{irreducible cuspidal} & \text{ if }\theta'^q\neq\theta',
    \end{cases}$$
    where $\overline{\theta}$ is the extension of the unique character $\theta$ of $T_1^F$ for which $(\theta,T_1)$ is geometrically conjugate to $(\theta',T_w)$.
\end{example}

\begin{example}
    Now suppose that $G=\SL_2(k)$ and $F=F_q:G\rightarrow G$ to be the standard Frobenius again. Similarly, $G^F=\SL_2(\FF_q)$ and $G$ has two $F$-stable tori up to $G^F$ conjugation, namely
    \begin{equation*}
        T_1=\left\{\begin{pmatrix}
            a & 0\\
            0 & a^{-1}\\
        \end{pmatrix}\ |\ a\in k^*\right\}\quad\text{and}\quad T_w=\left\{\begin{pmatrix}
            a & b\\
            ub & a\\
        \end{pmatrix}\ |\ a,b\in k,\ a^2-ub^2=1\right\},
    \end{equation*}
    where $u\in\FF_q^\times$ is a non-square. Then $T_1^F\cong \FF_q^\times$ while $T_2^F\cong C_{q+1}$. If $\theta$ is a character of $T_1^F$ then
    $$R_{T_1,\theta}={\theta_{B^F}}^{G^F}=\begin{cases}
        1\oplus\St &\text{ if } \theta=1,\\
        R_+(\xi)\oplus R_-(\xi)&\text{ if }\theta=\xi=\mathrm{sgn},\\
        \text{irreducible principal series} & \text{ if }\theta\neq\theta^{-1},
    \end{cases}$$
    where $R_+(\xi)\neq R_-(\xi)$ are conjugate under $\GL_2(\FF_
    q)$ and so they have the same dimension $(q+1)/2$. On the other hand, if $\theta'$ is a character of $T_w^F$, then
    $$R_{T_w,\theta'}=\begin{cases}
        1\ominus\St &\text{ if } \theta'=1,\\
        \ominus R'_+(\xi)\ominus R'_-(\xi)&\text{ if }\theta'=\xi=\mathrm{sgn},\\
        \ominus\text{ irreducible cuspidal} & \text{ if }\theta'\neq\theta'^{-1},
    \end{cases}$$
\end{example}
\vspace{0.5cm}

Amongst all geometric conjugacy classes of characters of $G^F$, there is a distinguished class that will take most of our attention.

\begin{definition}
    An irreducible character $\chi$ of $G^F$ is called \textit{unipotent} if there is some maximal $F$-stable torus $T$ of $G$ such that $(R_{T,1},\chi)\neq0$. The set of unipotent characters of $G^F$ form a geometric conjugacy class of characters.
    %An irreducible character $\chi$ of the group $G^F$ is called \textit{semisimple} if 
    %\begin{equation*}
        %\sum_{\substack{u\in G^F\\u \text{ reg unipotent}}}\chi(u)\neq0.
    %\end{equation*}
\end{definition}

We remark that if $\chi$ is a unipotent character of $G^F$, then $(\chi,R_{T,\theta})=0$ for any $(T,\theta)$ with $\theta\neq1$. This is an immediate consequence of property $7.$ above. Lusztig realized that a classification of unipotent characters of $G^F$ and certain subgroups (also finite groups of Lie type) was enough to understand all irreducible characters. Firstly, he observed that the study of unipotent characters of $G^F$ can be reduced to the case when $G$ is simple of adjoint type. Secondly, following the same approach as Harish--Chandra, he noticed that it was enough to classify \textit{cuspidal} unipotent characters. There are many equivalent definitions for this notion; the idea is that $\chi$ is not obtained by parabolically inducing any unipotent character of a proper parabolic subgroup. The precise statement the following.

\begin{proposition}\label{prop:harishchandra}
    Let $\chi$ be an irreducible character of $G^F$. 
    \begin{enumerate}
        \item There is an $F$-stable parabolic subgroup $P$ of $G$ with $F$-stable Levi decomposition $P=LN$ and a cuspidal character $\phi$ of $L^F$ such that $(\chi,{\phi_{P^F}}^{G_F})\neq 0$.
        \item Moreover, the pair $(P,\phi)$ is unique up to $G^F$-conjugacy.
        \item The character $\chi$ of $G^F$ is unipotent if and only if $\phi$ is a unipotent character of $L^J$.
    \end{enumerate}     
\end{proposition}

Therefore, to classify the unipotent characters of $G^F$, it is enough to determine the cuspidal unipotent representations $\phi$ of the standard Levi subgroups $L_J^F$ of $G^F$ and then calculate the decomposition of ${\phi_{P_J^F}}^{G^F}$ into irreducible characters. The later task can be achieved by Howlett--Lehrer theory (Carter, \S10), while the former was achieved by Lusztig in a case by case analysis. For example, he proved that if $G^F$ has classical type, then it has either $0$ or $1$ cuspidal unipotent characters (if the type $A_n,n\geq1$, there are no cuspidal unipotent characters). In exceptional types, one can find more exotic behaviour. In type $G_2$ we find $4$ cuspidal unipotent characters and in $F_4$ we find $7$.

\iffalse
It is clear from the definitions that unipotent characters form one geometric conjugacy class of irreducible characters, and that if $\chi$ is a unipotent character, then $(\chi,R_{T,\theta})=0$ for any $\theta\neq 1$. Semisimple characters, on the other hand, have the opposite property. To explain this, we define the class function $\Xi$ to be supported on regular unipotent elements with constant value of $|Z^F|q^l$. By using properties of character duality, one can show that $(\Xi,\Xi)=|Z^F|q^l$ and that $(\Xi,\chi)\in\{-1,0,1\}$ for all irreducible characters $\chi$ of $G^F$. Note that this implies that there are exactly $|Z^F|q^l$ semisimple characters. In fact, one can furthermore show that 
\begin{equation*}
    \Xi=\sum_{\kappa}\varepsilon_\kappa\chi_\kappa^{ss} \quad\text{where }\chi_\kappa^{ss}\text{ is irreducible and }\quad \varepsilon_\kappa\chi_\kappa^{ss}=\sum_{\substack{(T,\theta)\in\kappa\\\text{mod }G^F}}\frac{R_{T,\theta}}{(R_{T,\theta},R_{T,\theta})},
\end{equation*}
where $\kappa$ runs over the conjugacy classes of pairs $(T,\theta)$. These results show that each geometric conjugacy class contains one unique semisimple irreducible character. 

\subsection{Jordan decomposition for irreducible characters}

Finally, we are ready to describe the \textit{Jordan decomposition for characters.} To simplify the discussion, we shall assume that the centre $(Z(G))^F$ of $G^F$ is connected. The idea is that one can completely understand all characters of a finite group of Lie type by understanding its semisimple representations and unipotent representations of its Levi subgroups. To state it, we first recall that there is a natural bijection between geometric conjugacy classes of $(T,\theta)$ and conjugacy classes of semisimple elements in the dual group $(G^*)^F$, both sets having size $|Z(G)^F|q^l$. 
\begin{definition}
    Let $(s)$ be a semisimple conjugacy class of the dual group $(G^*)^{F^*}$. Then the \textit{Lusztig series} $\mathcal{E}(G^F,(s))$ associated to $(s)$ is the set of irreducible characters of $G^F$ appearing in $R_{T,\theta}$ for some pair $(T,\theta)$ corresponding to $(s)$.
\end{definition}

The Lusztig series $\mathcal{E}(G^F,(s))$ are the geometric conjugacy classes of characters defined in the previous section. If $(s)$ is regular semisimple, then $(T,\theta)$ is in general position and $\mathcal{E}(G^F,(s))$ is a singleton. On the other end, the series $\mathcal{E}(G^F,(1))$ contains the unipotent characters.

\begin{theorem}
    Let $(s)$ be a semisimple conjugacy class of $(G^*)^F$ and let $H$ be the dual group of the centralizer $Z_{G^*}(s)$. Then there is a bijection 
    $$\mathcal{E}(G^F,(s))\rightarrow \mathcal{E}(H^F,(1)),\quad\chi\mapsto\chi_u,$$
    such that for any pair $(T,\theta)$ corresponding to $s\in(G^*)^{F^*}$ and any pair $(S,\psi)$ corresponding to $s\in(H^*)^{F^*}$,
    $$(\chi,\varepsilon_G\varepsilon_T\cdot R_{T,\theta}^G)=(\chi_u,\varepsilon_H\varepsilon_S\cdot R_S^H(\psi)).$$
    In addition, the unique semisimple character $\chi_s\in\mathcal{E}(G^F,(s))$ corresponds to the trivial character of $H^F$ and for any $\chi\in\mathcal{E}(G^F,(s))$, we have that 
    $$\chi(1)=\chi_s(1)\chi_u(1).$$
\end{theorem}

To summarize, for any irreducible character $\chi$, there is one unique semisimple character $\chi_s$ geometrically conjugate to $\chi$, corresponding to some semisimple conjugacy class of $(G^*)^{F^*}$. One can in fact show that 
$$\chi_s(1)=|(G^*)^{F^*}:C^{F^*}|_{p'},$$
where $C$ is the centralizer of $s^*$ in $G^*$.
Finally, there is a natural bijection $\chi\mapsto\chi_u$ between characters in the class containing $\chi_s$ and unipotent characters of the dual group of $C^{F^*}$ satisfying
$$\chi(1)=\chi_s(1)\chi_u(1).$$

As it turns out, studying semisimple characters is easy since we have explicit formulas to understand them. So Lusztig turned his attention into understanding unipotent representations of finite groups of Lie type. Lusztig first observed that the study of unipotent characters of $G^F$ can be reduced to the case when $G$ is simple of adjoint type. That's because every unipotent character appears as a component of some $R_{T,1}$, where $R_{T,1}(g)=\mathcal{L}(g,\mathfrak{B}_w)$. But $Z^F$ acts trivially on $\mathfrak{B}_w$, so it lies in the kernel of every unipotent representation. So $G$ can be assumed to be semisimple, and a simple argument shows that $G$ can be further assumed to be simple.


Secondly, following the same approach as Harish--Chandra to classify irreducible characters of $G^F$, Lusztig showed the following result.

\begin{proposition}
    Let $\chi$ be an irreducible character of $G^F$. 
    \begin{enumerate}
        \item There is an $F$-stable parabolic subgroup $P$ of $G$ with $F$-stable Levi decomposition $P=LN$ and a cuspidal character $\phi$ of $L^F$ such that $(\chi,{\phi_{P^F}}^{G_F})\neq 0$.
        \item Moreover, the pair $(P,\phi)$ is unique up to $G^F$-conjugacy.
        \item The character $\chi$ of $G^F$ is unipotent if and only if $\phi$ is a unipotent character of $L^J$.
    \end{enumerate}     
\end{proposition}

\begin{proof}
    The proof of parts $1.$ and $2.$ are classical, so we only give a sketch. We fix some $F$-stable maximal torus $T$ and some integral basis $\Delta\subset\Phi(G,T)$. For each $J\subseteq\Delta$, let $P_J=L_JU_J$ be the standard $F$-stable parabolic with standard $F$-stable Levi $L_J$. Since any parabolic subgroup of $G^F$ is conjugate to some $P_J^F$, it is enough to prove the assertions for standard parabolics of $G^F$.
    
    Let $V$ be the $G^F$ representation affording $\chi$ and let $\mathcal{J}=\{J\subseteq\Delta\ |\ (1_{U_J},\chi|_{U_J})\neq 0\}=\{J\subseteq\Delta\ |\ V^{U_J}\neq 0\},$ which is non-empty since $\Delta\in\mathcal{J}$. If $J\in\mathcal{J}$ is minimal with respect to inclusion, we may write $V^{U_J}=\oplus_{i=1}^k U_i$ as a direct sum of irreducible $L_J$-representations, all of which are cuspidal. The character $\phi$ afforded by $U_1$ satisfies the conditions of $1.$, and part $2.$ is contained in Carter 9.1.5.

    To prove the last assertion, fix some $J\subseteq\Delta$ and some irreducible character $\phi$ of $L_J$. Let $(T,\theta)$ be such that $(\phi,R_{T,\theta}^{L_J^F})_{L_J^F}\neq 0$, and let $\chi$ be an irreducible component of ${\phi_{P_J^F}}^{G^F}$. Then by Frobenius reciprocity, we have that 
    $$(\chi|_{P_J^F},\phi_{P_J^F})_{P_J^F}=(\chi,{\phi_{P_J^F}}^{G^F})_{G_F}\neq 0,$$
    and since $\phi_{P_J^F}$ is an irreducible $P_J^F$ representation, we have that
    $$(\chi,R_{T,\theta}^{G^F})=(\chi,{(R_{T,\theta}^{L_J^F})_{P_J^F}}^{G^F})_{G^F}=(\chi|_{P_J^F},{(R_{T,\theta}^{L_J^F})_{P_J^F}})_{P_J^F}\neq 0.$$

    This calculation, together with the fact that unipotent representations form a geometric conjugacy class yield the last part.
\end{proof}

Therefore, to classify the unipotent characters of $G^F$, it is enough to determine the cuspidal unipotent representations $\phi$ of the standard Levi subgroups $L_J^F$ of $G^F$ and then calculate the decomposition of ${\phi_{P_J^F}}^{G^F}$ into irreducible characters. The later task can be achieved by Howlett--Lehrer theory (Carter, \S10), while the former was achieved by Lusztig by a case by case analysis. For example, Lusztig showed that if $G^F$ is of classical type, then the number of cuspidal unipotent characters is either $0$ or $1$.

\fi
\subsection{Families of unipotent characters}\label{subsec:unipotent_families}
Lusztig further observed that the unipotent characters of $G^F$ naturally form families in a remarkable way. Firstly, he parametrized the principal series characters with irreducible characters of $W$ by showing that there is a natural bijection
\begin{align}\label{eqn:bijection_charsW}
    \{\text{Irreducible characters of }W\}&\longrightarrow\{\text{Irreducible components of }\Ind_{B^F}^{G^F}1\}\\
    \phi&\longmapsto\chi_\phi.
\end{align}
This fact can be elegantly seen as follows. The choice of Borel subgroup $B$ containing the torus determines a set of simple roots, and thus a set of simple reflections $S$ generating $W$. Now consider the endomorphism algebra $\cH(T^F,1):=\End(\Ind_{B^F}^{G^F}1)$. Using some Mackey theory, one can show that $\cH(T^F,1)$ has natural basis $\{T_w,w\in W\}$ satisfying the multiplication rules
\begin{align*}
    T_s^2=(q-1)T_s+qT_1 & \quad\text{ if }s\in S,\\
    T_{w_1}T_{w_2}=T_{w_1w_2} & \quad\text{ if }l(w_1w_2)=l(w_1)+l(w_2).
\end{align*}
Thus, $\cH(T^F,1)$ is isomorphic to the Coxeter algebra $\cH(W,S,q)$ with constant parameter $q$. This algebra is a deformation of the group algebra $\CC[W]$ (by setting $q=1$). Using Tit's deformation theorem, it follows that 
$$\cH(T^F,1)\cong\CC[W],$$
from which the above bijection can be easily deduced. We can also deduce the nice formula
$$(\chi_\phi,\Ind_{B^F}^{G^F}1)=\dim\phi.$$

\iffalse
\newpage

To prove this, we first note that
\begin{align*}
    \End(\Ind_{B^F}^{G^F}1)\cong \Hom_{B^F}(1,\Ind_{B^F}^{G^F}1|_{B^F})\cong\bigoplus_{w\in B\backslash G/B}\Hom_{B^F\cap \prescript{w}{}{B^F}}(1,1),
\end{align*}
where in the first step we have applied Frobenius reciprocity and the Mackey decomposition formula for the second one. By the Bruhat decomposition, $W$ is canonically isomorphic to $B\backslash G/B$ and the borel subgroup $B$ gives a natural choice of simple roots, and therefore of simple reflections $S\subset W$.

If we let $T_w\in\End(\Ind_{B^F}^{G^F}1)$ be the image of the identity map in $\Hom_{B^F\cap \prescript{w}{}{B^F}}(1,1)$, then one can prove that $\{T_w:w\in W\}$ is a basis for $\End(\Ind_{B^F}^{G^F}1)$ satisfying 
\begin{align*}
    T_s^2=(q-1)T_s+qT_1 & \quad\text{ if }s\in S,\\
    T_{w_1}T_{w_2}=T_{w_1w_2} & \quad\text{ if }l(w_1w_2)=l(w_1)+l(w_2).
\end{align*}
And therefore, $\End(\Ind_{B^F}^{G^F}1)$ is isomorphic to the coxeter algebra $\cH(W,S,q)$ of the pair $(W,S)$ with constant parameter $q$. It is possible to do a change of variables that give the important isomorphism
$$\End(\Ind_{B^F}^{G^F}1)\cong\CC[W].$$
The algebra $\CC[W]$ acts on itself by left multiplication, and the irreducible submodules are precisely the irreducible representations of $W$. By the isomorphism above, $\End(\Ind_{B^F}^{G^F}1)$ also decomposes into a direct sum of irreducible submodules under post composition. If $\Ind_{B^F}^{G^F}1=\oplus_{i=1}^kV_i^{a_i}$ is a direct sum into $G^F$ irreducible components, then 
$$\End(\Ind_{B^F}^{G^F}1)=\bigoplus_{i=1}^k\Hom_{G^F}(V_i,\oplus_{j=1}^kV_j^{a_j})^{a_i},$$
and the modules on the right hand side are precisely the irreducible submodules, each one corresponding to one unique irreducible component of $\Ind_{B^F}^{G^F}1$. Thus, irreducible characters of $W$ parametrize principal series cuspidal characters of $G^F$.
\fi

\begin{example}\label{ex:G2_reps}
    Let $G=G_2(k)$ and let $F=F_q:G\to G$ be the standard Frobenius. Then $G^F=G_2(\FF_q)$, whose Weyl group $W$ is isomorphic to $D_{12}$. Following Carter, we label the six irreducible representations by $\phi_{1,0},\phi_{1,3}',\phi_{1,3}'',\phi_{1,6},\phi_{2,1},\phi_{2,2}$, where the first subindex gives the dimension and, for example, $\phi_{1,0}=1$ and $\phi_{1,6}=\det$. Moreover, the Weyl group has $6$ conjugacy classes often labelled by $e,a,b,ab,(ab)^2$ and $(ab)^3$, where $a$ and $b$ are short and long reflections, respectively. Then
    \begin{align*}
        R_{T_1,1}&=\phi_{1,0}+\phi_{1,6}+\phi_{1,3}'+\phi_{1,3}''+2\phi_{2,1}+2\phi_{2,2},& \\
        R_{T_a,1}&=\phi_{1,0}-\phi_{1,6}+\phi_{1,3}'-\phi_{1,3}'',& \\
        R_{T_b,1}&=\phi_{1,0}-\phi_{1,6}-\phi_{1,3}'+\phi_{1,3}'',& \\
        R_{T_{ab},1}&=\phi_{1,0}+\phi_{1,6}-\phi_{2,1}&+G_2[-1]+G_2[\theta]+G_2[\theta^2],\\
        R_{T_{(ab)^2},1}&=\phi_{1,0}+\phi_{1,6}-\phi_{2,2}&+G_2[1]-G_2[\theta]-G_2[\theta^2],\\
        R_{T_{(ab)^3},1}&=\phi_{1,0}+\phi_{1,6}-\phi_{1,3}'-\phi_{1,3}''&-2G_2[1]-2G_2[-1],
    \end{align*}
    where $G_2[1],G_2[-1],G_2[\theta],G_2[\theta^2]$ are the commonly used labels for the $4$ cuspidal unipotent characters.
\end{example}

For type $A_l$-groups, one can explicitly describe the representations $\chi_\phi$ in terms of Deligne--Lusztig characters.

\begin{example}
    Let $G$ be a reductive group of type $A_l$.
    No standard Levi subgroup $L^F$ of $G^F$ has a cuspidal unipotent representation other than the maximally split torus since they all have type $A_m,m\leq l$.
    Consequently, all unipotent representations of $G^F$ are in the principal series. This means that irreducible characters of $W$  parametrize unipotent characters of $G^F$. Explicitly, given some irreducible character $\phi$ of $W$, 
    \begin{equation*}
        \chi_\phi=\frac{1}{|W|}\sum_{w\in W}\phi(w)R_{T_w,1}
    \end{equation*}
    For $G=\GL_2(k)$ or $\SL_2(k)$, $\chi_1=1$ and $\chi_{\mathrm{sgn}}=\St$.
\end{example}

In general, however, finite groups of Lie type do have cuspidal unipotent characters, and the virtual characters 
$$R_\phi=\frac{1}{|W|}\sum_{w\in W}\phi(w)R_{T_w,1}$$
as defined above are not irreducible, and are denoted \textit{unipotent almost characters}. Lusztig then divided irreducible characters of $W$ into equivalence classes (called \textit{families}) by the rule that two characters $\phi_1$ and $\phi_2$ are equivalent if $R_{\phi_1}$ and $R_{\phi_2}$ share an irreducible component. Similarly, he extended this notion of families to unipotent characters of $G^F$ by the rule that two unipotent characters appearing in the same $R_\phi$ are in the same family and then extending by transitivity. Tracing through the definitions, it is not hard to show that there is a bijection between families of characters of $W$ and families of unipotent characters of $G^F$. Remarkably, Lusztig proved the latter families can be parametrized in the following manner. 

\begin{theorem}
    For each family of unipotent representations $\mathcal{F}$ of $G^F$ there is a group $\Gamma
    _\mathcal{F}\in\{1,C_2\times\cdots\times C_2,S_3,S_4,S_5\}$
    and a bijection 
    \begin{align*}
        M(\Gamma_\mathcal{F})&\longrightarrow\mathcal{F}\\
        (x,\sigma)&\longmapsto\chi_{(x,\sigma)}^\mathcal{F}
    \end{align*}
    satisfying 
    \begin{equation*}
        (\chi_{(x,\sigma)}^\mathcal{F},R_\phi)=\begin{cases}
            \{(x,\sigma),(y,\tau)\} & \text{ if } \chi_\phi=\chi_{(y,\tau)}^\mathcal{F}\in\mathcal{F},\\
            0 & \text{ if }\chi_\phi\not\in\mathcal{F},
        \end{cases}
    \end{equation*}
    where
    \[
        \{(x,\sigma),(y,\tau)\}=\frac{1}{|C_\Gamma(x)||C_\Gamma(y)|}\sum_{\substack{g\in \Gamma\\ xgyg^{-1}=gyg^{-1}x}}\sigma(gyg^{-1})\overline{\tau(g^{-1}xg)}.
    \]
    
\end{theorem}
Since $R_{T_1,1}=\sum_{\phi\in\hat{W}}R_\phi$, it follows that for any family $\mathcal{F}$, $(\chi_{(1,1)}^\mathcal{F},R_{T_1,1})>0$, so $\chi_{(1,1)}^\mathcal{F}=\chi_\phi$ for some character $\phi$ of $W$. Characters arising this way are called \textit{special characters} of $W$ and they have distinct characterizations. They are the distinguished elements of the families of characters of $W$ as described above. The upshot of this discussion is that families of unipotent characters can be parametrized by special characters of the Weyl group. 

\begin{example}
    Consider again the group $G^F=G_2(\FF_q)$. Following the notation in Example \ref{ex:G2_reps}, the special characters of $W\cong D_{12}$ are $\phi_{1,0},\phi_{1,6},\phi_{2,1}$ whose corresponding families are
    $$(\phi_{1,0}),(\phi_{2,1},\phi_{2,2},\phi_{1,3}',\phi_{1,3}''),(\phi_{1,6}).$$
    On the other hand, the $10$ unipotent characters of $G^F$ fall into three families, parametrized as follows. 
    \begin{figure}[ht]
        \centering
        \includegraphics[width=0.7\textwidth]{cuspidals G_2.png}
        \caption{Unipotent characters of $G_2(\FF_q)$ and its family labels.}
        \label{fig:G2_labels}
    \end{figure}
\end{example}

Finally, Lusztig used these families to define a \textit{nonabelian Fourier transform} $\FT_{G^F}$ on the set of irreducible characters of $G^F$. This is a linear map on complex vector space $R_{\un}(G^F)$ defined as the $\CC$-span of the irreducible unipotent characters of $G^F$. The families described above induce a direct sum decomposition 
$$R_{\un}(G^F)=\bigoplus_\mathcal{F}R_{\un,\mathcal{F}}(G^F),$$
where $R_{\un,\mathcal{F}}(G^F)$ is the $\CC$-span of the unipotent representations of $G^F$ in the family $\mathcal{F}$. Lusztig's nonabelian Fourier transform acts on each block as follows. For a family $\mathcal{F}$ with associated group $\Gamma_\mathcal{F}$, consider the $|M(\Gamma_\mathcal{F})|\times|M(\Gamma_\mathcal{F})|$ matrix $\mathcal{M}(\mathcal{F})$ whose $((x,\sigma),(y,\tau))$ entry is the value $\{(x,\sigma),(y,\tau)\}$. This matrix is Hermitian and squares to the identity. The map $\FT_{G^F}:R_{\un}(G^F)\rightarrow R_{\un}(G^F)$ then acts on each block $R_{\un,\mathcal{F}}(G^F)$ by the matrix $\mathcal{M}(\mathcal{F})$. Since all the matrices square to the identity, $\FT_{G^F}$ is an involution. 

In particular, Lusztig's nonabelian transform satisfies $\FT_{G^F}(\chi_\phi)=R_\phi$, so it maps the unipotent characters of $G^F$ to almost unipotent characters. The latter ones have very nice geometric properties, which were exploited by Lusztig to prove other remarkable results. We will not pursue this further here, but we will see an explicit example for type $G_2$. %By this we mean that every almost character agrees up to a scalar with a characteristic function associated to an $F$-stable character sheaf on $G^F$ (see Shoji's article).

%\begin{example}
%    If $G^F$ is of type $A_l$, then the unipotent characters coincide with the almost characters.
%\end{example}
\begin{example}
    If $G^F$ is of type $G_2$, then $\FT_{G^F}$ fixes the characters $\phi_{1,0}$ and $\phi_{1,6}$ but transforms the third family according to the Fourier transform matrix
    \begin{figure}[ht]
        \centering
        \includegraphics[width=0.6\textwidth]{Fourier matrix for S_3.png}
        \caption{Fourier matrix when $\Gamma_\mathcal{F}\cong S_3$.}
        \label{fig:S3matrix}
    \end{figure}
\end{example}
%The almost characters satisfy certain \textit{stability properties}. \textcolor{red}{Search what do they exactly mean by this.}
%The aim of the next chapter is to discuss a lift of this map for $p$-adic groups.