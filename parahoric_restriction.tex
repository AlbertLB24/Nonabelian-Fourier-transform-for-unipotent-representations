\section{Parahoric restriction of unipotent representations}

Let $G$ be a connected split simple group \textit{of adjoint type} over a $p$-adic field $F$. In the previous section we stated the bijection
\begin{align*}
    \LLC^p_{\un}: G^\vee\backslash\{(x,\phi)\ |\ x\in G^\vee, \phi\in\widehat{A_x}\}\longleftrightarrow&\bigsqcup_{G'\in\InnT^p(G)}\Irr_{\un}(G')\\
    (x,\phi)\hspace{1.5cm}\longmapsto \hspace{1.5cm}&\quad(\pi_{(x,\phi)},V_{(x,\phi)}),
\end{align*}
a result that parametrizes unipotent representations of $G$ in terms of the geometry of its complex dual group $G^\vee$ and lies in the heart of the local Langlands correspondence. 

In order to verify the \textcolor{red}{conjecture from the introduction} for the $p$-adic group $G$, one needs to explicitly compute the parahoric restriction maps
\[
    \res_{\un}^{J}:R_{\un}(G)\longrightarrow R_{\un}(\overline{K}_J), \quad V\longmapsto V^{U_J}
\]
from unipotent representations of $G$ to unipotent representations of $\overline{K}_J$. The aim of this section is to describe a general approach to compute the parahoric restriction maps above. These methods are quite complex and subtle, and involve deep mathematics -- the difficulty resides in the fact that the parametrization of the unipotent representations of $p$-adic groups (given by Langlands parameters) and that of finite groups of Lie type (given by Lusztig's labels) are not related in an obvious way.

%For now, we will restrict our attention to Iwahori-spherical representations $V_{(x,\phi)}$ of $G$, since the approach differs significantly for other unipotent representations that are not Iwahori-spherical. 
To do this, we closely follow the discussion in \textcolor{red}{[Re2000,\S4,5,6,8]}. Let us fix some subset $J\subsetneq S_{\aff}$ and a unipotent representation $V=V_{(x,\phi)}$ of $G$, with $x\in G^\vee$ and $\phi\in\widehat{A_x}$. We want to decompose the $\overline{K}_J$-module $V^{U_J}$ as a direct sum of irreducible submodules. More concretely, for each irreducible $\overline{K}_J$-representation $\chi$, we want to calculate the value of 
$$\langle\chi,V^{U_J}\rangle_{\overline{K}_J}=\langle\chi,V^{U_J}\rangle_{K_J},$$
where we view the $\overline{K}_J$-representations as $K_J$-representations trivial on $U_J$.

From Theorem \ref{thm:unip_restriction} and Corollary \ref{cor:disjoint_reps}, we deduce that all irreducible $\overline{K}_J$-constituents of $V^{U_J}$ lie in the same series representations. Such representations are in bijection with irreducible representations of some canonical associated reflection group. 
For example, if $V$ is Iwahori-spherical, then all irreducible $\overline{K}_J$-constituents of $V^{U_J}$ lie in the principal series. 
These representations are in bijection with irreducible representations of the Weyl group $W_J$ of $\overline{K}_J$ (see Section \ref{subsec:unipotent_families}), generated by the simple reflections corresponding to the simple affine roots in $J$. With this fact in mind, the method strategy becomes transparent: starting with the subset $J\subsetneq S_{\aff}$ and the unipotent representation $V$, we construct a representation over the associated reflection group whose direct sum into irreducible representations is compatible with the decomposition of $V^{U_J}$ into $K_J$-irreducible representations under the bijection stated above. 
This method consists in two major reduction steps; firstly, a reduction to Hecke algebras modules and, secondly, a reduction to modules over affine Weyl groups.

For each irreducible $K_J$-representation $\chi$ trivial on $U_J$, we want to calculate the value of $\langle\chi,V^{U_J}\rangle_{K_J}$. Firstly, by the well-known results of Harish--Chandra, there is some parahoric subgroup $(K,U_K,\overline{K})$ contained in $(K_J,U_J,\overline{K}_J)$ and cuspidal representation $\sigma$ of $K$ trivial on $U_K$ such that 
$$\chi^\sigma:=\Hom_K(\sigma,\chi)=\Hom_{K_J}(\Ind_K^{K_J}\sigma,\chi)\neq 0.$$
The space $\chi^\sigma$ is a naturally a $\cH(K_J,K,\sigma)$-module, where $\cH(K_J,K,\sigma)\cong\End_{K_J}(\Ind_K^{K_J}\sigma)$ is the subalgebra of functions of $\cH(G,K,\sigma)$ supported on $K_J$. On the other hand, the vector space
$$V^\sigma:=\Hom_K(\sigma,V)$$
is naturally an irreducible $\cH(G,K,\sigma)$-module and therefore a (potentially reducible) $\cH(K_J,K,\sigma)$-module by restriction.
\begin{lemma}\label{lem:multiplicity}[First reduction]
    The multiplicity of the simple $K_J$-module $\chi$ in $V^{U_J}$ is given by 
    $$\langle\chi,V^{U_J}\rangle_{K_J}=\langle\chi^\sigma,V^\sigma\rangle_{\cH(K_J,K,\sigma)}.$$
\end{lemma}
\begin{proof}
    Since $U_J\subseteq U$, we have that
    \begin{align*}
        V^\sigma=\Hom_K(\sigma,V)&=\Hom_K(\sigma,V^{U_J})\simeq\Hom_{K_J}(\Ind_K^{K_J}\sigma,V^{U_J})\\
        &\simeq\bigoplus_\eta\langle\eta,V^{U_J}\rangle_{K_J}\Hom_{K_J}(\Ind_K^{K_J}\sigma,\eta)\\
        &\simeq\bigoplus_\eta\langle\eta,V^{U_J}\rangle_{K_J}\eta^\sigma,
    \end{align*}
    and therefore
    \begin{equation*}
        \langle\chi^\sigma,V^\sigma\rangle_{\cH(K_J,K,\sigma)}=\sum_\eta\langle\chi^\sigma,\eta^\sigma\rangle_{\cH(K_J,K,\sigma)}\langle\eta,V^{U_J}\rangle_{K_J}=\langle\chi,V^{U_J}\rangle_{K_J},
    \end{equation*}
    as desired.
\end{proof}

For example, if $V$ is Iwahori-spherical of $G$, all irreducible components $\chi$ of $V^{U_J}$ are principal series representations, so $\chi^I=\Hom_I(\mathbf{1},\chi)\neq 0$.
\iffalse and let $\chi$ be an irreducible $K_J$-representation trivial on $U_J$ such that 
$$\langle\chi,V^{U_J}\rangle_{K_J}\neq 0.$$
Let $(K,U_K,\overline{K})$ and $\sigma$ be as above, so that $$\chi^\sigma=\Hom_K(\sigma,\chi)\neq 0.$$
Since $\sigma$ is irreducible and trivial on $U_K$ it follows by composing both conditions that
$$\langle\sigma,V^{U_K}\rangle_{K}=\langle\sigma,V^{U_J}\rangle_{K}\neq 0.$$
However, $V$ is assumed to be Iwahori-spherical, so by Theorem \ref{thm:unip_restriction} we have that $K=\cI$ is the Iwahori subgroup and $\sigma=\mathbf{1}$ is the trivial representation.\fi
Lemma \ref{lem:multiplicity} states that 
$$\langle\chi,V^{U_J}\rangle_{K_J}=\langle\chi^\cI,V^\cI\rangle_{\cH(K_J,\cI,\mathbf{1})}.$$

Now let us consider the second reduction step. For this one, we let $R=\CC(v,v^{-1})$, where $v$ is an indeterminate. We then define $\cH(G,K,\sigma)_v$ to be the \textit{generic Hecke algebra} defined over $R$ with the same generators and relations as $\cH(G,K,\sigma)$, but with $q$ replaced by $v^2$. The upshot of considering this generic Hecke algebra is that by specializing $v$ we can recover
$$\cH(G,K,\sigma)_{\sqrt{q}}=\cH(G,K,\sigma),\quad\quad \cH(G,K,\sigma)_1=\CC[\tilde{W}(K,\sigma)].$$
When $K=\mathcal{I}$ is the Iwahori subgroup and $\sigma=\mathbf{1}$ is the trivial representation, $\mathcal{H}(G,\mathcal{I},\mathbf{1})=\CC[\widetilde{W}]$, where $\widetilde{W}$ is the extended affine Weyl group.

\textbf{Fact:} For any simple $\cH(G,K,\sigma)$-module $E$ \textit{considered in this document}, there is a $\cH(G,K,\sigma)_v$-module $E_v$ such that 
\[
    E\simeq E_v\otimes_R\CC, \text{ where }f\in R\text{ acts on $\CC$ by }f(\sqrt{q}).
\]
It then follows that we have a $q=1$ operation that takes simple modules over Hecke algebras to modules over the corresponding Weyl groups, obtained by setting $v=1$ in all matrix coefficients of the generic module.

\begin{proposition}[Second reduction]\label{prop:second_red}
    Let $J\subsetneq S_{\aff}$ and let $(K,U_K,\overline{K})$ be a parahoric subgroup contained in $(K_J,U_J,\overline{K}_J)$ with cuspidal unipotent $\overline{K}$-representation $\sigma$. Then the diagram 
    \[
        \begin{tikzcd}
        \cH(G,K,\sigma)\textrm{-mod} \arrow[r, "q=1"] \arrow[d, "\Res_{\cH(K_J,K,\sigma)}"'] & \tilde{W}(K,\sigma)\textrm{-mod} \arrow[d, "\Res_{\tilde{W}(K,\sigma)_J}"] \\
        \cH(K_J,K,\sigma)\textrm{-mod} \arrow[r, "q=1"] & \tilde{W}(K,\sigma)_J\textrm{-mod}
        \end{tikzcd}
    \]
    is commutative, and the bottom arrow is an isometry with respect to the usual inner product of character. That is, for any irreducible $K_J$-module $\chi$, we have that 
    \begin{equation}
        \langle\chi^\sigma,V^\sigma\rangle_{\cH(K_J,K,\sigma)}=\langle\chi^\sigma_{q=1},V^\sigma_{q=1}\rangle_{\tilde{W}(K,\sigma)_J}.
    \end{equation}
\end{proposition}
\begin{proof}
\end{proof}


Combining both reduction steps, we obtain the identity
$$\langle\chi,V^{U_J}\rangle_{\overline{K}_J}=\langle\chi^\sigma,V^\sigma\rangle_{\mathcal{H}(K_J,K,\sigma)}=\langle\chi^\sigma_{q=1},V^\sigma_{q=1}\rangle_{\tilde{W}(K,\sigma)_J}$$
Thus, the $\tilde{W}(K,\sigma)_J$-module $V_{q=1}^\sigma$ satisfies the desired properties. By calculating its decomposition into $\tilde{W}(K,\sigma)$-irreducible representation, one can directly obtain the decomposition of $V^{U_J}$ into irreducible $\overline{K}_J$-representations. 

At this point, we restrict our attention to the parahoric restriction of Iwahori-spherical representations, for the method we use to calculate the $K(I,\mathbf{1})_J$-module $V^\mathbf{1}_{q=1}$ differs significantly from the non Iwahori-spherical case. As stated above, for Iwahori-spherical representations, the reflection subgroup $K(I,\mathbf{1})\cong\widetilde{W}$ is simply the extended affine Weyl group of $G$, and $\widetilde{W}_J$ is the subgroup of $\widetilde{W}$ generated by the simple reflections of the roots in $J$.

\subsection{The Iwahori-spherical case and Springer Correspondence}
As indicated in the previous paragraph, we now fix an Iwahori-spherical representation $V=V_{(x,\phi)}$ of $G$ and some proper subset $J\subsetneq S_{\aff}$. The two reduction steps described above imply that for any principal series representation $\chi$ of $\overline{K}_J$, 
\begin{equation}\label{eqn:multiplicity_Iwahori}
    \langle\chi,V^{U_J}\rangle_{\overline{K}_J}=\langle\chi^\mathbf{1}_{q=1},V^\mathbf{1}_{q=1}\rangle_{\widetilde{W}_J}.
\end{equation}

When $V_{(x,\phi)}$ is Iwahori-spherical, we can explicitly describe the structure of the $\widetilde{W}_J$-module $V^\mathbf{1}_{q=1}$ explicitly in terms of the geometry of the dual group $G^\vee$ and the data $(x,\phi)$. In order to extract this information, Springer theory is the key theoretical tool that we shall require. To explain this, consider the Jordan decomposition $x=us$, with commuting $u\in G^\vee$ unipotent and $s\in G^\vee$ semisimple. Let $\mathcal{B}$ and $\mathcal{B}_s$ be the flag varieties of the complex reductive groups $G^\vee$ and $G^\vee_s:=Z_{G^\vee}(s)$, respectively. These are algebraic varieties parametrizing the set of Borel subgroups in $G^\vee$ and $G^\vee_s$, respectively, and admit a rational action by conjugation. Noting that $u\in G^\vee_s$, we consider the stabilizers $\mathcal{B}^x$ and $\mathcal{B}_s^u$, called partial flag varieties. These are algebraic subvarieties parametrizing the set of Borel subgroups of $G^\vee$ containing $x$ and the set of Borel subgroups of $G^\vee_s$ containing $u$. 

The varieties $\mathcal{B}_x$ and $\mathcal{B}_s^u$ admit a natural action of $A_x$ induced by the action of $Z_{G^\vee}(x)$ by conjugation. On the other hand, these varieties do not admit a natural action of the Weyl groups $W$ and $W_s$ of $G^\vee$ and $G^\vee_s$, respectively. To obtain representations over the Weyl groups, one needs to consider the singular cohomology spaces $H^*(\mathcal{B}_x)$ and $H^*(\mathcal{B}_s^u)$. These spaces admit not only an $A_x$-action inherited from the action on the partial flag varieties, but also a natural representation of Weyl groups $W$ and $W_s$, respectively. Moreover, both actions commute. These last two statements are at the heart of Springer theory, which we shall use continuously from now on. The mathematics behind these results involve sophisticated geometric machinery, so for the most part we will state the results we need without proof. At the top of Springer theory there is the Springer correspondence, which we now state for convenience. 

\begin{theorem}
    Let $G^\vee$ be a complex algebraic reductive group with Weyl group $W$. For each pair $(u,\phi)$, where $u\in G^\vee$ is a unipotent element and $\phi$ is an irreducible character of $A_u$, the $\phi$-isotropic subspace of the top cohomology group $H^{\mathrm{top}}(\mathcal{B}^u)^\phi$ is either trivial or an irreducible $W$-representation. Moreover, each irreducible character of $W$ arises this way for exactly one pair $(u,\phi)$ up to $G^\vee$ conjugation. In other words, there is an injection
    $$\Irr(W)\xhookrightarrow{\quad\quad}\{\textrm{pairs }(u,\phi)\}/G^\vee.$$
    Finally, for any unipotent $u\in G^\vee$, the pair $(u,\mathbf{1})$ lies in the image of the map above.
\end{theorem}
This result is of great importance for it allows us to extract information from the top cohomology group directly. When $G^\vee$ is a classical group, one can give precise description of the injection in terms of combinatorial data. When $G^\vee$ is of exceptional type, one only has a finite amount of information and the injection above can be found in the literature (\textcolor{red}{See tables from Carter}). Another important fact that greatly simplifies calculations is the following:

\begin{theorem}\label{thm:odd_vanish}
    The odd-degree cohomology spaces of $\mathcal{B}^x$ vanish, so $H^*(\mathcal{B}^x)=\sum_{i=1}^{\dim\mathcal{B}^x}H^{2i}(\mathcal{B}^x)$.
\end{theorem}

\begin{example}
    If $G^\vee=\GL_n(\CC)$, by the Jordan decomposition theorem, unipotent conjugacy classes are parametrized by partitions of $n$ and $A_u=\{1\}$ for any unipotent element $u$. Moreover, $W\cong S_n$, and its irreducible representations are also labelled by partitions of $n$. If $\lambda$ is a partition of $n$, then the Springer correspondence maps $V_\lambda$ to $(u_\lambda,\mathbf{1})$, where $V_\lambda$ is the Specht module of $S_n$ and $u_\lambda$ is a unipotent element of $\GL_n(\CC)$ corresponding to $\lambda$. In particular, the Springer correspondence for $\GL_n(\CC)$ is a bijection.
    
    If $G^\vee=\SL_n(\CC)$, conjugacy classes of unipotent elements are still parametrized by partitions of $n$. This time, however, $A_u$ may be non-trivial for some unipotent classes. The Springer correspondence is therefore not a bijection, and the image of the injection are all pairs $\{(u,\mathbf{1}), u\in\SL_n(\CC)\textrm{ unipotent}\}$.

    When $G^\vee$ has type $A_n$, one can describe the entire cohomology complex $H^*(\mathcal{B}^u)^\mathbf{1}$ explicitly. Suppose that $u=u_\lambda$ for a partition $\lambda=(\lambda_1,\ldots,\lambda_r)$ of $n$. If $S_\lambda=S_{\lambda_1}\times\cdots\times S_{\lambda_r}\leq S_n$, then Lusztig and Spaltenstein showed
    $$H^*(\mathcal{B}^{u_\lambda})\cong\Ind_{S_\lambda}^{S_n}\mathbf{1}$$
    as graded $S_n$-modules.
\end{example}

\begin{example}
    The complex reductive group $G^\vee=G_2(\CC)$ has Weyl group $W\cong D_6$ and $5$ unipotent classes labelled and partially ordered by 
    $$1<A_1<\tilde{A}_1<G_2(a_1)<G_2.$$
    The component groups are all trivial except the subregular unipotent class, whose component group is isomorphic to $S_3$, with irreducible representations $\mathbf{1},\varepsilon$ and $r$ (the unique $2$-dimensional).
    The characters of $W$ are labelled in the literature by the symbols $\phi_{1,0}$ (trivial), $\phi_{1,6}$ (sign), $\phi_{1,3}',\phi_{1,3}''$ (the other two characters), $\phi_{2,1}$ (faithful), $\phi_{2,2}$ (lifted from $S_3$). Then the Springer correspondence gives the pairing 
    $$\phi_{1,0}\leftrightarrow(G_2,\mathbf{1}),\quad\phi_{2,1}\leftrightarrow(G_2(a_1),\mathbf{1}),\quad\phi_{1,3}'\leftrightarrow(G_2(a_1),r),\quad\phi_{2,2}\leftrightarrow(\tilde{A}_1,\mathbf{1}),\quad\phi_{1,3}''\leftrightarrow(A_1,\mathbf{1}),\quad\phi_{1,6}\leftrightarrow(1,\mathbf{1}),$$
    and the pair $(G_2(a_1),\varepsilon)$ is not in the image of the correspondence.
\end{example}

Having explained the Springer correspondence, we can go back to our initial discussion. We now know that $H^*(\mathcal{B}^x)$ has the structure of a $A_x\times W$-module. We can extend this to a $A_x\times\widetilde{W}$-action as follows. The extended Weyl group can be decomposed as a semidirect product $\widetilde{W}=W\ltimes X$, where $X$ is the character lattice of $T^\vee$. Then there is a natural evaluation pairing $X\rightarrow \CC^\times$ given by $\mu\mapsto\langle s,\mu\rangle$. The character lattice $X$ then acts on $H^*(\mathcal{B}^x)$ by scalars $\langle s,\cdot\rangle$, and this extends the action as desired. Analogously, the $A_x\times W_s$-action of $H^*(\mathcal{B}^u_s)$ can be extended to a $A_x\times\widetilde{W}_s$-action. The main result we need is due to Lusztig.

\begin{theorem}\label{thm:Kato_module}
    Let $V_{x,\phi}$ be an Iwahori-spherical irreducible representation of $G$. After letting $q\rightarrow 1$ and then taking semisimplification, the simple $\mathcal{H}(G,I,\mathbf{1})$-module $V^\mathbf{1}_{x,\phi}$ becomes the $\widetilde{W}$-module $\varepsilon\otimes H^*(\mathcal{B}^x)^\phi$. 
\end{theorem}

\begin{remark}\label{rem:Weylgroups}
    An important remark is due at this point. The above theorem relates the structure of $V^\mathbf{1}_{(x,\phi),q=1}$ as a $\widetilde{W}$-module with the structure of $H^*(\mathcal{B}^x)^\phi$ as a $\widetilde{W}$-module too. However, the former object is naturally $p$-adic, while the latter is complex analytic. The Weyl groups of $G$ and $G^\vee$ are canonically isomorphic, but not equal, since, for example, the isomorphism swaps short with long roots when the group is not simply laced. It is a standard abuse of notation to denote both Weyl groups with the same symbol, but one needs to be careful during explicit calculations inside which Weyl group one is working. Of course, we are ultimately interested in the $p$-adic side, but since we will be computing the cohomology groups, we will mostly work inside the complex side. In particular, the reflection $s_0$ inside $W$ will be a \textbf{short} reflection, corresponding to the highest short root. At the very end, we then trace the representations back to the $p$-adic side. 
\end{remark}

This theorem is the key theoretical tool that allows the use of Springer theory to compute parahoric restrictions of unipotent representations of $G$. However, it is not immediately clear how to use Springer theory yet, since $x\in G^\vee$ need not be a unipotent element. Naturally, the solution comes from relating $H^*(\mathcal{B}^x)$ to $H^*(\mathcal{B}^u_s)$. The result is due to Kato.

\begin{proposition}\label{prop:ind_cohomo}
    The natural restriction map $H^*(\mathcal{B}^x)^\phi\rightarrow H^*(\mathcal{B}_s^u)^\phi$ is $\widetilde{W}_s$-equivariant, and induced an isomorphism of $\widetilde{W}$-modules
    $$H^*(\mathcal{B}^x)^\phi\cong\Ind_{\widetilde{W}_s}^{\widetilde{W}} H^*(\mathcal{B}_s^u)^\phi.$$
\end{proposition}

By \eqref{eqn:multiplicity_Iwahori}, Theorem \ref{thm:Kato_module} and Proposition \ref{prop:ind_cohomo}, the restriction of $V_{x,\phi}$ can be determined by computing the restriction
\begin{equation*}
    \left(\Ind_{\widetilde{W}_s}^{\widetilde{W}}H^*(\mathcal{B}^u_s)^\phi\right)|_{\widetilde{W}_J},
\end{equation*}
where $\widetilde{W}_J$ is the subgroup of $\widetilde{W}$ generated by all simple reflections of the roots in $J$. Working inside $\widetilde{W}$ is very delicate, so following \textcolor{red}{Reeder's development}, we decide to work inside $W=\widetilde{W}_{J_0}$. The downside is the fact that, even though $W$ contains an isomorphic copy of $\widetilde{W}_J$, one needs to carefully track this isomorphism since it can twist the original representation by some character. 

More concretely, consider the image $W_J$ of the group $\widetilde{W}_J$ under the natural projection map $\widetilde{W}\twoheadrightarrow W$. This restricts to an isomorphism $\widetilde{W}_J\cong W_J$ with inverse map $\psi_J:W_J\rightarrow \widetilde{W}_J$, satisfying $\Psi_J(s_\alpha)=s_\alpha$ if $\alpha\in J, \alpha\neq-\alpha_0$ and $\psi_J(s_0)=\tilde{s}_0=t_{\alpha_0}s_0$ if $\tilde{s}_0\in J$. Let $\psi_J^*$ the pullback of representation of $\widetilde{W}_J$ to $W_J$.

Given a semisimple element $t\in G^\vee$, let $W_{J,t}=W_t\cap W_J$ and define the character
\begin{equation}
    \chi_t^J:=\chi_t\circ\psi_J:W_{J,t}\longrightarrow\CC^\times.
\end{equation}

Then by Mackey theory we obtain the isomorphism
\begin{equation}\label{eqn:mackey_easycase}
    \psi_J^*\left[\left(\Ind_{\widetilde{W}_s}^{\widetilde{W}}H^*(\mathcal{B}^u_s)^\phi\right)|_{\widetilde{W}_J}\right]\cong\bigoplus_{w\in W_s\backslash W/W_J}\Ind_{W_{J,s^w}}^{W_J}\chi_{s^w}^J\otimes\left[H^*(\mathcal{B}^u_s)^\phi\right]^w.
\end{equation}
Springer theory tells us the structure of the cohomology groups, so it remains to understand the groups $W_{J,t}$ and the characters $\chi_t^J$. The next result shows that, in many cases, the character is trivial. To state it, we note that, associated to $J$, there is a pseudo-Levi subgroup $G^\vee_J$ of $G^\vee$, with simply connected cover $$G^{\vee,\mathrm{sc}}_J\longrightarrow G^\vee_J$$
whose kernel we denote by $Z_J$.
\begin{lemma}
    If the orders of $s$ and $Z_J$ are relatively prime, then $\chi_{s^w}^J$ is trivial on $W_{J,s^w}$ for all $w\in W_s\backslash W/W_J$.
\end{lemma}
In general, these characters have been studied by Reeder (add citation here), where he gives an explicit formula. All nontrivial characters in this project have been calculated using that formula. 

The following easy lemma shows that if the characters $\chi_t^{J}$ are trivial, then computations are greatly simplified.
\begin{lemma}\label{lem:mackey_restriction}
    With the same notation as in \eqref{eqn:mackey_easycase}, if the characters $\chi_{s^w}^{J}$ are trivial for all $w\in W_s\backslash W/W_J$, then
    $$\psi_J^*\left[\left(\Ind_{\widetilde{W}_s}^{\widetilde{W}}H^*(\mathcal{B}^u_s)^\phi\right)|_{\widetilde{W}_J}\right]\cong\left(\Ind_{W_s}^W H^*(\mathcal{B}^u_s)^\phi\right)|_{W_J}$$
\end{lemma}
\begin{proof}
    This is an immediate consequence of Mackey's formula applied to the right hand side.
\end{proof}

\subsection{Parahoric restriction for type \texorpdfstring{$G_2$}{PDFstring}}
To finish this chapter, we finish with a worked example of the above methods by computing some parahoric restrictions for the $p$-adic group $G=G_2(F)$. We will restrict our attention to classes of maximal parahoric subgroups, for the other ones can be easily deduced from these. The group $G_2(F)$ has simple roots $\{-\beta_0,\beta_1,\beta_2\}$ with Dynkin diagram 
\[
    0 \relbar 1 \xRightarrow{\quad} 2.
\]
Thus $G_2(F)$ has $3$ classes of maximal parahoric subgroups $K_0$, $K_1$, $K_2$ corresponding to the subsets of affine simple roots $J_0=\{\beta_1,\beta_2\}, J_1=\{-\beta_0,\beta_2\}$ and $J_2=\{-\beta_0,\beta_1\}$ and with reductive types $G_2$, $A_1\times \tilde{A}_1$ and $A_2$, respectively. We now compute these parahoric restrictions for all representations $\pi(us,\phi)$ of $G=G_2(F)$, where $u$ is a subregular unipotent element of $G_2(\CC)$ labelled by $G_2(a_1)$. This is an interesting case since $u$ is a distinguished unipotent element and $\Gamma_u=A_u\cong S_3$. There are $8$ such representations, given by 
$$\pi(G_2(a_1),1,\phi),\phi\in\{\mathbf{1},\varepsilon,r\},\quad\pi(G_2(a_1),g_2,\phi),\phi\in\{\mathbf{1},\varepsilon\},\quad\pi(G_2(a_1),g_3,\phi),\phi\in\{\mathbf{1},\theta,\theta^2\}.$$
It turns out that $4$ of them are unipotent supercuspidal representations of $G_2(F)$ given by 
$$\pi(G_2(a_1),1,\epsilon)=\cInd_{K_0}^{G}G_2[1],\quad\pi(G_2(a_1),g_2,\epsilon)=\cInd_{K_0}^{G}G_2[-1],\quad\pi(G_2(a_1),g_3,\theta^k)=\cInd_{K_0}^{G}G_2[\theta^k], k=1,2,$$
while the remaining $4$ are Iwahori-spherical. 
For the supercuspidal representations, their restrictions can be easily computed. By Frobenius reciprocity, their parahoric restrictions to $K_0$ is irreducible. The restrictions to $K_1$ and $K_2$ are both $0$, since all representations of $A_1\times \tilde{A}_1$ and $A_2$ are principal series. 

Thus, we can focus our attention to Iwahori-spherical representations. The computations are quite lenghty, so we give a complete sketch while omitting some unenlightening steps. We recall that we are working inside the \textit{complex} Weyl group $W$ and consequently $s_0$ is seen as a short reflection along the highest short root, and we denote by $s^0$ the reflection along the highest root. We first note that $u$ is a subregular unipotent element in $G^\vee=G_2(\CC)$, while it is a regular unipotent element in $G^\vee_{g_2}=\GL_2(\CC)$ and $G^\vee_{g_3}=\SL_3(\CC)$. Therefore, $\dim\mathcal{B}^u=1$, while $\dim\mathcal{B}_{g_2}^u=\dim\mathcal{B}_{g_3}^u=0$. By Springer theory, we obtain that 
$$H^*(\mathcal{B}^u)^\mathbf{1}=H^0(\mathcal{B}^u)^\mathbf{1}+H^2(\mathcal{B}^u)^\mathbf{1}=\phi_{1,0}+\phi_{2,1},\quad\text{and}\quad H^*(\mathcal{B}^u)^r=H^0(\mathcal{B}^u)^r+H^2(\mathcal{B}^u)^r=\phi_{1,3}',$$
as $W$-modules, while $H^0(\mathcal{B}_{g_2}^u)$ and $H^0(\mathcal{B}_{g_3}^u)$ afford the trivial representations of $W_{g_2}$ and $W_{g_3}$ respectively, and
$$\Ind_{W_{g_2}}^W\mathbf{1}=\phi_{1,0}+\phi_{1,3}''\quad\text{and}\quad\Ind_{W_{g_3}}^W\mathbf{1}=\phi_{1,0}+\phi_{2,2}.$$

To obtain the parahoric restrictions to $K_0$, we note that all characters $\chi_t^{J_0}$ so it remains to twist by $\varepsilon=\phi_{1,6}$ (swapping $\phi_{1,0}\leftrightarrow\phi_{1,6}$ and $\phi_{1,3}'\leftrightarrow\phi_{1,3}''$) and then by the outer automorphism of $W$ exchanging short and long roots (swapping $\phi_{1,3}'\leftrightarrow\phi_{1,3}''$ only). 

To compute the other two columns of Table \ref{tbl:G2_restriction} we compute the remaining characters $\chi_t^J$. Using their properties above, it is not hard to show that all characters we are considering are trivial except for $\chi_{g_2^w}^{J_1}$ and $\chi_{g_3^w}^{J_2}$, and we can use directly Lemma \ref{lem:mackey_restriction} to compute the restrictions. The last cases must be dealt with separately.
\begin{itemize}
    \item \textbf{Restriction of $\pi(G_2(a_1),g_3,1)$ to $K_{J_2}$.} In this case, $W_{J_2}=\langle s_0,s_1\rangle$ (where $s_0$ is the reflection along the highest short root) and $W_{g_3}=\langle s^0,s_2\rangle$ (where $s^0$ is the reflection along the highest root). Then $W_{g_3}\backslash W/W_{J_2}=\{1\}$, $W_{J_2,g_3}\cong C_3$ and $\chi_{g_3}^{J_2}$ is a primitive character. Thus, 
    $$\Ind_{W_{J_2,g_3}}^{W_{J_2}}\chi_{g_3}^{J_2}=r.$$ 
    \item \textbf{Restriction of $\pi(G_2(a_1),g_2,1)$ to $K_{J_1}$.} In this case, $W_{J_1}=\langle s_0,s_2\rangle$ and $W_{g_2}=\langle s^0,s_1\rangle$. Then $W_{g_2}\backslash W/W_{J_1}=\{1,w\}$ has two elements, and $W_{J_1,g_2}\cong C_2$ with $\chi_{g_2}^{J_1}$ non-trivial and $W_{J_1,g_2^w}=W_{J_1}$ with $\chi_{g_2^w}^{J_1}$ the trivial character. Thus,
    $$\Ind_{W_{J_1,g_2}}^{W_{J_1}}\chi_{g_2}^{J_1}+\Ind_{W_{J_1,g_2^w}}^{W_{J_1}}\chi_{g_2^w}^{J_1}=\varepsilon\boxtimes\mathbf{1}+\mathbf{1}\boxtimes\varepsilon+\mathbf{1}\boxtimes\mathbf{1}.$$ 
\end{itemize}

\begin{figure}[ht]
    \centering
    \includegraphics[scale=0.7]{subregular_element.png}
    \caption{Restrictions of $G_2$-representations $\pi(G_2(a_1),s,\phi)$}
    \label{tbl:G2_restriction}
\end{figure}

The proof of Conjecture \ref{conj:main} is now just a matter of putting all the pieces together. We can now compute the the parahoric restriction of the virtual characters $\Pi(u,s,h)$. The results are in the table below. Lusztig's nonabelian Fourier transform acts trivially on every representation, except for the family 
$$\{\phi_{2,1},\phi_{2,2},\phi_{1,3}',\phi_{1,3}'',G_2[1],G_2[-1],G_2[\theta],G_2[\theta^2]\}.$$
The action of Lusztig's nonabelian fourier transform on this family is given by the matrix \ref{fig:S3matrix}, whose labelling is explained in Figure \ref{fig:G2_labels}. It is then a routine calculation to see that the square in Conjecture \ref{conj:main} commutes.
\begin{table}[ht]
    \centering
    \begin{tabular}{|c|c|c|c|}
        \hline
        & $K_0=G_2$ & $K_1=A_1\times\tilde{A}_1$ & $K_2=A_2$ \\\hline
        $\Pi(G_2,1,1)$ & $\phi_{1,6}$ & $\epsilon\boxtimes\epsilon$ & $\epsilon$ \\\hline
        $\Pi(G_2(a_1),1,1)$ & $\phi_{2,1}+G_2[1]+2\phi_{1,3}'+\phi_{1,6}$ & $\mathbf{1}\boxtimes\epsilon+3\epsilon\boxtimes\mathbf{1}+\epsilon\boxtimes\epsilon$ & $3\epsilon+r$ \\\hline
        $\Pi(G_2(a_1),1,g_2)$ & $\phi_{2,1}-G_2[1]+\phi_{1,6}$ & $\mathbf{1}\boxtimes\epsilon+\epsilon\boxtimes\mathbf{1}+\epsilon\boxtimes\epsilon$ & $\epsilon+r$ \\\hline
        $\Pi(G_2(a_1),g_2,1)$ & $\phi_{2,1}+\phi_{2,2}+G_2[-1]+\phi_{1,6}$ & $\mathbf{1}\boxtimes\epsilon+\epsilon\boxtimes\mathbf{1}+\epsilon\boxtimes\epsilon$ & $\epsilon+r$ \\\hline
        $\Pi(G_2(a_1),1,g_3)$ & $\phi_{2,1}+G_2[1]-\phi_{1,3}'+\phi_{1,6}$ & $\mathbf{1}\boxtimes\epsilon+\epsilon\boxtimes\epsilon$ & $r$ \\\hline
        $\Pi(G_2(a_1),g_3,1)$ & $\phi_{1,3}''+G_2[\theta]+G_2[\theta^2]+\phi_{1,6}$ & $\mathbf{1}\boxtimes\epsilon+\epsilon\boxtimes\epsilon$ & $r$ \\\hline
        $\Pi(G_2(a_1),g_2,g_2)$ & $\phi_{2,1}+\phi_{2,2}-G_2[-1]+\phi_{1,6}$ & $\mathbf{1}\boxtimes\epsilon+\epsilon\boxtimes\mathbf{1}+\epsilon\boxtimes\epsilon$ & $\epsilon+r$ \\\hline
        $\Pi(G_2(a_1),g_3,g_3)$ & $\phi_{1,3}''+\theta G_2[\theta]+\theta^2 G_2[\theta^2]+\phi_{1,6}$ & $\mathbf{1}\boxtimes\epsilon+\epsilon\boxtimes\epsilon$ & $r$ \\\hline
        $\Pi(G_2(a_1),g_3,g_3^{-1})$ & $\phi_{1,3}''+\theta^2 G_2[\theta]+\theta G_2[\theta^2]+\phi_{1,6}$ & $\mathbf{1}\boxtimes\epsilon+\epsilon\boxtimes\epsilon$ & $r$ \\\hline
    \end{tabular}
\end{table}
